%----------------------------------------------------------------------------------------
%	CONFIGURACIÓN DEL DOCUMENTO
%----------------------------------------------------------------------------------------

\documentclass[11pt, letterpaper]{report}

%--- PAQUETES ---
\usepackage[utf8]{inputenc}
\usepackage[spanish, es-tabla]{babel}
\unaccentedoperators
\usepackage{geometry}
\usepackage{graphicx}
\usepackage{amsmath}
\usepackage{amsthm}
\usepackage{amssymb}
\usepackage{amsfonts}
\usepackage{hyperref}
\usepackage{xcolor}
\usepackage{listings}
\usepackage{fancyhdr}
\usepackage{microtype}
\usepackage{parskip}
\usepackage{lipsum}
\usepackage{wrapfig}
\usepackage{caption}
\usepackage{float}
\usepackage{physics}
\usepackage{tikz}
\usepackage{quantikz}
\usetikzlibrary{babel}
\usepackage{booktabs}

% --- Definición del entorno 'teorema' ---
\theoremstyle{definition} 
\newtheorem{teorema}{Teorema}[section] 

%--- CONFIGURACIÓN DE GEOMETRÍA ---
\geometry{
    letterpaper,
    top=2.5cm,
    bottom=2.5cm,
    left=2.5cm,
    right=2.5cm
}

%--- CONFIGURACIÓN DE HIPERVÍNCULOS ---
\hypersetup{
    colorlinks=true,
    linkcolor=black,
    filecolor=magenta,      
    urlcolor=cyan,
    citecolor=blue,
    pdftitle={Reporte de Prácticas Profesionales - Jorge Toral},
    pdfpagemode=FullScreen,
}

%--- CONFIGURACIÓN PARA EL CÓDIGO FUENTE ---
\definecolor{codegreen}{rgb}{0,0.6,0}
\definecolor{codegray}{rgb}{0.5,0.5,0.5}
\definecolor{codepurple}{rgb}{0.58,0,0.82}
\definecolor{backcolour}{rgb}{0.95,0.95,0.92}

\lstdefinestyle{mystyle}{
    backgroundcolor=\color{backcolour},   
    commentstyle=\color{codegreen},
    keywordstyle=\color{magenta},
    numberstyle=\tiny\color{codegray},
    stringstyle=\color{codepurple},
    basicstyle=\footnotesize\ttfamily,
    breakatwhitespace=false,         
    breaklines=true,                 
    captionpos=b,                    
    keepspaces=true,                 
    numbers=left,                    
    numbersep=5pt,                  
    showspaces=false,                
    showstringspaces=false,
    showtabs=false,                  
    tabsize=2
}

\lstset{
    style=mystyle,
    inputencoding=utf8,
    extendedchars=true,
    literate={á}{{\'a}}1 {é}{{\'e}}1 {í}{{\'i}}1 {ó}{{\'o}}1 {ú}{{\'u}}1 {ñ}{{\~n}}1 {Á}{{\'A}}1 {É}{{\'E}}1 {Í}{{\'I}}1 {Ó}{{\'O}}1 {Ú}{{\'U}}1 {Ñ}{{\~N}}1 {¿}{{?`}}1 {¡}{{!`}}1
}

%--- ENCABEZADO Y PIE DE PÁGINA ---
\pagestyle{fancy}
\fancyhf{} % Limpia todos los campos
\fancyhead[L]{Análisis de Datos enfocado a física de partículas}
\fancyhead[R]{\thepage}
\fancyfoot[C]{Jorge Luis Toral Gamez} % Reemplaza con tu nombre


%----------------------------------------------------------------------------------------
%	INICIO DEL DOCUMENTO
%----------------------------------------------------------------------------------------

\begin{document}

%----------------------------------------------------------------------------------------
%	PORTADA
%----------------------------------------------------------------------------------------

\begin{titlepage}
    \centering
    % \includegraphics[width=0.3\textwidth]{logo_buap.png} % Descomenta si tienes el logo
    \vspace*{1cm}
    {\LARGE \textbf{Benemérita Universidad Autónoma de Puebla} \par}
    \vspace{0.5cm}
    {\Large Facultad de Ciencias Físico Matemáticas \par}
    \vspace{2cm}
    {\Large Reporte Final de Prácticas Profesionales \par}
    \vspace{0.5cm}
    {\huge \textbf{Análisis de Datos enfocado a física de partículas} \par}
    \vspace{0.5cm}
    {\large Implementación de Algoritmos Cuánticos y Modelos de IA \par}
    \vspace{4.5cm}
    {\Large Presentado por: \par}
    {\Large \textbf{Jorge Luis Toral Gamez} \par}
    \texttt{jorge.toralga@alumno.buap.mx}
    \vspace{3cm}
    {\large Enero 2026 \par}
\end{titlepage}

%--- RESUMEN ---

\chapter*{Resumen} 
\addcontentsline{toc}{chapter}{Resumen}

Este documento reporta las actividades de investigación y desarrollo tecnológico 
realizadas durante las prácticas profesionales en el Centro Interdisciplinario de 
Investigación y Enseñanza de la Ciencia (CIIEC) de la BUAP, abarcando el periodo 
del 6 de enero al 6 de julio de 2025. El trabajo se alineó con los desafíos de 
computación científica propuestos por la organización \textit{Machine Learning 
for Science} (ML4SCI), integrando métodos de Inteligencia Artificial y Computación 
Cuántica aplicados a la física de altas energías.

El desarrollo técnico se estructuró en tres ejes fundamentales. En primer lugar, 
se abordó la computación cuántica mediante la implementación y simulación del 
\textbf{Algoritmo de Shor}, analizando críticamente su capacidad para la búsqueda 
de periodos y sus implicaciones directas en la vulnerabilidad de la criptografía 
RSA actual.

En segundo lugar, se aplicaron técnicas de \textit{Geometric Deep Learning} para 
la clasificación de jets (quarks vs. gluones) utilizando el conjunto de datos 
ParticleNet. El estudio comparativo entre Redes de Paso de Mensajes (MPNN) y 
Redes de Atención en Grafos (GAT) reveló una disyuntiva técnica clave: mientras la 
arquitectura GAT maximiza la pureza de la selección (Alta Precisión), la MPNN 
ofrece una mayor eficiencia de recuperación de señales (Alto Recall), proporcionando 
criterios claros para su selección según el objetivo físico.

Finalmente, se evaluó el potencial de las arquitecturas emergentes mediante las 
\textbf{Redes Kolmogorov-Arnold (KAN)}. Los experimentos sobre el conjunto de 
datos MNIST demostraron que una arquitectura KAN-Híbrida alcanza una exactitud del 
88.03\%, superando a las variantes puras. El reporte culmina con la propuesta 
teórica de la \textbf{Quantum KAN (Q-KAN)}, una arquitectura diseñada para explotar 
la expresividad del espacio de Hilbert mediante circuitos variacionales, buscando 
mitigar los costos computacionales de los splines clásicos en la era NISQ.

\newpage


%--- ÍNDICE ---
\tableofcontents
\newpage

%----------------------------------------------------------------------------------------
%	CONTENIDO DEL REPORTE
%----------------------------------------------------------------------------------------
\part{Introducción}
\chapter{Introducción}

El presente documento detalla las actividades de investigación y desarrollo 
tecnológico realizadas durante el periodo de prácticas profesionales en el 
\textbf{Centro Interdisciplinario de Investigación y Enseñanza de la Ciencia, CIIEC,}
de la Benemérita Universidad Autónoma de Puebla (BUAP). Este periodo comprendió del 
6 de enero al 6 de julio de 2025.

Si bien el programa se tituló oficialmente «Análisis de Datos enfocado a física de 
partículas», la dinámica de trabajo y los intereses del grupo de investigación 
permitieron una evolución natural hacia áreas de vanguardia. Las actividades se 
centraron en el grupo de computación cuántica, explorando la intersección crítica 
entre el aprendizaje automático, \textit{Machine Learning}, la criptografía 
moderna y el diseño de algoritmos cuánticos para la física de altas energías (HEP).

El repositorio donde se encuentran los resultados de los proyectos realizados durante
las prácticas profesionales está disponible en:
\begin{center}
    \url{https://github.com/Jorge-1501/Practicas}
\end{center}

\section{Contexto: La convergencia entre IA y Física}
La física contemporánea enfrenta un desafío por la cantidad de datos generados. 
Los colisionadores, como el LHC, generan petabytes de información que los 
métodos tradicionales de análisis tardan años en procesar. En este contexto, la 
Inteligencia Artificial, IA, se ha convertido en un pilar fundamental para el 
descubrimiento científico.

Paralelamente, la Computación Cuántica ha entrado en la era NISQ, \textit{Noisy 
Intermediate-Scale Quantum}. La promesa de esta tecnología no solo radica en la 
aceleración de cálculos, como se verá con el algoritmo de Shor; sino en su 
capacidad para modelar sistemas naturales de manera innata, como la mecánica cuántica. 
La fusión de estos campos, conocida como \textit{Quantum Machine Learning}, QML, 
busca aprovechar el espacio de Hilbert para procesar información, que para las 
redes neuronales clásicas sería inalcanzable. Este reporte explora precisamente 
esa frontera: cómo las herramientas de IA geométrica y cuántica pueden aplicarse a 
problemas fundamentales como la clasificación de jets y la optimización de funciones.

\section{Descripción del CIIEC y el Programa}
El Centro Interdisciplinario de Investigación y Enseñanza de la Ciencia, CIIEC, es 
una unidad académica de la BUAP dedicada a la investigación de frontera y la formación 
de recursos humanos especializados. El ambiente multidisciplinario del centro propició 
un espacio ideal para integrar conceptos de física teórica con ciencias de la computación.

Durante el programa, se trabajó bajo un esquema de mentoría semanal, donde el enfoque, 
más allá de la aplicación de herramientas existentes, fue la comprensión 
profunda de los fundamentos matemáticos detrás de las arquitecturas de redes 
neuronales y los circuitos cuánticos.

\section{Objetivos Generales}
El propósito principal de la estancia fue desarrollar competencias avanzadas en 
computación científica. Los objetivos específicos se desglosaron de la siguiente 
manera:

\begin{enumerate}
    \item \textbf{Investigación del estado del arte:} Realizar una revisión 
    bibliográfica exhaustiva sobre técnicas del machine Learning así como los modelos
    más usados al momento.
    \item \textbf{Implementación de modelos de IA:} Construir desde cero modelos de 
    aprendizaje profundo, incluyendo MLPs, CNNs y etc.
    \item \textbf{Implementación de Algoritmos Cuánticos:} Estudiar y replicar 
    algoritmos fundamentales, comprendiendo sus implicaciones y su implementación en 
    simuladores reales.
    \item \textbf{Desarrollo de pruebas técnicas para ML4SCI:} Resolver los desafíos 
    de código propuestos por la organización \textit{Machine Learning for Science} 
    (ML4SCI) para el programa Google Summer of Code (GSoC), abarcando desde circuitos 
    básicos hasta nuevas arquitecturas híbridas.
\end{enumerate}

\chapter{Metodología y herramientas}

El desarrollo del proyecto siguió una metodología iterativa e incremental. El flujo 
de trabajo se estructuró mediante la definición de objetivos, comenzando con la 
revisión teórica, seguida de la implementación de modelos y finalizando con la 
aplicación a datos reales o simulados.

Semanalmente se realizaron sesiones con la asesora para verificar avances, 
depurar el código y discutir las interpretaciones físicas de los resultados 
obtenidos. Este enfoque permitió comprender rápidamente entre diferentes arquitecturas, 
por ejemplo, de MLPs a GNNs. Además, se fomentó la experimentación con diferentes 
hiperparámetros y técnicas de regularización para optimizar el rendimiento de los 
modelos. 

\section{Entorno de Desarrollo y Librerías}
Para la ejecución de las tareas, se configuró un entorno de desarrollo basado en el 
ecosistema de Python, aprovechando su hegemonía tanto en la ciencia de datos como 
en la computación cuántica. A continuación, se describen las herramientas clave y 
su función específica en el proyecto:

\subsection{Análisis y Visualización de Datos}
El manejo de los datasets de física, como \textit{ParticleNet}, requirió herramientas 
óptimas para la manipulación tensorial y gráfica.
\begin{itemize}
    \item \textbf{NumPy y Pandas:} Utilizados para el preprocesamiento de datos 
    estructurados, manejo de arrays multidimensionales y carga de archivos en 
    formatos científicos, como \textit{.npz, .h5}.
    \item \textbf{Matplotlib y Seaborn:} Fundamentales para la generación de 
    gráficas de pérdida, matrices de confusión y visualización de 
    distribuciones de variables cinemáticas ($p_T, \eta, \phi$).
\end{itemize}

\subsection{Machine Learning Clásico y Geométrico}
Se emplearon los frameworks de aprendizaje profundo más extendidos para construir 
los modelos de referencia y las arquitecturas avanzadas.
\begin{itemize}
    \item \textbf{PyTorch y PyTorch Geometric:} Fueron las herramientas principales 
    para la Tarea II. PyTorch Geometric permitió la implementación eficiente de las 
    redes de paso de mensajes (MPNN) y atención (GAT) sobre grafos no euclidianos.
    \item \textbf{TensorFlow y Keras:} Utilizados para la implementación rápida de 
    prototipos y para la construcción de las Redes Kolmogorov-Arnold (KAN) clásicas 
    en la Tarea IX.
\end{itemize}

\subsection{Computación Cuántica (SDKs)}
Dado el enfoque comparativo del trabajo, se utilizaron múltiples backends para 
evaluar las diferencias en la simulación de circuitos.
\begin{itemize}
    \item \textbf{PennyLane (Xanadu) y Cirq (Google):}
    Empleados para la construcción de circuitos híbridos en la Tarea I.
    \item \textbf{Qiskit (IBM):} Empleado para la implementación canónica del 
    Algoritmo de Shor y la verificación de resultados en simuladores de ruido.
\end{itemize}

\subsection{Herramientas de Soporte}
\begin{itemize}
    \item \textbf{Jupyter Notebooks:} Entorno interactivo principal para la 
    experimentación rápida y documentación de código.
    \item \textbf{Git y GitHub:} Utilizados para el control de versiones y la 
    entrega final del portafolio de evidencias para la evaluación de GSoC.
\end{itemize}

\part{Machine Learning}
%--- ENERO-FEBRERO ---
\chapter{Aprendizaje automático (Machine learning)}
Esta primera etapa se centró en establecer una base sólida sobre los temas clave del proyecto. 
En esta sección se desarrollan el primer objetivo establecido en la introducción, 
se escriben las actividades realizadas para comprender los fundamentos del machine
learning y la computación cuántica y también se da el contenido teórico recabado
durante esta fase. 

Se realizó una investigación exhaustiva sobre las arquitecturas de redes neuronales más relevantes 
para la clasificación de imágenes. Se profundizó en los conceptos de capas convolucionales, pooling, 
y funciones de activación. Como resultado práctico, se implementó una Red Neuronal Convolucional (CNN) 
para un problema de clasificación de imágenes estándar, como MNIST. Es importante recordar las fechas
en las que se realizó la recabación de la información ya que fue a inicios de 2025, basándose
principalmente en el libro \textit{Machine Learning Crash Course for Engineers} de Eklas Hossain del 2024. 
Es importante esta aclaración porque la inteligencia artificial ha estado avanzado rápidamente y los 
modelos que las distintas empresas han estado lanzando enn los últimos meses han cambiado el panorama 
del machine learning. Además, desde la popularización de los modelos de lenguaje, Large Language Models, 
como ChatGPT, la salida y actualización de los modelos ha sido muy rápida, en cuestion de meses empezamos
a ver nuevos modelos que superan a los anteriores en distintas áreas y tareas. También se han generado
modelos generativos muy poderosos que crean imágenes, videos, audios, código etc.

\section{Definición y conceptos clave}
Primero definamos qué es machine learning o aprendizaje automático.

El \textit{Machine Learning} es una rama de la inteligencia artificial (IA) 
que permite a las computadores aprender de información existente y aplicar
dicho aprendizaje a tareas similares sin necesidad de programarlo 
explícitamente \cite{hossain}. 

Esta herramienta fue creada para automatizar procesos monótonos y 
suceptibles a errores humanos, como la clasificación de imágenes. Con el propósito de 
dotar a las computadores de la habilidad de hacer que las tareas sena más
accesibles y eficientes se logró automatizar, personalizar y descubir conocimiento nuevos
permitiendo la innovación en diversos campos como la medicina, finanzas, transporte,
entre otros.

El flujo de trabajo tipico del machine learning lo podemos estructurar en las sigueintes 
4 etapas \cite{hossain}:
\begin{enumerate}
    \item \textbf{Recolección de datos:} Se recopilan datos relevantes para el problema que se desea resolver.
    \item \textbf{Preprocesamiento de datos:} Los datos se limpian y transforman para que sean adecuados para el entrenamiento del modelo.
    \item \textbf{Entrenamiento del modelo:} Se selecciona un algoritmo de machine learning y se entrena utilizando los datos preprocesados.
    \item \textbf{Evaluación y ajuste del modelo:} El modelo se evalúa utilizando datos de prueba y se ajusta para mejorar su rendimiento.
\end{enumerate}

\begin{figure}
    \centering
    \includegraphics[width=0.7\textwidth]{imagenes/Flujo_de_trabajo_ML.png}
    \caption{Flujo de trabajo típico del machine learning. \cite{hossain}}
    \label{fig:ml_workflow}
\end{figure}

Las aplicaciones del machine learning son diversas y abarcan múltiples campos
como la visión por computadora, el procesamiento de lenguaje natural, la medicina,
filtros de spam, entre otros. En la visión por computadora, por ejemplo, se utilizan
redes neuronales convolucionales (CNN) para tareas como el reconocimiento de imágenes.

\section{Algoritmos}
Existen varias formas de clasificar los algoritmos del machine learning, una de ellas
es según cómo aprenden, es decir, qué tipo de datos utiliza y cómo recibe retroalimentación.
Con base en esto, los algoritmos se pueden clasificar en tres categorías principales:
\begin{itemize}
    \item \textbf{Aprendizaje supervisado:} El modelo se entrena con datos etiquetados, es decir,
    cada entrada tiene una salida deseada asociada. El objetivo es que el modelo aprenda a mapear
    entradas a salidas correctas. Ejemplos de algoritmos supervisados incluyen regresión lineal,
    máquinas de vectores de soporte (SVM) y redes neuronales.
    
    \item \textbf{Aprendizaje no supervisado:} El modelo trabaja con datos no etiquetados y debe
    encontrar patrones o estructuras subyacentes en los datos. Ejemplos incluyen clustering
    (agrupamiento) y reducción de dimensionalidad.
    
    \item \textbf{Aprendizaje por refuerzo:} El modelo aprende a tomar decisiones mediante la
    interacción con un entorno. Recibe recompensas o castigos según las acciones que realiza,
    y su objetivo es maximizar la recompensa acumulada a lo largo del tiempo.
\end{itemize}

\section{Estado del arte}
\subsection{Redes neuronales de grafos (GNN)}
Las redes neuronales de grafos, o Graph Neural Networks (GNN),  son una clase de modelos de machine
learning diseñados para trabajar con datos estructurados en forma de grafos o estructuras amorfas. 
Estas redes son especialmente útiles para tareas donde las relaciones entre los datos son tan 
importantes como los propios datos. Las GNN han demostrado ser efectivas en una gran variedad de 
aplicaciones, por mencionar algunas:
\begin{itemize}
    \item Clasificación de nodos: Asignar etiquetas a los nodos en un grafo basado en sus caracteríticas
    y relaciones.
    \item PLN (Procesamiento de lenguaje natural): Modelar relaciones semánticas entre palabras o frases.
    \item Química molecular: predecir propiedades de moléculas basándose en sus propiedades.
    \item Redes sociales: Analizar interacciones entre usuarios y comunidades.
    \item Tráfico y transporte: Modelar redes de transporte y optimizar rutas.
\end{itemize}

\begin{figure}[h!]
    \centering
    \includegraphics[width=0.7\textwidth]{imagenes/Fig_7.1-Flujo de una GNN.png}
    \caption{Arquitectura de una red neuronal de grafos. \cite{hossain}}
    \label{fig:GNN_arquitectura}
\end{figure}

\subsection{EfficientNet}
EfficientNet es una familia de arquitecturas de red neuronal convolucional (CNN) que utiliza el 
método de escalado compuesto para aumentar la eficiencia y precisión del modelo, ajustando 
todas las dimensiones (ancho, profundidad y resolución de entrada) de manera equilibrada mediante
coeficientes redefinidos. Esto permite obtener mejores resultados con menos parámetros y menor
costo computacional en comparación con arquitecturas tradicionales como ResNet o Inception.
Podemos llamar a EfficientNet usando librerías populares como TensorFlow y PyTorch, facilitando
su integración en proyectos de machine learning.

\begin{lstlisting}[language=Python, caption={Ejemplo de código para cargar EfficientNet en TensorFlow}]
from tensorflow.keras.applications import EfficientNetB0
\end{lstlisting}

\subsection{InceptionV3}
Inception v3 es una red neuronal convolucional (CNN) desarrollada 
para GoogLeNet, se utiliza para la detección de objetos y análisis 
de imágenes. Algunos aspectos clave son:
\begin{itemize}
    \item contiene 42 capas,
    \item 25 millones de parámetros,
    \item Permite redes más profundas sin aumentar excesivamente los parámetros,
    \item Para evitar sobreajuste, utiliza arquitecturas conectadas de manera dispersa.
\end{itemize}

\begin{figure}[h!]
        \centering
        \includegraphics[width=1\linewidth]{imagenes/Fig_7.6_Arquitectura-Inception-v3.png}
        \caption{Arquitectura de inception v3. Extraída de Eklas Hossain (2024)}
        \label{fig: arquitectura_inception}
\end{figure}

\subsection{Yolo}
YOLO (You Only Look Once) es un algoritmo avanzado de detección de objetos que 
ofrece resultados en tiempo real. YOLO trata la detección como un problema de 
regresión, analizando toda la imagen a la vez.

\begin{itemize}
    \item Tiempo de procesamiento: 45 FPS a 155 FPS.
    \item Detector de objetos más rápido: Poseé el mayor promedio de precisión 
    (mAP) con menor error usando la base de datos COCO
    \item Generalización
    \item Open-source
\end{itemize}


\subsection{Facebook Prophet}
Algoritmo para pronosticar datos de series temporales. Su objetivo es proporcionar 
una herramienta de predicción potente y fácil de interpretar sin necesidad de 
conocimientos profundos.
    \begin{itemize}
        \item Trabaja con tendencias lineales y no lineales.
        \item Maneja múltiples estacionalidades e intervalos irregulares.
        \item Captura cambios en datos históricos debido a eventos importantes.
        \item No requiere mucho preprocesamiento de datos, maneja datos faltantes y no es afectado por valores atípicos.
    \end{itemize}


\subsection{ChatGPT}
ChatGPT es un modelo de lenguaje desarrollado por OpenAI, lanzado en noviembre de 
2022. LAgunas caracteríticas clave de ese modelo son:
    \begin{itemize}
        \item domina 80 idiomas
        \item fue entrenado con 300 mil millones de palabras
        \item contenía 175 millones de parámetros
        \item se usó aprendizaje supervisado y por refuerzo
        \item fue escrito en Python usando TensorFlow y Pytorch
    \end{itemize}

Algunas limitaciones
    \begin{itemize}
        \item puede generar información incorrecta, comunmente se le llama «alucinaciones».
        \item Formulación de preguntas. Aunque el modelo ha avanzado demasido en los
        años posteriores para que la forma de la pregunta influye menos, sigue siento
        importante y presenta ventajas significativas un correcto promt\footnote{
            Este término fue apodado más adelante y se refiere a cómo generar el texto 
            inicial o la pregunta. Ya que el modelo solo contruye respuestas usando una
            secuancia de palabras más problables en continuar la oración es mejor darle 
            bastante contexto para que sea más fácil obtener una buena respuesta.
        }
        \item Filtro en preguntas «inapropiadas» o «dañinas». Este filtro a evolucionado
        \item con el tiempo, al igual que muchas características pero sigue siendo tema
        de búsqueda, el cómo es que el modelo genera contenido sencible.
        \item Redundancia.
        \item Contexto. La cantidad de información que pueden recordar los modelos es variada
        e influye mucho en los resultados.
        \item Ventana de contexto y/o tokens. A la par con el contexto o memoria que puede
        recordar un modelo también está cuánta información puede recibir el modelo en un solo 
        prompt.
\end{itemize}

\subsection{Otro modelos actuales del primer semestre del 2026}

En la industria e investigación lso modelos continuan evolucionando\footnote{Esto se agrega a la fecha 
de redactado este reporte, enero del 2026.}, algunos que han 
ganado presencia en el entorno mundial son:

\begin{itemize}
    \item \textbf{Modelos generativos}. En esencia son aquellos modelos que generan contenido
    a partir de la información proporcionada. Por ejemplo, para generar imágenes tenemos a 
    Nano Banana, para código tenemos a Claude, video tenemos a Sora, entre muchos otros.
    \item \textbf{Modelos agentivos} Un agente de Ia es un sistema autónomo sofisticado que
    percibe su entorno, razona sobra la información recibida, aprende de la experiencia y 
    ejecuta acciones deliberadas para cumplir objetivos específicos \cite{ai_curso}. 
\end{itemize}


\section{Desafíos y optimización de modelos}

\subsection{Desafíos en seguridad}

Los desafíos de seguridad en el uso de la IA se puede ver reflejados en las aplicaciones 
prácticas en tres ejes:
    \begin{itemize}
        \item Ataques de entrada adversarial. Estos ataques consisten en agregar 
        ruido a los datos de entrenamiento, imperceptible para el humano pero 
        dañino para el entrenamiento y en consecuencia, los resultados.
        \item Ataques de peso adversarial. Estos problemas radican en el cambiar
        o modificar los parámetros almacenados en la memoria de la computadora, 
        ya se a de forma física o digital. Por ejemplo, si estamos en un entorno con 
        alta radiación los bits pueden cambiar.
        \item Alteriación estadística de los datos. Cuando los datos no son fiables 
        podemos estar entrenando un modelo que no funcionará para el propósito que se 
        quería. Es necesario analizar la fuentes y distribuciones de los datos.
    \end{itemize}

\begin{figure}[h!]
        \centering
        \includegraphics[width=0.75\linewidth]{imagenes/Fig_7.10_Attacks_in_AI-ML.png}
        \caption{Problemas de seguridad. Extraída de Eklas Hossain (2024)}
        \label{Flujo de GNN}
    \end{figure}

\subsection{Desafíos en hardware}

El despliegue de modelos de Deep Learning en entornos de física de altas energías 
(como FPGAs en detectores o sistemas embebidos) presenta restricciones severas de 
latencia y consumo energético. Para abordar estos desafíos, es necesario reducir 
la complejidad computacional y el tamaño de los modelos sin sacrificar 
significativamente su precisión (\textit{accuracy}). Dos de las técnicas más 
predominantes son la cuantización y la poda de pesos.

\subsection{Cuantización}
La cuantización es el proceso de mapear valores de entrada de un conjunto grande 
(a menudo continuo, como números de punto flotante de 32 bits, FP32) a un conjunto 
más pequeño y discreto (como enteros de 8 bits, INT8). Este proceso reduce la 
huella de memoria y acelera la inferencia al permitir operaciones aritméticas con 
enteros.

Matemáticamente, la cuantización de un número real $r$ a un valor cuantizado $q$ 
se define mediante un factor de escala $S$ y un punto cero $Z$:

\begin{equation}
    q = \text{round}\left( \frac{r}{S} + Z \right)
\end{equation}

Donde la recuperación del valor real (descuantización) se obtiene mediante:
\begin{equation}
    r \approx S(q - Z)
\end{equation}

Existen dos esquemas principales para determinar $S$ y $Z$:

\subsubsection{Cuantización Afín (Affine Quantization)}
También conocida como \textbf{cuantización asimétrica}. Se utiliza cuando la 
distribución de los datos no es simétrica respecto al cero (por ejemplo, las 
salidas de una función de activación ReLU, que son siempre no negativas).

En este esquema, el rango de valores reales $[r_{min}, r_{max}]$ se mapea al rango 
de enteros $[q_{min}, q_{max}]$ (por ejemplo, $[0, 255]$ para 8 bits). El factor 
de escala $S$ y el punto cero $Z$ se calculan como:

\begin{equation}
    S = \frac{r_{max} - r_{min}}{q_{max} - q_{min}}
\end{equation}

\begin{equation}
    Z = \text{round}\left( q_{min} - \frac{r_{min}}{S} \right)
\end{equation}

Este método aprovecha todo el rango dinámico de los enteros disponibles, pero 
introduce un costo computacional adicional durante la inferencia debido a que 
$Z \neq 0$, lo que requiere sumar este término en cada operación matricial.

\subsubsection{Cuantización por Escala (Scale Quantization)}
Conocida también como \textbf{cuantización simétrica}. Se utiliza comúnmente para 
cuantizar los pesos del modelo, que tienden a distribuirse normalmente alrededor 
del cero. 

Aquí, forzamos a que el punto cero sea $Z = 0$. El rango de valores reales se 
define simétricamente como 
$[-r_{max}, r_{max}]$, donde $r_{max} = \max(|r_{min}|, |r_{max}|)$. 
El factor de escala se simplifica a:

\begin{equation}
    S = \frac{r_{max}}{2^{n-1} - 1}
\end{equation}

Donde $n$ es el número de bits. Al ser $Z=0$, la operación de descuantización se 
reduce a una simple multiplicación ($r \approx S \cdot q$), lo que hace que la 
inferencia sea computacionalmente más eficiente que en el esquema afín.

\subsection{Poda de Pesos (Weight Pruning)}
La poda consiste en eliminar sistemáticamente las conexiones (pesos) menos 
importantes de una red neuronal, convirtiendo las matrices densas en matrices 
dispersas (\textit{sparse matrices}). La hipótesis subyacente es que muchos 
parámetros son redundantes y su eliminación tiene un impacto mínimo en la 
salida del modelo.

El proceso se puede formalizar definiendo una máscara binaria $M$ del mismo 
tamaño que la matriz de pesos $W$:

\begin{equation}
    W' = W \odot M
\end{equation}

Donde $\odot$ denota el producto de Hadamard (elemento a elemento). La máscara 
se determina según un criterio de importancia, típicamente la magnitud del peso:

\begin{equation}
    M_{ij} = 
    \begin{cases} 
    1 & \text{si } |W_{ij}| \geq \lambda \\
    0 & \text{si } |W_{ij}| < \lambda 
    \end{cases}
\end{equation}

Donde $\lambda$ es un umbral predefinido. Existen dos estrategias principales:
\begin{itemize}
    \item \textbf{Poda no estructurada:} Elimina pesos individuales arbitrarios. 
    Logra altas tasas de compresión pero requiere hardware específico para 
    acelerar las operaciones con matrices dispersas irregulares.
    \item \textbf{Poda estructurada:} Elimina estructuras completas (neuronas 
    enteras, canales o filtros de convolución). Aunque la tasa de compresión 
    es menor, el modelo resultante mantiene su estructura densa y puede 
    acelerarse directamente en GPUs y TPUs estándar.
\end{itemize}

\section{Conclusiones}
La IA y el aprendizaje automático ML han avanzado significativamente, pero aún 
dependen de los humanos para evitar errores triviales.
Estas tecnologías están destinadas a complementar la fuerza laboral humana, 
volviéndose cada vez más esenciales en las tecnologías modernas.
\chapter{Algoritmo desarrollado}
Para ejemplificar los conceptos aprendidos se implemento una red neuronal convolucional
(CNN) utilizando keras. Esta CNN fue entrenada para clasificar número escritos a mano.
Se utilizó el conjunto de datos MNIST, que contiene 60 000 imágenes de entrenamiento y 
10 000 imágenes de prueba de dígitos del 0 al 9. Los objetivos de esta sección son:
\begin{itemize}
    \item Construir y entrenar una red neuronal para clasificar imágenes de números 
    escritos con su respectivo nombre.
    \item Analizar y comprender los componentes necesarios para hacer un 
    entrenamiento efectivo.
\end{itemize}

\section{Introducción}
El cerebro humano es quien dicta la forma en la que percibimos sabores, olores, 
colores, formas, sonidos, etc. Nos permite guardar memorias, emociones y sueños. 
Por ello se ha buscado de usar herramientas que traten de asemejarse a este tipo 
de procesamiento.

A través de los años, el ser humano se desarrolla y mucho de su aprendizaje se basa
en prueba y error hasta que se aprende de forma intuitiva a diferenciar un perro de un 
gato, el rojo del azul, etc.

\begin{figure}[h!]
    \centering
    \includegraphics[width=0.35\linewidth]{imagenes/01_Datos_MNIST.png}
    \caption{Números dibujados. Extraído del dataset MNIST y graficado con matplotlib.}
    \label{fig:datos_MNIST}
\end{figure}

Un problema complejo que podemos estudiar es el diferenciar los número escritos. Cada
persona tiende a tener una forma particular de escribi los número, aún cuando les 
enseñaron el mismo símbolo, es posible que dos personas no escriben de la misma forma 
dicho número.

Uno podría pensar en enunciar la lista de caracteríticas que diferencia un perro de un 
gato, un 1 con un 7, un 4 con un 9, pero como existen una gran divesidad de perros y gatos
así como una gran diversidad de «versiones» del mismo número, entonces lo mejor sería
dejar que la red neuronal aprenda a diferenciarlos.

Para lograr esto necesitamos cambiar el paradigma de cómo «instruir» que un programa 
nos realice una tarea. Las computadores son buenas haciendo cálculos matemáticos rápidos
y también son buenas en seguir instrucciones. Ahora dejaremos que la red neuronal explore
los datos a los cuales nosotros ya les pusimos una nota con el número que representa 
dicha imagen. 

\subsubsection{Definición de modelo}
Ante todo esto, necesitamos definir qué es un modelo. Un modelo lo definiremos como una
función 

\begin{equation}
    h(\vec{x},\theta)
\end{equation}

donde $\vec{x}$ representará el conjunto de datos de entrada mientras que $\theta$ son
los parámetros del modelo. Por ejemplo, en la función

\begin{equation*}
    f(x,y) = 5x^3 -3 x^2 + 12 \cos{\pi y}
\end{equation*}

$5$, $-3$, $12$, $\pi$ son los parámetros del modelo, al igual que el valor de los 
exponentes de las variables $x$ y $y$, mientras que $x$ y $y$ son nuestros datos de
entrada. Una red neuronal será un modelo, es decir, estos procesos que queremos que 
aprenda nuestra red neuroanl artificial los podremos manejar como una serie de funciones
con sus respectivos parámetros e información de entrada. 

\subsubsection{Estructura básica de una red neuronal artificial}
Las redes neuronales artificiales buscan imitar el funcionamiento de las redes neuronales
dentro del cerebro. La neurona es la unidad fundamental del cerebro, procesa información
mediante dentritas y conexiones dinámicas que dependen de procesos eléctricos y químicos
\cite{khan_curso_neuronas}. Esta información es tranmitida por medio de las células de
Schawamm que en conjunto forman el axón, es decir, el puente de conexión entre distintas 
neuronas. En cuanto mayor es el uso de esta neurona, mayor es la plasticidad de este, 
esto quiere decir, es más fácil que pase información. Para visualizar mejor esto podemos
ver la figura \ref{fig:Sinapsis} 

\begin{figure}[h!]
    \centering
    \includegraphics[width=0.75\linewidth]{imagenes/01_Sinápsis.png}
    \caption{Estructura de una neurona y sinapsis.}
    \label{fig:Sinapsis}
\end{figure}

Con esto en mente podemos construir nuestro modelo. Necesitamos una unidad básica 
de procesamiento de información que, en el contexto de las redes neuronales 
artificiales, toma el nombre de \textit{perceptrón}. Para imitar las señales 
electroquímicas de las neuronas biológicas usaremos \textit{funciones de activación}, 
y para modelar la facilidad con la que la información pasa de una neurona a otra 
usaremos los \textit{pesos} del modelo (sus parámetros que serán ajustables a pueba 
y error).

Esta estructura define a una neurona individual. Sin embargo, para que el modelo 
logre aprender características complejas y no lineales, necesitaremos anidar varios 
perceptrones en capas secuenciales, formando así una Red Neuronal Profunda 
(Deep Neural Network).

\begin{figure}[h!]
    \centering
    \includegraphics[width=0.65\linewidth]{imagenes/01_Red_Neuronal.jpg}
    \caption{Estructura de una red neuronal artificial. Extraída de 
    https://es.linkedin.com/pulse/7-consejos-para-trabajar-con-redes-neuronales-en-rodríguez-mgs}
    \label{fig:Red_neuronal}
\end{figure}

La primera capa, por donde ingresan los datos, recibe el nombre de \textit{capa 
de entrada}. La última capa, que entrega el resultado, se llama \textit{capa de 
salida}. Las capas intermedias se denominan \textit{capas ocultas}. 

Para formalizar matemáticamente la transmisión de información, adoptaremos la 
notación estándar: sea $w_{jk}^l$ el peso que conecta la neurona $k$ de la 
capa anterior $(l-1)$ con la neurona $j$ de la capa actual $l$. De esta forma, 
la activación $a_j^l$ de la $j$-ésima neurona en la capa $l$ se define como:

\begin{equation}
    a_j^l = f \left( \sum_k w_{jk}^l a_k^{l-1} + b_j^l \right)
    \label{eq:expresion_perceptron}
\end{equation}

Donde:
\begin{itemize}
    \item $f$: es la función de activación (no lineal).
    \item $b_j^l$: es el sesgo (\textit{bias}) de la neurona $j$ en la capa $l$.
    \item $a_k^{l-1}$: es la salida (activación) de la neurona $k$ en la capa anterior.
    \item La sumatoria recorre todas las $k$ neuronas de la capa anterior.
\end{itemize}


\section{Entrenamiento}
\subsection{¿Cómo aprende una red neuronal?}
Para entender cómo podemos hacer que una red neuronal aprenda podemos estructurarlo
en los siguientes pasos:
\begin{enumerate}
    \item \textbf{Inicialización de parámetros:} 
    Antes de comenzar el entrenamiento, los pesos $\omega$ y sesgos $b$ del modelo 
    deben definirse. No se pueden iniciar en cero (para evitar problemas de simetría 
    que impedirían el aprendizaje), por lo que se inician usualmente de forma 
    aleatoria siguiendo distribuciones estadísticas específicas (como la 
    inicialización de \textit{Xavier} o \textit{He}), lo que asegura que la 
    varianza de las activaciones se mantenga estable a través de las capas.

    \item \textbf{Propagación hacia adelante (Forward Propagation):} 
    Se ingresan los datos de entrenamiento a la red. La información fluye desde la 
    capa de entrada, atravesando las capas ocultas mediante operaciones matriciales 
    y funciones de activación no lineales, hasta generar una predicción final 
    $\hat{y}$.
    
    \item \textbf{Cálculo de la función de costo:} 
    Se evalúa qué tan lejos está la predicción $\hat{y}$ del valor real $y$. Para 
    esto utilizamos una \textit{función de costo} (o Loss Function), denotada como 
    $J(\theta)$. Dependiendo del problema, se utilizan métricas como el Error 
    Cuadrático Medio (MSE) para regresión o la Entropía Cruzada para clasificación.

    \item \textbf{Retropropagación (Backpropagation):} 
    Este es el paso crucial. Utilizando el algoritmo de retropropagación, 
    calculamos el gradiente de la función de costo con respecto a cada parámetro 
    de la red ($\nabla_\omega J$).Esto lo podemos ver de forma intuitiva como calcular
    cuanto contribuye cada parámetro en el costo final para así saber qué influye más 
    y qué influye menos. Matemáticamente, esto implica aplicar la 
    \textbf{regla de la cadena} desde la última capa hacia atrás, para determinar 
    la sensibilidad del error ante cambios infinitesimales en cada peso.

    \item \textbf{Optimización y actualización de pesos:} 
    Con los gradientes calculados, utilizamos un algoritmo optimizador (como 
    \textit{Stochastic Gradient Descent} o \textit{Adam}) para actualizar los 
    pesos. El objetivo es mover los parámetros en la dirección opuesta al gradiente 
    para encontrar el mínimo global (o un buen mínimo local) de la superficie de 
    error. La actualización sigue la regla general:
    \begin{equation}
        \omega_{nuevo} = \omega_{viejo} - \eta \cdot \nabla J
    \end{equation}
    donde $\eta$ es la \textit{tasa de aprendizaje} (learning rate).

    \item \textbf{Iteración (Épocas):} 
    El proceso anterior (pasos 2 al 5) constituye una iteración. Se repite este 
    ciclo múltiples veces sobre todo el conjunto de datos. Cada pase completo por 
    el dataset se conoce como una \textit{época}. El ciclo continúa hasta que la 
    función de costo converge a un valor mínimo aceptable o se cumple un criterio 
    de parada temprana.
\end{enumerate}

\subsection{Optimización y Algoritmo de Retropropagación}

En esta sección se expandirá los temas de optimización y el algoritmo de retropropagación
o backpropagation. Estos temas son muy interesantes, profundos y puntos críticos en el 
entrenamiento de una red neuronal.

Una vez que la red ha realizado una predicción (Forward Propagation), el siguiente 
paso es cuantificar el error y ajustar los parámetros del modelo para minimizarlo. 
Este proceso es análogo a encontrar el estado de mínima energía en un sistema físico, 
donde la superficie de energía está definida por la función de costo.

\subsubsection{El Descenso del Gradiente}
El objetivo central del entrenamiento es encontrar el conjunto de parámetros 
$\theta = \{W, b\}$ que minimice la función de costo $J(\theta)$. Visualmente, 
podemos imaginar $J(\theta)$ como una hipersuperficie en un espacio de dimensión 
$D$ (donde $D$ es el número total de parámetros, que puede ser de millones).

Para encontrar el mínimo global (o un mínimo local aceptable), utilizamos el método 
del \textit{Descenso del Gradiente}. La intuición matemática se basa en el cálculo 
vectorial: el gradiente de una función escalar, denotado como $\nabla J(\theta)$, 
es un vector que apunta en la dirección de mayor crecimiento de la función. Por 
consiguiente, el negativo del gradiente $-\nabla J(\theta)$ apunta en la dirección 
de mayor descenso.



La regla de actualización iterativa para un parámetro $\omega$ se define como:

\begin{equation}
    \omega_{t+1} = \omega_t - \eta \frac{\partial J}{\partial \omega_t}
    \label{eq:gradient_descent}
\end{equation}

Donde $\eta$ (tasa de aprendizaje) es un escalar positivo que controla el tamaño 
del paso. Si $\eta$ es muy pequeño, la convergencia es lenta; si es muy grande, el 
sistema puede oscilar o divergir. En la práctica, no calculamos el gradiente usando 
todo el dataset simultáneamente (Batch Gradient Descent) debido a limitaciones de 
memoria, sino que utilizamos aproximaciones estocásticas con subconjuntos de datos 
(\textit{Mini-batch Stochastic Gradient Descent}), lo que introduce cierto "ruido" 
beneficioso que ayuda a escapar de mínimos locales poco profundos.

\subsubsection{Retropropagación (Backpropagation)}
Para aplicar la ecuación (\ref{eq:gradient_descent}), necesitamos calcular 
eficientemente las derivadas parciales de la función de costo con respecto a cada 
peso y sesgo de la red. En una red profunda, esto no es trivial debido a la 
composición de funciones anidadas. La solución es el algoritmo de 
\textit{Backpropagation}, que no es más que una aplicación recursiva y eficiente 
de la \textbf{Regla de la Cadena}.

Definamos el error en la capa $l$ no como la diferencia directa con la salida, 
sino como la sensibilidad del costo a los cambios en la entrada ponderada $z^l$. 
Definimos este error $\delta^l$ como:

\begin{equation}
    \delta^l \equiv \frac{\partial J}{\partial z^l}
\end{equation}

El algoritmo se desarrolla en cuatro ecuaciones fundamentales que permiten propagar 
este error desde la capa de salida hacia atrás:

\subsubsection{1. Error en la capa de salida ($L$)}
Primero, calculamos cuánto contribuyó la última capa al error total. Usando la 
regla de la cadena:
\begin{equation}
    \delta^L = \nabla_a J \odot \sigma'(z^L)
\end{equation}
Donde $\nabla_a J$ es la derivada del costo respecto a la activación de salida, 
$\sigma'$ es la derivada de la función de activación y $\odot$ denota el producto 
de Hadamard (elemento a elemento).

\subsubsection{2. Propagación del error a capas ocultas}
Una vez conocemos el error en la capa $L$, podemos calcular el error en la capa 
anterior $L-1$, y así sucesivamente hasta la primera capa. El error fluye hacia 
atrás a través de los pesos transpuestos:
\begin{equation}
    \delta^l = ((W^{l+1})^T \delta^{l+1}) \odot \sigma'(z^l)
\end{equation}
Físicamente, esto se puede interpretar como «distribuir» el error de la 
capa posterior a las neuronas de la capa anterior, ponderado por el peso de la 
conexión ($W$).


\subsubsection{3. Cálculo de los gradientes}
Finalmente, una vez que tenemos los valores de $\delta^l$ para todas las capas, 
las derivadas parciales necesarias para el descenso del gradiente son inmediatas:

Para el sesgo ($bias$):
\begin{equation}
    \frac{\partial J}{\partial b_j^l} = \delta_j^l
\end{equation}

Para los pesos ($weights$):
\begin{equation}
    \frac{\partial J}{\partial w_{jk}^l} = a_k^{l-1} \delta_j^l
\end{equation}

De esta última ecuación podemos intuir que el gradiente de un peso que 
conecta la neurona $k$ con la $j$ es proporcional a la activación de la neurona de 
entrada ($a_{in}$) multiplicada por el error de la neurona de salida ($\delta_{out}$). 
Esto significa que los pesos que conectan neuronas muy activas con neuronas que 
contribuyen mucho al error serán los que más se ajusten durante el entrenamiento.

\subsection{Conjunto de datos}

Para entrenar una red neuronal necesitamos información para que aprenda de ella y 
se realicen los 6 pasos mencionados anteriormente (Incialización de parámetros, forward propagation,
canlculo del costo, backpropagation, optmización y actualización de pesos). Usualmente
una base de datos la podemos dividir de 2 formas. La primera forma es dividirla en 3,
1) el primer conjunto es información de entrada, el conjunto de \textit{entrenamiento}; 
2) el segundo conjunto es de \textit{validación} de información, es decir, donde le mostramos
las respuestas de lo que queremos y con el que se mide el costo; y 3) el conjunto de
\textit{evaluación}, este conjunto no se usa nunca en los datos y es para ver cómo
se comporta el modelo cuando tiene información desconocida. Los porcentajes usuales 
aproximados suelen ser de 70 \% entrenamiento, 15 \% validación, y 15 \% evaluación
aunque uno puede cambiar los porcentajes con forme sea mejor siempre se recomienda 
dejar la mayor cantidad de datos posibles para entrenamiento sin ignorar por completo
los otros dos para evitar que la red «memorice» en lugar de «aprender».

La segunda forma es usar toda la base de datos y dividirla en entrenamiento y validación.
Para evaluar el modelo se usa otra base de datos distinta.

Resumiendo un poco

\textbf{1. Conjunto de Entrenamiento (Train Set)}
    \begin{itemize}
        \item Suele usarse entre el \textbf{70-80\%} de los datos.
        \item Se usa para \textbf{ajustar los pesos} de la red.
    \end{itemize}
    
\textbf{2. Conjunto de Validación (Validation Set)}
    \begin{itemize}
        \item \textbf{10-15\%} de los datos.
        \item Se usa para \textbf{ajustar hiperparámetros} (capas, tasa de aprendizaje, regularización, etc.).
        \item Ayuda a detectar \textbf{sobreajuste (overfitting)}.
    \end{itemize}

\textbf{3. Conjunto de Prueba (Test Set)}
    \begin{itemize}
        \item \textbf{10-15\%} de los datos.
        \item Evalúa el \textbf{rendimiento final} del modelo en datos nunca vistos.
    \end{itemize}
\textbf{¿Por qué son importantes?}
    \begin{itemize}
        \item Evitan que el modelo \textbf{memorice} los datos de entrenamiento.
        \item Permiten evaluar la \textbf{generalización} del modelo.
        \item Aseguran un buen desempeño en datos reales.
    \end{itemize}



\section{Componentes clave}
Esta sección es una lista de herramientas, es decir, se definirán funciones de activación
usuales, así como las funciones de optmizadores y de regularización.

\subsection{Funciones de activación}

Como se ha descrito anteriormente, la información fluye a través de la red neuronal 
mediante operaciones matriciales. Sin embargo, si se usara únicamente 
combinaciones lineales de los inputs (del tipo $y = Wx + b$), la red neuronal 
colapsaría matemáticamente a un simple modelo de regresión lineal, 
independientemente de cuántas capas ocultas tenga.
Esto se debe a que la composición de funciones lineales es, a su vez, una función 
lineal. Si tenemos dos capas lineales consecutivas, $f(x) = W_1 x$ y $g(h) = W_2 h$, 
la salida final sería 
\begin{equation*}
    g(f(x)) = W_2(W_1 x) = (W_2 W_1)x = W_{new}x
\end{equation*} 

Es decir, 
toda la profundidad de la red podría resumirse en una sola matriz de pesos.

Para romper esta linealidad y permitir que el modelo aprenda patrones complejos y 
fronteras de decisión no convexas (basándose en el \textit{Teorema de Aproximación 
Universal}), es estrictamente necesario introducir no-linealidades mediante las 
\textbf{funciones de activación} $\sigma(\cdot)$ al final de cada neurona.

\subsubsection{Teorema de Aproximación Universal}
Desde la perspectiva del álgebra lineal, una red neuronal sin activaciones no lineales 
sería equivalente a una composición de transformaciones lineales. Dado que la 
composición de transformaciones lineales es, a su vez, una transformación lineal 
\cite{friedberg_linear}, una red profunda sin no-linealidades colapsaría 
matemáticamente a una sola capa (un modelo de regresión lineal), perdiendo toda 
capacidad de modelar estructuras complejas.

Para resolver esto, se introducen no-linealidades. El \textbf{Teorema de 
Aproximación Universal}, demostrado formalmente por Cybenko \cite{cybenko1989approximation} 
y generalizado posteriormente por Hornik \cite{hornik1991approximation}, 
establece que una red neuronal \textit{feedforward}, con al menos una capa oculta
y funciones de activación no lineales acotadas y monotona creciente, puede aproximar 
cualquier función continua, $f: \mathbb{R}^n \to \mathbb{R}^m$, con un 
grado de precisión arbitrario.

Esto implica que la no linealidad es la característica que otorga a las redes 
neuronales su capacidad de actuar como 
\textit{aproximadores universales} de funciones complejas.

A continuación, se describen las funciones más utilizadas:

% --- SIGMOID ---
\subsubsection{Función Sigmoide (Logística)}
\noindent
\begin{minipage}{0.55\textwidth}
    Históricamente utilizada por su similitud con la tasa de disparo de una neurona 
    biológica\cite{haykin2009neural}. Comprime cualquier valor real a un rango entre 0 y 1, lo que la hace 
    útil para interpretar la salida como una probabilidad en clasificación binaria.
    
    \begin{equation}
        \sigma(x) = \frac{1}{1+e^{-x}}
        \label{eq:Sigmoid}
    \end{equation}
    
    \textbf{Propiedades:}
    \begin{itemize}
        \item \textbf{Dominio:} $(-\infty, \infty)$
        \item \textbf{Codominio:} $(0,1)$
    \end{itemize}
    
    \textbf{Desventajas:}
    No está centrada en cero y sufre severamente del problema de \textit{vanishing 
    gradients}: para valores muy altos o muy bajos de $x$, la derivada es casi cero, 
    lo que detiene el aprendizaje en redes profundas.
\end{minipage}%
\hfill
\begin{minipage}{0.42\textwidth}
    \centering
    \includegraphics[width=\linewidth]{imagenes/sigmoid.png}
    \captionof{figure}{Función Sigmoid.}
    \label{fig:Sigmoid}
\end{minipage}
\vspace{0.5cm}

% --- TANH ---
\subsubsection{Función Tangente Hiperbólica (Tanh)}
\noindent
\begin{minipage}{0.55\textwidth}
    Es una versión reescalada de la sigmoide. Se prefiere en capas ocultas sobre 
    la sigmoide porque sus salidas están centradas en cero, lo que facilita la 
    convergencia durante el descenso del gradiente.
    
    \begin{equation}
        \tanh(x) = \frac{e^x - e^{-x}}{e^x + e^{-x}}
        \label{eq:tanh}
    \end{equation}
    
    \textbf{Propiedades:}
    \begin{itemize}
        \item \textbf{Dominio:} $(-\infty, \infty)$
        \item \textbf{Codominio:} $(-1,1)$
    \end{itemize}
    
    \textbf{Desventajas:}
    Aunque mejora la convergencia, mantiene el problema de los gradientes que se 
    desvanecen (\textit{vanishing gradients}) en los extremos de saturación.
\end{minipage}%
\hfill
\begin{minipage}{0.42\textwidth}
    \centering
    \includegraphics[width=\linewidth]{imagenes/tanh.png}
    \captionof{figure}{Función Tanh.}
    \label{fig:Tanh}
\end{minipage}
\vspace{0.5cm}

% --- RELU ---
\subsubsection{Función ReLU (Rectified Linear Unit)}
\noindent
\begin{minipage}{0.55\textwidth}
    Es el estándar actual para capas ocultas en Deep Learning. Su simplicidad 
    computacional y su capacidad para mitigar el desvanecimiento del gradiente (su 
    derivada es 1 para $x>0$) permiten entrenar redes muy profundas.
    
    \begin{equation}
        f(x) = \max(0, x)
        \label{eq:ReLU}
    \end{equation}
    
    \textbf{Propiedades:}
    \begin{itemize}
        \item \textbf{Dominio:} $(-\infty, \infty)$
        \item \textbf{Codominio:} $[0, \infty)$
    \end{itemize}
    
    \textbf{Desventajas:}
    Sufre del problema de \textit{Dying ReLU}: si una neurona cae en la zona 
    negativa, su gradiente es cero y los pesos jamás se actualizarán de nuevo, 
    «matando» la neurona permanentemente.
\end{minipage}%
\hfill
\begin{minipage}{0.42\textwidth}
    \centering
    \includegraphics[width=\linewidth]{imagenes/relu.png}
    \captionof{figure}{Función ReLU.}
    \label{fig:ReLU}
\end{minipage}
\vspace{0.5cm}

% --- SOFTMAX ---
\subsubsection{Función Softmax}
\noindent
\begin{minipage}{0.55\textwidth}
    Fundamental para problemas de clasificación multiclase (como la clasificación 
    de jets). Transforma un vector de números reales ($logits$) en una distribución 
    de probabilidad normalizada.
    
    \begin{equation}
        \sigma(x)_i = \frac{e^{x_i}}{\sum_{j=1}^K e^{x_j}}
        \label{eq:Softmax}
    \end{equation}
    
    \textbf{Propiedades:}
    \begin{itemize}
        \item \textbf{Codominio:} $(0,1)$
        \item \textbf{Restricción:} $\sum \sigma(x)_i = 1$
    \end{itemize}
    
    Se utiliza casi exclusivamente en la \textbf{capa de salida}, permitiendo 
    seleccionar la clase con mayor probabilidad asociada.
\end{minipage}%
\hfill
\begin{minipage}{0.42\textwidth}
    \centering
    \includegraphics[width=0.9\linewidth]{imagenes/softmax.png}
    \captionof{figure}{Esquema Softmax.}
    \label{fig:Softmax}
\end{minipage}


\subsection{Funciones de Costo (Loss Functions)}

La función de costo, o función de pérdida cuantifica la discrepancia entre la 
predicción del modelo $\hat{y}$ y el valor real $y$. La elección de esta función 
dicta cómo se penalizan los errores y afecta drásticamente la velocidad de 
convergencia.

Dependiendo de la naturaleza del problema, regresión o clasificación, se usan 
diferentes distribuciones de probabilidad, con $L$ representado dicha función:

\subsubsection{Entropía Cruzada (Cross Entropy)}
Es la función estándar para problemas de clasificación. Desde la teoría de la 
información de Shannon\cite{shannon1948}, mide la diferencia entre dos 
distribuciones de probabilidad: la real y la estimada. Como señalan Goodfellow et 
al. \cite{goodfellow2016deep}, el uso de esta función en lugar del error cuadrático 
medio para funciones de activación sigmoideas/softmax evita la saturación del 
gradiente, penalizando logarítmicamente las predicciones que son confiadas pero 
erróneas.

\begin{itemize}
    \item \textbf{Para clasificación binaria:}
    \begin{equation}
        L = - y \log(\hat{y}) - (1 - y) \log(1 - \hat{y})   
        \label{eq:Binary_CE}
    \end{equation}
    
    \item \textbf{Para clasificación multiclase (Categorical Cross Entropy):}
    \begin{equation}
        L = - \sum_{i} y_i \log(\hat{y}_i)
        \label{eq:Categorical_CE}
    \end{equation}
\end{itemize}

\subsubsection{Error Cuadrático Medio (MSE)}
Utilizado principalmente en regresión. Asume implícitamente que el ruido en los 
datos sigue una distribución normal. Debido al término cuadrático, penaliza de 
forma severa los grandes errores, lo que lo hace sensible a valores atípicos 
(\textit{outliers}).

\begin{equation}
    L = \frac{1}{n} \sum_{i=1}^{n} (y_i - \hat{y}_i)^2
    \label{eq:MSE}
\end{equation}

% --- OPTIMIZADORES (Con Minipage como pediste) ---
\subsection{Algoritmos de Optimización}

\noindent
\begin{minipage}{0.55\textwidth}
    Una vez definida la superficie de la función de error, necesitamos un método 
    para movernos hacia el mínimo global. Los optimizadores son algoritmos que 
    actualizan los pesos $w$ basándose en el cálculo del gradiente.
    
    El método fundamental es el \textbf{Descenso del Gradiente Estocástico (SGD)}. 
    Matemáticamente, representa mover los parámetros en dirección opuesta a la 
    derivada parcial del error:
    
    \begin{equation}
        w_{t+1} = w_t - \eta \nabla L(w_t)
        \label{eq:Descenso_grad}
    \end{equation}
    
    Sin embargo, el SGD puro puede ser lento y propenso a quedarse atrapado en 
    mínimos locales o puntos de silla, por lo que se han desarrollado variantes 
    más avanzadas.
\end{minipage}%
\hfill
\begin{minipage}{0.42\textwidth}
    \centering
    \includegraphics[width=\linewidth]{imagenes/Ejemplo_error.png}
    \captionof{figure}{Representación de la superficie de error y la búsqueda de 
    mínimos. Extraída de 
    https://interactivechaos.com/es/manual/tutorial-de-deep-learning/conclusiones-partir-de-la-funcion-de-error}
    \label{fig:Superficie_Error}
\end{minipage}
\vspace{0.5cm}

\subsubsection{Método Momentum}
Inspirado en la mecánica newtoniana, el método de Momentum introduce una variable 
de velocidad física a la optimización. En lugar de actualizar los pesos basándose 
únicamente en el gradiente instantáneo, el algoritmo acumula una media móvil 
exponencial de los gradientes pasados \cite{rumelhart1986learning}.

Matemáticamente, esto tiene dos efectos descritos en \cite{goodfellow2016deep}:
\begin{enumerate}
    \item \textbf{Amortiguación de oscilaciones:} En dimensiones donde el gradiente 
    cambia de signo frecuentemente (direcciones ruidosas o paredes de un valle 
    estrecho), los términos positivos y negativos se cancelan mutuamente en la 
    media móvil.
    \item \textbf{Aceleración direccional:} En dimensiones donde el gradiente 
    mantiene el mismo signo consistentemente (a lo largo del fondo del valle), 
    la variable de velocidad se acumula, permitiendo que el optimizador avance 
    más rápido de lo que permitiría la tasa de aprendizaje estándar.
\end{enumerate}

\begin{equation}
    \begin{aligned}
    v_t &= \beta v_{t-1} + (1 - \beta) \nabla L(w_t) \\
    w_{t+1} &= w_t - \eta v_t
    \end{aligned}
    \label{eq:Opt_Momentum}
\end{equation}

Donde $\beta \in [0, 1)$ es el hiperparámetro de momento (coeficiente de fricción o decaimiento).

\textbf{Ventaja:} Permite moverse a través de mínimos locales poco profundos 
gracias a la velocidad acumulada.

\subsubsection{RMSprop (Root Mean Square Propagation)}
Diseñado para abordar el problema de tasas de aprendizaje fijas. RMSprop normaliza 
el gradiente dividiéndolo por una media móvil de su magnitud cuadrática. Esto 
significa que los pesos con gradientes grandes tendrán una tasa de aprendizaje 
efectiva menor para evitar inestabilidad, y viceversa.

\begin{equation}
    \begin{aligned}
    S_t &= \beta S_{t-1} + (1 - \beta) (\nabla L(w_t))^2 \\
    w_{t+1} &= w_t - \frac{\eta}{\sqrt{S_t} + \epsilon} \nabla L(w_t)
    \end{aligned}
    \label{eq:RMSprop}
\end{equation}

Donde:
\begin{itemize}
    \item $S_t$: Acumulador de la media móvil exponencial de los cuadrados del gradiente.
    \item $\beta$: Factor de decaimiento (hiperparámetro, típicamente 0.9).
    \item $\eta$: Tasa de aprendizaje inicial.
    \item $\epsilon$: Término de estabilidad (usualmente $\approx 10^{-8}$) para evitar la división por cero.
\end{itemize}

\subsubsection{Adam (Adaptive Moment Estimation)}
Combina lo mejor de dos mundos: la inercia del \textit{Momentum},
estimación del primer momento, es decir, la media; y la adaptación del paso de \textit{RMSprop}, 
estimación del segundo momento, es decir, varianza no centrada.

\begin{equation}
    \begin{aligned}
    m_t &= \beta_1 m_{t-1} + (1 - \beta_1) \nabla L(w_t) \quad \text{(Estimación de media)}\\
    v_t &= \beta_2 v_{t-1} + (1 - \beta_2) (\nabla L(w_t))^2 \quad \text{(Estimación de varianza)}\\
    \hat{m}_t &= \frac{m_t}{1 - \beta_1^t}, \quad \hat{v}_t = \frac{v_t}{1 - \beta_2^t} \quad \text{(Corrección de sesgo)}\\
    w_{t+1} &= w_t - \frac{\eta}{\sqrt{\hat{v}_t} + \epsilon} \hat{m}_t
    \end{aligned}
\end{equation}

Donde:
\begin{itemize}
    \item $m_t$: Estimación del primer momento (la media) de los gradientes.
    \item $v_t$: Estimación del segundo momento (la varianza no centrada) de los gradientes.
    \item $\hat{m}_t, \hat{v}_t$: Estimaciones corregidas por sesgo. Dado que $m_0$ y $v_0$ se inicializan en 0, estas fórmulas compensan el sesgo hacia cero durante los primeros pasos de tiempo $t$.
    \item $\beta_1, \beta_2$: Tasas de decaimiento exponencial para los momentos (valores estándar: $0.9$ y $0.999$ respectivamente).
    \item $\eta$: Tasa de aprendizaje (step size).
\end{itemize}

\textbf{Ventaja:} Es robusto, requiere poco ajuste de hiperparámetros y converge 
rápidamente en una amplia variedad de problemas.

% --- REGULARIZACIÓN ---
\subsection{Técnicas de Regularización}
El objetivo del aprendizaje automático no es solo minimizar el error en los 
datos de entrenamiento, sino generalizar a datos nuevos. La regularización 
introduce restricciones matemáticas para evitar el sobreajuste (\textit{overfitting}).

\begin{itemize}
    \item \textbf{Regularización L1 (Lasso):} Añade una penalización proporcional 
    al valor absoluto de los pesos ($\lambda |w|$). Esto tiende a generar vectores 
    de pesos dispersos, forzando a que muchos pesos sean exactamente cero, lo que 
    ayuda en la selección de características.
    
    \item \textbf{Regularización L2 (Ridge):} Añade una penalización proporcional 
    al cuadrado de los pesos ($\lambda w^2$). Esto fuerza a que los pesos sean 
    pequeños y difusos, evitando que una sola neurona domine la decisión.
    
    \item \textbf{Dropout:} Una técnica empírica poderosa que consiste en 
    «apagar» aleatoriamente un porcentaje $p$ de neuronas durante cada paso del 
    entrenamiento. Esto evita la coadaptación compleja de neuronas, obligando 
    a la red a aprender características robustas y redundantes. Matemáticamente, 
    equivale a entrenar un ensamble bastante grande de sub-redes diferentes.
\end{itemize}

\section{Caso práctico}

Para validar los conceptos teóricos desarrollados, se diseñó e implementó una red 
neuronal profunda (DNN) para una tarea de clasificación de imágenes. El objetivo 
fue evaluar el impacto de los diferentes hiperparámetros y optimizadores en el 
rendimiento del modelo.

\subsection{Arquitectura del Modelo}

El diseño de la red neuronal se fundamentó en una arquitectura de perceptrón 
multicapa (MLP) con una estructura de «embudo», reduciendo progresivamente la 
dimensionalidad de los datos a medida que se avanza en la profundidad de la red. 
Esto fuerza al modelo a aprender características cada vez más abstractas y compactas.

La configuración de la red se detalla a continuación:

\begin{itemize}
    \item \textbf{Capa de Entrada:} Consta de 784 neuronas. Esta dimensión deriva 
    de la naturaleza de los datos de entrada, que son imágenes de $28 \times 28$ 
    píxeles aplanadas ($28 \times 28 = 784$).
    \item \textbf{Capas Ocultas:} Se implementaron cuatro capas densas consecutivas 
    con 512, 256, 128 neuronas respectivamente.
    \item \textbf{Capa de Salida:} Compuesta por 10 neuronas, correspondientes a 
    las 10 clases posibles de clasificación del dataset.
\end{itemize}



Para el funcionamiento correcto de esta arquitectura, se seleccionaron los 
siguientes componentes clave:
\begin{enumerate}
    \item \textbf{Funciones de activación:} Se utilizó \textit{ReLU} en las capas 
    ocultas para mitigar el desvanecimiento del gradiente y acelerar la convergencia, 
    y \textit{Softmax} en la salida para obtener probabilidades.
    \item \textbf{Función de pérdida:} Se eligió la \textit{Categorical Cross 
    Entropy} dada la naturaleza multiclase del problema.
    \item \textbf{Optimizador:} Se optó por el algoritmo \textit{Adam} debido a su 
    eficiencia adaptativa.
    \item \textbf{Regularización:} Para combatir el sobreajuste se implementaron 
    técnicas de penalización L2 y capas de \textit{Dropout}.
\end{enumerate}

\begin{figure}[H]
    \centering
    \begin{minipage}{0.48\textwidth}
        \centering
        \includegraphics[width=\linewidth]{imagenes/Arquitectura.png}
        \caption{Diagrama de bloques de la estructura de la red [784, 512, 256, 
        128, 10].}
        \label{fig:Arquitectura_bloques}
    \end{minipage}\hfill
    \begin{minipage}{0.48\textwidth}
        \centering
        \includegraphics[width=\linewidth]{imagenes/Arquitectura_codigo.png}
        \caption{Implementación en código de la arquitectura del modelo.}
        \label{fig:Codigo_modelo}
    \end{minipage}
\end{figure}

\subsection{Entrenamiento y Evaluación}

El proceso de entrenamiento se configuró para ejecutarse durante un máximo de 
60 épocas. Para garantizar la reproducibilidad y modularidad del código, se 
definieron funciones específicas encargadas de iterar sobre el dataset, realizar 
el paso hacia adelante (\textit{forward pass}), calcular el error y ejecutar la 
retropropagación (\textit{backward pass}).

\begin{figure}[h!]
    \centering
    \includegraphics[width=0.6\linewidth]{imagenes/Código_3.png}
    \caption{Función de entrenamiento automatizada.}
    \label{fig:Codigo_entrenamiento}
\end{figure}

Para cuantificar el desempeño del modelo más allá de la función de pérdida, se 
implementó una métrica de exactitud (\textit{accuracy}). Esta métrica representa 
la proporción de predicciones correctas sobre el total de muestras evaluadas:

\begin{equation}
    Accuracy = \frac{\text{No. de Aciertos}}{\text{No. Total de Muestras}}
\end{equation}

Esta métrica se calcula al final de cada época utilizando una función de evaluación 
dedicada, que opera sobre un conjunto de datos de validación separado del 
entrenamiento.

\begin{figure}[h!]
    \centering
    \includegraphics[width=0.7\linewidth]{imagenes/Código_4.png}
    \caption{Función de evaluación para cálculo del accuracy.}
    \label{fig:Codigo_evaluacion}
\end{figure}

\section{Resultados}

El análisis de resultados se dividió en dos fases experimentales. En la primera 
fase, se compararon diferentes marcos de trabajo (Keras vs PyTorch) y funciones 
de activación básicas. En la segunda fase, se profundizó en el ajuste fino de 
hiperparámetros y técnicas de regularización.

\subsection{Fase 1: Comparativa de Modelos Base}
Las primeras cuatro versiones del modelo mostraron tiempos de entrenamiento 
similares, oscilando entre los 26 y 30 minutos. Los resultados obtenidos se 
resumen en la Tabla \ref{tab:Modelos_1}.

\begin{table}[h!]
    \centering
    \renewcommand{\arraystretch}{1.2} % Espaciado para que se vea mejor
    \begin{tabular}{|c|l|c|c|c|c|}
        \hline
        \textbf{Framework} & \textbf{Activadores} & \textbf{Pérdida} & 
        \textbf{Optimizador} & \textbf{Loss} & \textbf{Acc} \\ \hline
        Keras & Sigmoid, Softmax & C. Cross Ent. & Adam & 0.1202 & 0.9819 \\ \hline
        Keras & ReLU, Softmax & C. Cross Ent. & Adam & 0.0943 & 0.9722 \\ \hline
        Keras* & ReLU, Softmax & C. Cross Ent. & Adam & 0.1016 & 0.9698 \\ \hline
        PyTorch & ReLU, Softmax & C. Cross Ent. & RMSprop & 1.4876 & 0.9737 \\ \hline
    \end{tabular}
    \caption{Resultados de la primera ronda de experimentación. Se observa una 
    consistencia en el accuracy cercano al 97-98\%.}
    \label{tab:Modelos_1}
\end{table}

Se observa que, aunque el uso de la función Sigmoid obtuvo el accuracy más alto 
(98.19\%), los modelos con ReLU mostraron un comportamiento más robusto en términos 
de convergencia de la función de pérdida (0.0943). Cabe destacar la diferencia 
significativa en el valor de pérdida del modelo en PyTorch con RMSprop, lo cual 
sugiere una sensibilidad distinta en la inicialización de pesos o en la tasa de 
aprendizaje del optimizador.

\begin{figure}[h!]
    \centering
    \includegraphics[width=0.65\linewidth]{imagenes/Gráficas_juntas.png}
    \caption{Curvas de aprendizaje de los modelos de la primera fase. Se grafica 
    epocas vs pérdida. La diferencia entre la perdida entre training y validación
    significa un sobreajuste, es decir, el modelo está «memorizando» en lugar de 
    aprendiendo}
    \label{fig:Resultados_Fase1}
\end{figure}

\subsection{Fase 2: Optimización y Regularización}
En esta etapa, el tiempo de entrenamiento aumentó al rango de 45-52 minutos debido 
a la introducción de regularizadores y la extensión de las pruebas. Se mantuvo el 
uso de ReLU en capas ocultas y Softmax en la salida. 

Un hallazgo crítico fue la identificación de sobreajuste (\textit{overfitting}) 
temprano. En el experimento 4, a pesar de estar programado para 60 épocas, el 
modelo comenzó a sobreajustar (el error de validación aumentaba mientras el de 
entrenamiento bajaba) alrededor de la época 15. Se aplicó una técnica de 
\textit{Check point} manual, guardando el modelo en ese punto óptimo.



\begin{table}[H]
    \centering
    \resizebox{\textwidth}{!}{
    \begin{tabular}{|c|c|c|c|c|c|}
        \hline
        \textbf{Exp.} & \textbf{Épocas} & \textbf{Learning Rate} & 
        \textbf{Regularizador} & \textbf{Loss} & \textbf{Accuracy} \\ \hline
        1 & 60 & $5 \times 10^{-6}$ & Dropout & 0.0856 & 0.9728 \\ \hline
        2 & 60 & $5 \times 10^{-6}$ & L2 & 0.0927 & 0.9723 \\ \hline
        3 & 60 & $5 \times 10^{-6}$ & Dropout + L2 & 0.0857 & 0.9733 \\ \hline
        \textbf{4} & \textbf{60 (15)} & \textbf{$1 \times 10^{-4}$} & 
        \textbf{Dropout + L2} & \textbf{0.0621} & \textbf{0.9829} \\ \hline
    \end{tabular}
    }
    \caption{Resultados de la segunda ronda. El experimento 4 representa el modelo 
    final con un guardado temprano.}
    \label{tab:Modelos_2}
\end{table}

El modelo 4 logró el mejor desempeño global (Accuracy: 98.29\%, Loss: 0.0621) al 
combinar una tasa de aprendizaje mayor ($10^{-4}$) con una estrategia de check point, 
lo que evitó la degradación del rendimiento por sobreajuste.

\begin{figure}[h!]
    \centering
    \includegraphics[width=0.8\linewidth]{imagenes/Gráficas_juntas_2.png}
    \caption{Comparativa del desempeño del mejor modelo (Exp. 4) frente a las 
    iteraciones previas.}
    \label{fig:Mejor_Resultado}
\end{figure}

\section{Conclusiones}

El desarrollo e implementación de estas redes neuronales permite extraer las 
siguientes conclusiones fundamentales sobre el comportamiento del aprendizaje 
profundo:

\begin{itemize}
    \item \textbf{Divergencia entre Accuracy y Pérdida:} Se observó 
    experimentalmente que es posible que la función de pérdida aumente mientras 
    el accuracy se mantiene estable. Esto ocurre cuando el modelo sigue 
    clasificando correctamente las muestras, pero con una \textit{confianza} 
    (probabilidad) menor, o cuando clasifica erróneamente unas pocas muestras 
    con una confianza extremadamente alta, lo que 
    penaliza severamente la entropía cruzada.
    
    \item \textbf{Impacto de la Regularización:} La combinación de técnicas de 
    regularización (Dropout + L2) demostró ser eficaz para prevenir el sobreajuste. 
    Esto tuvo un efecto colateral positivo: permitió utilizar una tasa de 
    aprendizaje (\textit{learning rate}) más agresiva sin desestabilizar el 
    entrenamiento, reduciendo así el tiempo necesario para alcanzar la 
    convergencia.
    
    \item \textbf{Costo Computacional:} Los procesos de entrenamiento de redes 
    profundas demandan recursos computacionales significativos. El aumento de 
    complejidad en la arquitectura y el número de épocas impacta linealmente en 
    el tiempo de ejecución, lo que resalta la importancia de técnicas como el 
    \textit{Early Stopping} y/o \textit{check point} para optimizar el uso de 
    recursos.
\end{itemize}

\part{Computación Cuántica}
%--- MARZO-ABRIL ---
\chapter{Computación cuántica}
\section{Introducción}
En esta sección se desarrollan conceptos básicos de la computación cuántica para poder
desarrollar después un proyecto aplicando dichos conceptos. Particularmente en este 
capítulo se resumen los fundamentos de la computación cuántica. Donde se abordaron 
conceptos clave como:
\begin{itemize}
    \item Qubits y superposición.
    \item Entrelazamiento cuántico.
    \item Compuertas cuánticas (Hadamard, CNOT, Pauli-X, etc.).
    \item Principales algoritmos cuánticos (Grover, Shor, etc.).
\end{itemize}


El proyecto principal fue el desarrollo de una implementación funcional del 
algoritmo de Shor. La investigación incluyó un estudio profundo de:
\begin{itemize}
    \item \textbf{Criptografía y el sistema RSA:} Comprensión del mecanismo de 
    clave pública-privada y su vulnerabilidad ante la factorización de números 
    grandes.
    \item \textbf{Teoría de Números:} Conceptos como la aritmética modular y la 
    función totiente de Euler, fundamentales para el algoritmo.
    \item \textbf{Transformada Cuántica de Fourier (QFT):} El núcleo cuántico del 
    algoritmo de Shor, que permite encontrar el periodo de una función de manera 
    eficiente.
\end{itemize}

Se entregaron dos productos principales: un script de Python (.py) y un Jupyter 
Notebook (.ipynb) que demuestran el proceso completo de encriptar un mensaje con 
RSA y utilizar el algoritmo de Shor para factorizar la clave pública y desencriptar 
el mensaje.

\begin{lstlisting}[language=Python, caption={Ejemplo de código para una compuerta 
    cuántica en Python}]
import cirq

# Create a circuit
qubit = cirq.GridQubit(0, 0)
circuit = cirq.Circuit(
    cirq.H(qubit),  # Hadamard gate
    cirq.measure(qubit, key='m')  # Measurement
)
print("Circuit:")
print(circuit)
\end{lstlisting}



\section{Fundamentos de la computación cuántica}

La computación cuántica no debe entenderse solamente como una continuación 
de la ley de Moore o una mejora incremental en la velocidad de los 
procesadores actuales. Se trata, fundamentalmente, de un paradigma distinto 
del procesamiento de información que aprovecha las propiedades no intuitivas 
de la mecánica cuántica —como la superposición y el entrelazamiento— para 
resolver problemas que son extremadamente complejos o intratables para la 
computación clásica.

Como definen Nielsen y Chuang en su texto:

\begin{quotation}
    \textit{La computación cuántica y la información cuántica son el estudio
    de las tareas de procesamiento de información que se pueden conseguir
    usando sistemas de mecánica cuántica.} \cite{nielsen_chuang}
\end{quotation}

Este campo es interdisciplinario y surge de la convergencia de dos grandes 
revoluciones científicas importantes del siglo XX: la mecánica cuántica 
y las ciencias de la computación. A continuación, se detalla el contexto 
histórico de ambas vertientes y el momento inevitable de su colisión.

\subsection{Computación cuántica y contexto histórico}

La historia de la computación cuántica es la historia de cómo los físicos 
se dieron cuenta de que la información es física, y de cómo los informáticos 
comprendieron que la computación es un proceso físico limitado por las leyes 
de la naturaleza.

\subsubsection{Mecánica cuántica}

A finales del siglo XIX, la física clásica parecía ser una área de estudio 
casi completa, regida por la mecánica newtoniana y el electromagnetismo de 
Maxwell. Sin embargo, existían discrepancias experimentales, que en ese 
tiempo se creían menores pero amenazaban a las bases de la física. 
La más notable fue la llamada \textit{Catástrofe Ultravioleta}. 

Según la teoría clásica, un cuerpo negro en equilibrio térmico debería 
emitir una cantidad infinita de energía a medida que la longitud de onda 
disminuye hacia el ultravioleta. Este resultado hizo que Max Planck, en 1900, 
postulara que la energía no es continua, sino que se intercambia en 
paquetes discretos o \textit{cuantos}.

Posteriormente, con la explicación del efecto fotoeléctrico de Einstein y 
el desarrollo del modelo atómico de Bohr, se determinó a escalas 
microscópicas la naturaleza no se comporta de forma determinista. 
La formulación matemática llegó en la década de 
1920 con Schrödinger y Heisenberg, estableciendo que el estado de un 
sistema no está definido por variables deterministas, sino por un vector 
de estado en un espacio de Hilbert complejo.

Este cambio de paradigma trajo consigo el concepto de 
\textit{superposición}: un sistema puede existir en una combinación lineal 
de múltiples estados posibles hasta que es medido. Durante décadas, estas 
propiedades se consideraron curiosidades del mundo subatómico, sin relación 
aparente con la lógica o el cálculo.

\subsubsection{Ciencias computacionales}

Paralelamente a la revolución cuántica, las matemáticas
atravesaban su propia crisis de fundamentos. En 1936, Alan Turing formalizó 
el concepto de «algoritmo» y «computación» mediante una construcción 
teórica que hoy conocemos como la \textbf{Máquina de Turing} 
\cite{turing1936computable}.

Una Máquina de Turing consiste, abstractamente, en una cinta infinita 
dividida en celdas, una cabeza lectora/escritora que se mueve sobre la 
cinta, y un conjunto finito de reglas, o estados, que dictan qué hacer en 
función del símbolo leído. La tesis de Church-Turing establece que 
cualquier función que pueda ser calculada por un algoritmo físico, puede 
ser calculada por una Máquina de Turing.

Durante casi medio siglo, la computación se desarrolló bajo la asunción 
implícita de que la Máquina de Turing universal podía simular eficientemente 
cualquier proceso físico. Sin embargo, a medida que los físicos intentaban 
simular sistemas cuánticos (como interacciones moleculares complejas) en 
computadoras clásicas, se percataron que la complejidad es exponencial. Para 
describir un sistema de $n$ partículas cuánticas, se requiere $2^n$ coeficientes
complejos, esta cifra crece tan rápido que vuelve completamente impráctico
simular sistemas pequeños de pocas moléculas, aún para sistemas de supercómputo.

\textbf{El nacimiento de la computación cuántica}

El punto de inflexión ocurrió en 1982, durante una conferencia en el MIT, cuando el 
físico Richard Feynman (e independientemente Yuri Manin) propuso una idea: 
si simular la física cuántica es computacionalmente costoso para una máquina clásica 
debido a la explosión exponencial del espacio de Hilbert, ¿por qué no construir 
una computadora que sea, en sí misma, un sistema cuántico?

En sus propias palabras, Feynman declaró:

\begin{quotation}
    \textit{Nature isn't classical, dammit, and if you want to make a simulation 
    of nature, you'd better make it quantum mechanical, and by golly it's a 
    wonderful problem, because it doesn't look so easy.} 
    \cite{feynman1982simulating}
\end{quotation}

Esta propuesta cambió el objetivo: en lugar de luchar contra la mecánica cuántica, 
tratando de minimizar efectos cuánticos en transistores cada vez más pequeños, 
se propuso aprovecharla.

En 1985, David Deutsch formalizó esta idea definiendo la \textbf{Máquina de Turing 
Cuántica Universal} \cite{deutsch1985quantum}, demostrando que tal máquina podría 
simular cualquier sistema físico de manera eficiente. La ventaja cuántica dejó 
de ser una especulación teórica en la década de los 90 con el descubrimiento de 
algoritmos que superaban a sus contrapartes clásicas:

\begin{itemize}
    \item \textbf{Algoritmo de Deutsch-Jozsa (1992):} Fue uno de los primeros 
    ejemplos deterministas de que una computadora cuántica podía resolver un 
    problema con menos consultas que una clásica.
    \item \textbf{Algoritmo de Shor (1994):} Peter Shor demostró que una 
    computadora cuántica podía factorizar números enteros en tiempo polinómico, 
    amenazando la seguridad de los sistemas criptográficos RSA actuales \cite{shor94}.
    \item \textbf{Algoritmo de Grover (1996):} Proveyó una aceleración cuadrática 
    para la búsqueda en bases de datos no estructuradas.
\end{itemize}

Estos hitos cimentaron a la computación cuántica no solo como una herramienta para 
simular física, sino como un nuevo paradigma computacional con implicaciones 
profundas en criptografía, optimización e inteligencia artificial.


\subsection{Qubits y estados cuánticos}

En la computación clásica, la unidad fundamental de información es el \textit{bit}, 
el cual puede existir en uno de dos estados discretos: 0 o 1. Por otro lado, la 
computación cuántica se construye sobre el concepto de \textbf{qubit} (quantum bit). 
Físicamente, un qubit es un sistema cuántico de dos niveles, que puede realizarse 
mediante el espín de un electrón, la polarización de un fotón o niveles de energía 
discretos en un átomo \cite{nielsen_chuang}.

Matemáticamente, el estado de un qubit, denotado como el ket $|\psi\rangle$, es un vector 
en un espacio vectorial complejo de dos dimensiones (Espacio de Hilbert $\mathcal{H} 
= \mathbb{C}^2$). A diferencia del bit clásico, el qubit puede existir en un estado 
de \textit{superposición} lineal de sus estados base:

\begin{equation}
    |\psi\rangle = \alpha |0\rangle + \beta |1\rangle
    \label{eq:superposicion}
\end{equation}

Donde $\alpha$ y $\beta$ son números complejos conocidos como \textit{amplitudes 
de probabilidad}. Los estados $|0\rangle$ y $|1\rangle$ forman la llamada 
\textbf{base computacional} (o base canónica $Z$) y se representan vectorialmente 
como:

\noindent
\begin{minipage}{0.5\textwidth}
    \begin{equation}
        |0\rangle = \begin{bmatrix} 1 \\ 0 \end{bmatrix}
        \label{eq:Vec0}
    \end{equation}
\end{minipage}%
\begin{minipage}{0.5\textwidth}
    \begin{equation}
        |1\rangle = \begin{bmatrix} 0 \\ 1 \end{bmatrix}
        \label{eq:Vec1}
    \end{equation}
\end{minipage}
\vspace{0.5cm}

Un postulado fundamental de la mecánica cuántica es que no podemos observar 
directamente las amplitudes $\alpha$ y $\beta$. Al realizar una medición en la 
base estándar, el qubit colapsa al estado $|0\rangle$ con probabilidad 
$|\alpha|^2$ o al estado $|1\rangle$ con probabilidad $|\beta|^2$. Debido a la 
conservación de la probabilidad total, se impone la condición de normalización:

\begin{equation}
    |\alpha|^2 + |\beta|^2 = 1
\end{equation}

Una herramienta visual indispensable para entender el estado de un qubit individual 
es la \textbf{Esfera de Bloch}. Dado que los factores de fase global no tienen 
consecuencias observables, podemos reescribir el estado general como $|\psi\rangle 
= \cos(\theta/2)|0\rangle + e^{i\phi}\sin(\theta/2)|1\rangle$. Esto permite mapear 
cualquier estado puro a un punto sobre la superficie de una esfera unitaria, donde 
el polo norte representa $|0\rangle$ y el polo sur $|1\rangle$.

\begin{figure}[h!]
    \centering
    \includegraphics[width=0.55\linewidth]{imagenes/bloch_sphere.png}
    \caption{Representación geométrica de un qubit en la Esfera de Bloch. Los 
    estados de superposición se encuentran en el ecuador o latitudes intermedias. 
    Tomado de Nielsen \& Chuang \cite{nielsen_chuang}.}
    \label{fig:Bloch_sphere}
\end{figure}



\subsection{Otras bases y representaciones}

Aunque la base computacional $\{|0\rangle, |1\rangle\}$ es la más utilizada para la 
medición final, los algoritmos cuánticos frecuentemente requieren rotar el estado a 
otras bases ortonormales para aprovechar la interferencia cuántica.

Una de las bases más importantes es la \textbf{base Hadamard} (o base $X$), 
compuesta por los vectores $|+\rangle$ y $|-\rangle$. Estos estados representan 
superposiciones equitativas y se definen como:

\begin{equation}
    \begin{aligned}
        |+\rangle &= \frac{|0\rangle + |1\rangle}{\sqrt{2}} \\
        |-\rangle &= \frac{|0\rangle - |1\rangle}{\sqrt{2}}
    \end{aligned}
    \label{eq:Bases_masmenos}
\end{equation}

Físicamente, si $|0\rangle$ y $|1\rangle$ representan espín arriba y abajo en el 
eje Z, $|+\rangle$ y $|-\rangle$ representan espín arriba y abajo a lo largo del 
eje X.

\subsubsection{Sistemas de múltiples qubits}
El verdadero poder de la computación cuántica surge cuando combinamos múltiples 
qubits. El espacio de estados de un sistema compuesto se describe mediante el 
\textit{producto tensorial} ($\otimes$) de los espacios individuales. Para un 
sistema de $n$ qubits, el espacio resultante tiene dimensión $2^n$.

Por convención, a menudo usamos la notación decimal para simplificar la 
representación de los estados base. Por ejemplo, para un sistema de 5 qubits 
($n=5$), el estado binario $|10100\rangle$ se puede escribir como:

\begin{equation*}
    |10100\rangle \equiv |1\rangle \otimes |0\rangle \otimes |1\rangle \otimes 
    |0\rangle \otimes |0\rangle \equiv |20\rangle
\end{equation*}

Esto es crucial para algoritmos como el de Shor, donde manipulamos registros 
que representan números enteros grandes mediante estados cuánticos.

\subsection{Compuertas y circuitos cuánticos}

De manera análoga a la computación clásica, donde las puertas lógicas (AND, OR, 
NOT) manipulan bits, en la computación cuántica se utilizan \textbf{compuertas 
cuánticas} para manipular qubits. Sin embargo, existe una restricción fundamental: 
debido a que la evolución temporal de un sistema cuántico cerrado debe preservar 
la norma del vector de estado (conservación de la probabilidad), todas las 
compuertas cuánticas deben ser representadas por \textbf{matrices unitarias}.

Una matriz $U$ es unitaria si su inversa es igual a su transpuesta conjugada 
($U^\dagger U = I$). Esto implica que, a diferencia de muchas compuertas clásicas, 
\textbf{todas las compuertas cuánticas son reversibles}.

A continuación, se presentan las compuertas de un solo qubit más relevantes, las 
cuales forman los bloques constructivos de algoritmos complejos 
\cite{nielsen_chuang}:

\subsubsection{Matrices de Pauli}
Son fundamentales tanto en computación como en la física.

\noindent
\begin{minipage}{0.32\textwidth}
    \centering
    \textbf{Identidad ($I$)}
    \begin{equation}
        I = \begin{pmatrix} 1 & 0 \\ 0 & 1 \end{pmatrix}
        \label{eq:identidad}
    \end{equation}
    No altera el estado.
\end{minipage}%
\hfill
\begin{minipage}{0.32\textwidth}
    \centering
    \textbf{Pauli-X (NOT)}
    \begin{equation}
        X = \begin{pmatrix} 0 & 1 \\ 1 & 0 \end{pmatrix}
        \label{eq:PauliX}
    \end{equation}
    Invierte las amplitudes: $\alpha|0\rangle + \beta|1\rangle \to \beta|0\rangle 
    + \alpha|1\rangle$.
\end{minipage}%
\hfill
\begin{minipage}{0.32\textwidth}
    \centering
    \textbf{Pauli-Z (Phase)}
    \begin{equation}
        Z = \begin{pmatrix} 1 & 0 \\ 0 & -1 \end{pmatrix}
        \label{eq:PauliZ}
    \end{equation}
    Introduce una fase relativa: $\alpha|0\rangle + \beta|1\rangle \to 
    \alpha|0\rangle - \beta|1\rangle$.
\end{minipage}

\vspace{0.5cm}

\subsubsection{Compuertas de Superposición y Fase}

\noindent
\begin{minipage}{0.48\textwidth}
    \centering
    \textbf{Compuerta Hadamard ($H$)}
    \begin{equation}
        H = \frac{1}{\sqrt{2}} \begin{pmatrix} 1 & 1 \\ 1 & -1 \end{pmatrix}
        \label{eq:Hadamard}
    \end{equation}
    Es una compuerta muy importante, ya que crea superposición. 
    Transforma estados base definidos en estados de probabilidad equitativa: 
    $|0\rangle \to |+\rangle$.
\end{minipage}%
\hfill
\begin{minipage}{0.48\textwidth}
    \centering
    \textbf{Compuerta T ($\pi/8$)}
    \begin{equation}
        T = \begin{pmatrix} 1 & 0 \\ 0 & e^{i \pi /4} \end{pmatrix}
        \label{eq:PhaseT}
    \end{equation}
    Añade una fase arbitraria necesaria para la computación cuántica universal.
\end{minipage}

\subsubsection{Circuitos Cuánticos}
Los algoritmos se representan visualmente mediante diagramas de circuitos. 
En estos diagramas, las líneas horizontales representan la evolución temporal 
de los qubits (de izquierda a derecha), y los bloques representan la aplicación 
de compuertas. A diferencia de los circuitos eléctricos, las líneas no son cables 
físicos, sino la trayectoria de la información cuántica en el tiempo.

\begin{figure}[H]
    \centering
    \includegraphics[width=0.90\linewidth]{imagenes/Quantum_circuit.png}
    \caption{Ejemplo esquemático de un circuito cuántico. La lectura se realiza 
    de izquierda (entrada) a derecha (medición). Tomado de Nielsen \& Chuang 
    \cite{nielsen_chuang}.}
    \label{fig:quantum_circuit_1}
\end{figure}


\section{Algoritmos cuánticos}

La promesa de la computación cuántica no reside en aumentar la velocidad de reloj 
de los procesadores, sino en cambiar la complejidad computacional fundamental de 
los problemas. Mientras que la computación clásica procesa datos de manera 
secuencial (o con paralelismo físico limitado), los algoritmos cuánticos aprovechan 
las propiedades del espacio de Hilbert para evaluar múltiples posibilidades 
simultáneamente.

\subsection{Paralelismo}
Una pregunta fundamental en la optimización de algoritmos es: \textit{¿Cómo podemos 
resolver un problema más rápido?} Tradicionalmente, esto se reestructura como: 
\textbf{¿Cómo podemos hacer muchas cosas a la vez?}

En el cómputo clásico, distinguimos dos tipos de paralelismo:

\begin{itemize}
    \item \textbf{Paralelismo trivial (Embarrassingly parallel):} Ocurre en tareas 
    donde los sub-problemas son completamente independientes. Un ejemplo es la suma 
    de un arreglo numérico gigante; podemos dividir el arreglo en lotes, enviarlos 
    a diferentes núcleos (CPUs) y sumar los resultados finales. Aquí, añadir más 
    hardware escala linealmente el rendimiento.
    
    \item \textbf{Paralelismo no trivial (Non-embarrassingly parallel):} Ocurre en 
    problemas donde existe una fuerte dependencia entre los datos. Por ejemplo, en 
    la inversión de una matriz, cada elemento de la matriz inversa depende 
    potencialmente de todos los elementos de la matriz original. En estos casos, 
    simplemente añadir más procesadores no resuelve la complejidad inherente del 
    algoritmo.
\end{itemize}

La computación cuántica introduce un tercer tipo: el \textbf{Paralelismo Cuántico}. 
Gracias al principio de superposición, una computadora cuántica puede evaluar una 
función $f(x)$ para todos los posibles valores de entrada $x$ simultáneamente, 
utilizando un solo circuito. Sin embargo, el reto no es calcular todos los 
resultados, sino acceder a ellos, ya que la medición colapsa el estado a una sola 
respuesta aleatoria. El arte del diseño de algoritmos cuánticos consiste en 
manipular estas probabilidades para que la respuesta correcta emerja.

\subsection{Estructura general de un algoritmo cuántico}
A pesar de la variedad de aplicaciones, la mayoría de los algoritmos cuánticos 
siguen una arquitectura canónica de cuatro etapas \cite{nielsen_chuang}:

\begin{enumerate}
    \item \textbf{Inicialización:} Se prepara el registro de qubits en un estado 
    cuántico conocido y controlable, típicamente el estado base $|0\rangle^{\otimes 
    n}$.
    \item \textbf{Procesamiento Cuántico:} Se aplica una secuencia de compuertas 
    lógicas (transformaciones unitarias). En esta etapa se generan superposiciones 
    para explorar el espacio de soluciones y se utiliza el entrelazamiento para 
    correlacionar qubits.
    \item \textbf{Medición:} Se realiza una medición en la base computacional. 
    Esto rompe la coherencia del sistema, colapsando la función de onda y 
    entregando un resultado clásico (bit).
    \item \textbf{Interpretación:} Los resultados clásicos obtenidos (a menudo 
    estadísticos, requiriendo múltiples ejecuciones o «shots») se traducen a la 
    solución del problema original.
\end{enumerate}

\subsection{Algoritmo de Deutsch}
Propuesto por David Deutsch en 1985, este fue el primer algoritmo en demostrar una 
separación entre lo que es posible clásicamente y lo que es posible cuánticamente.

\textbf{El Problema:} Dada una función booleana de una variable $f: \{0,1\} \to 
\{0,1\}$ (una «caja negra» u oráculo), determinar si la función es:
\begin{itemize}
    \item \textbf{Constante:} $f(0) = f(1)$ (ambos son 0 o ambos son 1).
    \item \textbf{Balanceada:} $f(0) \neq f(1)$ (uno es 0 y el otro 1).
\end{itemize}

Clásicamente, necesitamos evaluar la función dos veces ($f(0)$ y luego $f(1)$) 
para comparar los resultados. El algoritmo de Deutsch resuelve esto con \textbf{una 
sola evaluación} del oráculo.

\subsubsection{El mecanismo de Retroceso de Fase (Phase Kickback)}

Un concepto fundamental para comprender cómo los algoritmos cuánticos extraen 
información global de una función es el \textit{Phase Kickback}. Clásicamente, 
consultar una función (leer un bit) no modifica la entrada. Sin embargo, en 
mecánica cuántica, la aplicación de una compuerta controlada puede alterar el 
estado del qubit de control dependiendo del estado del qubit objetivo.

Consideremos el oráculo unitario estándar $U_f$ que actúa como $U_f |x\rangle 
|y\rangle = |x\rangle |y \oplus f(x)\rangle$. Si preparamos el qubit objetivo 
$|y\rangle$ no en un estado base, sino en el estado de superposición $|-\rangle 
= \frac{|0\rangle - |1\rangle}{\sqrt{2}}$, ocurre un fenómeno de interferencia 
interesante.

Al aplicar el operador $U_f$:
\begin{equation}
    U_f |x\rangle |-\rangle = |x\rangle \left( \frac{|0 \oplus f(x)\rangle - 
    |1 \oplus f(x)\rangle}{\sqrt{2}} \right)
\end{equation}

Analizando el comportamiento según el valor de $f(x)$:
\begin{itemize}
    \item Si $f(x) = 0$, el estado del segundo qubit permanece inalterado: 
    $|-\rangle$.
    \item Si $f(x) = 1$, se aplica una operación NOT, invirtiendo los estados: 
    $\frac{|1\rangle - |0\rangle}{\sqrt{2}} = -|-\rangle$.
\end{itemize}

Podemos unificar ambos casos factorizando el signo. El estado $|-\rangle$ es, 
de hecho, un eigenestado del operador $X$ con eigenvalor $-1$. Por lo tanto, 
el oráculo escribe este eigenvalor en la fase global del sistema:

\begin{equation}
    U_f |x\rangle |-\rangle = (-1)^{f(x)} |x\rangle |-\rangle
\end{equation}

Crucialmente, dado que el qubit objetivo $|-\rangle$ permanece inalterado 
(salvo por el signo), podemos reasignar este factor de fase al qubit de 
control $|x\rangle$. De esta forma, la información de la función $f(x)$, 
que originalmente estaba codificada en el bit de paridad del objetivo, ha 
«retrocedido» (kicked back) hacia la fase relativa del registro de entrada, 
permitiéndonos medirla posteriormente mediante interferencia.

\textbf{Procedimiento:}
El algoritmo utiliza dos qubits inicializados en $|0\rangle|1\rangle$.
\begin{enumerate}
    \item \textbf{Superposición:} Aplicamos compuertas Hadamard a ambos qubits 
    ($H \otimes H$), creando un estado de superposición equitativa.
    \item \textbf{Oráculo ($U_f$):} Aplicamos la función mediante una 
    transformación unitaria. Debido al fenómeno de \textit{retroceso de fase}, 
    la información de $f(x)$ no se escribe en el qubit objetivo, 
    sino que se transfiere a la fase relativa del qubit de control:
    \begin{equation}
        |x\rangle \left( \frac{|0\rangle - |1\rangle}{\sqrt{2}} \right) 
        \xrightarrow{U_f} (-1)^{f(x)} |x\rangle \left( \frac{|0\rangle - 
        |1\rangle}{\sqrt{2}} \right)
    \end{equation}
    \item \textbf{Interferencia:} Aplicamos una compuerta Hadamard final al 
    primer qubit. La interferencia destructiva o constructiva dependerá de si los 
    términos $(-1)^{f(0)}$ y $(-1)^{f(1)}$ tienen el mismo signo o no.
    \item \textbf{Medición:} Medimos el primer qubit.
\end{enumerate}

\textbf{Resultado:} Si medimos $|0\rangle$, la función es constante. Si medimos 
$|1\rangle$, es balanceada. Esto demuestra una ventaja de velocidad de $2$ a $1$ 
sobre la computación clásica.

\subsection{Algoritmo de Deutsch-Jozsa}
Este algoritmo generaliza el problema anterior para una función de $n$ variables 
$f: \{0,1\}^n \to \{0,1\}$. Nuevamente, se garantiza que la función es constante 
(misma salida para todas las entradas) o balanceada (0 para la mitad de las 
entradas, 1 para la otra mitad).

\textbf{Comparativa de complejidad:}
\begin{itemize}
    \item \textbf{Clásica:} En el peor de los casos, necesitamos evaluar la 
    función $2^{n-1} + 1$ veces para estar 100\% seguros.
    \item \textbf{Cuántica:} El algoritmo de Deutsch-Jozsa resuelve el problema 
    con \textbf{una sola evaluación} del oráculo, ofreciendo una ventaja 
    exponencial determinista.
\end{itemize}

\textbf{Desarrollo Matemático:}
\begin{enumerate}
    \item \textbf{Inicialización:} Comenzamos con un registro de $n$ qubits en 
    $|0\rangle$ y un qubit auxiliar en $|1\rangle$:
    \begin{equation}
        |\psi_0\rangle = |0\rangle^{\otimes n} |1\rangle
    \end{equation}
    
    \item \textbf{Superposición:} Aplicamos compuertas Hadamard a todo el 
    registro. Esto genera una superposición de todos los $2^n$ estados posibles en 
    el registro de entrada:
    \begin{equation}
        |\psi_1\rangle = \frac{1}{\sqrt{2^n}} \sum_{x=0}^{2^{n}-1}|x\rangle 
        \left[ \frac{|0\rangle - |1\rangle}{\sqrt{2}} \right]
    \end{equation}
    
    \item \textbf{Consulta al Oráculo:} Aplicamos $U_f$. Al igual que en el 
    algoritmo de Deutsch, el qubit auxiliar introduce un signo $(-1)^{f(x)}$ a 
    cada estado base $|x\rangle$:
    \begin{equation}
        |\psi_2\rangle = \sum_{x=0}^{2^{n}-1} \frac{(-1)^{f(x)}|x\rangle}
        {\sqrt{2^n}} \left[ \frac{|0\rangle - |1\rangle}{\sqrt{2}} \right]
    \end{equation}
    
    \item \textbf{Interferencia Final:} Aplicamos compuertas Hadamard nuevamente 
    a los primeros $n$ qubits. La transformación de Hadamard sobre un estado base 
    $|x\rangle$ se define como $\sum_z (-1)^{x \cdot z} |z\rangle$, donde $x \cdot 
    z$ es el producto punto bit a bit módulo 2. Sustituyendo esto:
    \begin{equation}
        |\psi_3\rangle = \sum_{z=0}^{2^{n}-1} \sum_{x=0}^{2^{n}-1} 
        \frac{(-1)^{x \cdot z + f(x)} |z\rangle}{2^n} \otimes \left[ \frac{|0
        \rangle - |1\rangle}{\sqrt{2}} \right]
    \end{equation}
    
    \item \textbf{Medición:} Medimos los primeros $n$ qubits. Nos interesa la 
    amplitud del estado $|0\rangle^{\otimes n}$ (donde $z=0$):
    \begin{equation}
        A_{z=0} = \sum_{x} \frac{(-1)^{f(x)}}{2^n}
    \end{equation}
    \begin{itemize}
        \item Si $f$ es \textbf{constante}, todos los términos tienen el mismo 
        signo ($+1$ o $-1$), sumando constructivamente a una amplitud de $\pm 1$ 
        (probabilidad 1).
        \item Si $f$ es \textbf{balanceada}, la mitad de los términos son $+1$ y 
        la otra mitad $-1$, cancelándose perfectamente a 0 (probabilidad 0).
    \end{itemize}
\end{enumerate}

Por lo tanto, si medimos el estado $|00\dots0\rangle$, la función es constante; 
cualquier otro resultado implica que es balanceada.

\begin{figure}[h!]
    \centering
    \includegraphics[width=1\linewidth]{imagenes/Deutsch-Josza.png}
    \caption{Circuito cuántico que implementa el algoritmo general de Deutsch-Jozsa. 
    Se observa la estructura de Hadamard-Oráculo-Hadamard característica de 
    algoritmos de interferencia. Tomado de Nielsen \& Chuang \cite{nielsen_chuang}.}
    \label{fig:Deutsch-Josza}
\end{figure}

\noindent \textbf{Tiempo de ejecución (Runtime):} $O(1)$ cuántico vs. $O(2^n)$ 
clásico.

\subsection{Transformada Cuántica de Fourier (QFT)}
Una de las estrategias más poderosas en matemáticas y ciencias de la computación consiste en transformar un problema difícil en otro dominio donde la solución es conocida o más sencilla de calcular. De manera análoga a la Transformada Discreta de Fourier (DFT) en el procesamiento de señales clásico, la computación cuántica utiliza la \textbf{Transformada Cuántica de Fourier (QFT)}.

La QFT es una transformación lineal sobre los qubits que mapea un estado de la base computacional $|x\rangle$ a una superposición de todos los estados de la base, con fases determinadas por la siguiente expresión matemática:

\begin{equation}
    QFT |x\rangle = \frac{1}{\sqrt{N}} \sum_{k=0}^{N-1} e^{2\pi i xk/N} |k\rangle
    \label{eq:QFT}
\end{equation}

Donde $N = 2^n$ es la dimensión del espacio de Hilbert para $n$ qubits.

Aunque la QFT no suele ser el resultado final de un algoritmo, es la subrutina 
necesaria para la \textbf{Estimación de Fase Cuántica} (Quantum Phase Estimation). 
Esta capacidad de extraer la información de la fase global o relativa de un sistema 
es el fundamento que impulsa algoritmos más complejos, siendo el componente clave en la 
construcción del algoritmo de Shor \cite{nielsen_chuang}.

\subsection{Algoritmo de Shor}
Publicado en 1994, el algoritmo de Shor aborda uno de los problemas fundamentales 
de la teoría de números: la factorización de enteros.

\textbf{El Problema:} Dado un número entero compuesto $N$, el objetivo es 
descomponerlo eficientemente en sus factores primos no triviales. La dificultad 
de este problema para computadoras clásicas es la base de la seguridad del 
cifrado RSA.

\textbf{Procedimiento:}
El algoritmo es híbrido, combinando pre-procesamiento clásico con una subrutina 
cuántica de búsqueda de periodo:
\begin{enumerate}
    \item \textbf{Selección aleatoria:} Se elige un número entero aleatorio $a$ 
    tal que $a < N$.
    \item \textbf{Verificación inicial:} Se calcula el Máximo Común Divisor (MCD) 
    usando el algoritmo de Euclides clásico. Si $\gcd(a, N) > 1$, hemos encontrado 
    un factor por suerte y el algoritmo termina.
    \item \textbf{Búsqueda de Periodo (Subrutina Cuántica):} Si no se encontró el 
    factor, utilizamos la QFT para encontrar el periodo $r$ de la función modular 
    $f(x) = a^x \pmod N$. Este es el paso que otorga la ventaja exponencial.
    \item \textbf{Cálculo de Factores:} Una vez obtenido el periodo $r$ (y 
    verificando que sea par), se calculan los factores mediante la fórmula clásica 
    $\gcd(a^{r/2} \pm 1, N)$.
\end{enumerate}



\textbf{Complejidad:}
Mientras que los mejores algoritmos clásicos operan en tiempo sub-exponencial, el 
algoritmo de Shor tiene una complejidad de $O((\log N)^3)$. Al ser un tiempo 
polinómico respecto al número de bits del entero ($\log N$), representa una amenaza 
real para la criptografía actual.

\subsection{Algoritmo de Grover}
Propuesto por Lov Grover en 1996, este algoritmo ofrece una solución al problema de 
búsqueda en bases de datos no estructuradas.

\textbf{El Problema:} Dada una base de datos desordenada con $N$ elementos, 
encontrar el índice de un elemento específico que cumple una condición dada por un 
oráculo.

\textbf{Procedimiento:}
El algoritmo se basa en la técnica de \textit{amplificación de amplitud}:
\begin{enumerate}
    \item \textbf{Inicialización:} Se prepara el registro de $n$ qubits, más un 
    auxiliar, en el estado base: $|\psi_0\rangle \rightarrow |0\rangle^{\otimes n} 
    |0\rangle$.
    
    \item \textbf{Superposición:} Se aplica una compuerta Hadamard a todo el 
    registro para crear una superposición uniforme. El último qubit se pone en 
    el estado $|-\rangle$ (usando $H$ y $X$) para habilitar el retroceso de fase:
    \begin{equation}
        |\psi_1\rangle \rightarrow \frac{1}{\sqrt{2^n}} \sum_{x=0}^{2^{n}-1}
        |x\rangle \left[ \frac{|0\rangle - |1\rangle}{\sqrt{2}} \right]
    \end{equation}
    
    \item \textbf{Iteración de Grover:} Se aplica repetidamente el operador de 
    Grover $G$. El número óptimo de repeticiones es $R \approx [\frac{\pi}{4}
    \sqrt{2^n}]$. Matemáticamente, cada iteración consiste en aplicar el oráculo 
    $O$ seguido del operador de difusión $(2 \|\psi \rangle \bra{\psi}-I)$:
    \begin{equation}
        |\psi_2 \rangle \rightarrow \left[ (2 \| \psi \rangle \bra{\psi}-I)O 
        \right]^R 
        \frac{1}{\sqrt{2^n}} \sum_{x=0}^{2^{n}-1}|x\rangle \left[ \frac{|0\rangle 
        - |1\rangle}{\sqrt{2}} \right]
    \end{equation}
    Geométricamente, esto rota el vector de estado acercándolo al estado solución 
    deseado.
    
    \item \textbf{Medición:} Se miden los primeros $n$ qubits para obtener el 
    índice $x_0$ con alta probabilidad.
\end{enumerate}

\textbf{Complejidad:}
El algoritmo encuentra el elemento en $O(\sqrt{N})$ pasos, proporcionando una 
aceleración cuadrática respecto al $O(N)$ clásico.

\subsection{Comparación de los algoritmos}
A continuación, se presenta una comparativa técnica de los cuatro algoritmos 
fundamentales discutidos, contrastando sus entradas, salidas y la complejidad 
computacional asintótica.

\begin{table}[H]
    \centering
    \renewcommand{\arraystretch}{1.2}
    \begin{tabular}{|l|c|c|c|}
        \hline
        \textbf{Algoritmo} & \textbf{Entrada} & \textbf{Salida} & 
        \textbf{Tiempo de Ejecución} \\ \hline
        Deutsch & Función $f(x)$ & Constante/Balanceada & $O(1)$ \\ \hline
        Deutsch-Jozsa & Función $f(x)$ ($n$ bits) & Constante/Balanceada & $O(1)$ \\ \hline
        Shor & Entero $N$ & Factor de $N$ & $O((\log N)^3)$ \\ \hline
        Grover & Base de datos tamaño $N$ & Índice objetivo & $O(\sqrt{N})$ \\ \hline
    \end{tabular}
    \caption{Comparación entre los 4 algoritmos principales. Nótese la 
    diferencia entre ventajas constantes, polinómicas y cuadráticas.}
    \label{tab:comp_algorithm}
\end{table}

\subsection{Perspectivas futuras y aplicaciones}

El desarrollo de la computación cuántica abre un abanico de aplicaciones que 
prometen transformar industrias enteras, aunque su implementación conlleva 
desafíos significativos.

\textbf{Áreas de impacto principal:}
\begin{itemize}
    \item \textbf{Criptografía:} La capacidad del algoritmo de Shor para romper 
    el cifrado RSA está impulsando la transición hacia la \textit{Criptografía 
    Post-Cuántica} (PQC), buscando protocolos resistentes a ataques cuánticos.
    \item \textbf{Optimización:} El algoritmo de Grover y sus variantes ofrecen 
    mejoras cuadráticas para problemas combinatorios complejos, útiles en logística 
    y finanzas.
    \item \textbf{Simulación Cuántica:} Como predijo Feynman, la aplicación más 
    natural es la simulación de sistemas químicos y moleculares para el 
    descubrimiento de nuevos materiales y fármacos.
\end{itemize}

\subsection{Conclusiones}
La computación cuántica ha transitado de ser una curiosidad teórica a una 
tecnología emergente con aplicaciones definidas en criptografía, optimización e 
inteligencia artificial. La demostración de la \textit{Supremacía Cuántica} por 
parte de procesadores como Sycamore de Google marca el inicio de una era donde 
estas máquinas pueden resolver tareas intratables clásicamente.

Sin embargo, persisten grandes desafíos de ingeniería, principalmente la 
escalabilidad del hardware (aumentar el número de qubits) y la corrección de 
errores cuánticos para mitigar la decoherencia. A pesar de estos obstáculos, los 
avances continuos en hardware y la adopción comercial sugieren que la computación 
cuántica se consolidará como una tecnología clave para la resolución de problemas 
complejos y las comunicaciones seguras en las próximas décadas.

\chapter{Proyecto en computación cuántica}
\label{cap:shor}

El objetivo central de este capítulo es la aplicación práctica de los 
conceptos de computación cuántica en el campo de la criptografía, 
específicamente en el análisis de vulnerabilidades del sistema RSA mediante 
el algoritmo de Shor. Para comprender la magnitud de este desafío y la 
estructura de la solución cuántica, es imperativo desglosar el problema 
en dos pilares fundamentales: 1) la estructura matemática del sistema 
RSA y 2) la Transformada Cuántica de Fourier como herramienta de estimación 
de fase.

La motivación primordial del algoritmo de Shor reside en su capacidad 
para comprometer la seguridad del cifrado RSA. Dado que RSA no es 
simplemente una secuencia de pasos algorítmicos, sino un sistema 
fundamentado en la teoría de números, esta sección comienza estableciendo 
las bases teóricas de la criptografía y la aritmética modular. Solo a 
través de la comprensión profunda de estos principios es posible 
identificar qué es exactamente lo que el algoritmo de Shor busca resolver.

La conexión entre el mundo clásico y el cuántico se establece mediante 
la reducción del problema de factorización a un problema de búsqueda de 
periodo. Para resolver esto, se introduce la Transformada Cuántica de 
Fourier (QFT) no como un fin en sí mismo, sino como el motor necesario 
para la \textit{Estimación de Fase Cuántica} (QPE). Es esta estimación 
la que permite explotar las propiedades de superposición e interferencia 
para resolver ecuaciones modulares con una mejora exponencial en el tiempo 
de ejecución respecto a los métodos clásicos.

Finalmente, el capítulo culmina con la integración de estos componentes 
en el algoritmo de Shor. Se complementa el análisis teórico con 
implementaciones prácticas en Python, replicando tanto el sistema de 
cifrado RSA como la subrutina cuántica de búsqueda de periodos. Estas 
simulaciones permiten visualizar el crecimiento de los recursos 
computacionales requeridos y evidencian la ventaja cuántica al optimizar 
la resolución del problema de factorización.

\section{Criptografía}

La criptografía, etimológicamente proveniente del griego \textit{kryptós} 
(oculto) y \textit{graphein} (escribir), se definía clásicamente como el 
arte de escribir o resolver códigos. Sin embargo, esta definición ha 
evolucionado drásticamente en la era de la información.

Según Katz y Lindell \cite{katz2020introduction}, la criptografía moderna 
ya no se limita al secreto, sino que constituye el estudio científico de 
técnicas matemáticas para asegurar la información digital, los sistemas y 
las computaciones distribuidas frente a adversarios maliciosos. Ya no es 
un arte subjetivo, sino una ciencia rigurosa basada en pruebas de seguridad 
y complejidad computacional.

\subsection{Contexto histórico}
La historia de la criptografía es una «carrera armamentista» constante 
entre los codificadores (criptógrafos) y los rompedores de códigos 
(criptoanalistas).

Uno de los primeros ejemplos documentados es el \textbf{Cifrado César}, 
utilizado por Julio César para proteger comunicaciones militares. Este 
método es un cifrado de sustitución simple donde cada letra se desplaza 
un número fijo de posiciones en el alfabeto. Aunque efectivo en su época 
debido al analfabetismo generalizado, hoy es trivial de romper.

Un punto de inflexión crucial ocurrió en el siglo XVI con la ejecución de 
María Estuardo, Reina de Escocia. Como narra Singh \cite{singh1999code}, 
María utilizaba un cifrado de sustitución para comunicarse con sus 
conspiradores desde la prisión, creyendo que sus mensajes eran seguros. 
Sin embargo, Sir Francis Walsingham, el jefe de espías de la reina Isabel I, 
empleó el \textbf{análisis de frecuencia} para descifrar los mensajes. 
Este método estadístico se basa en que, en cualquier idioma, ciertas letras 
aparecen con mayor frecuencia que otras (por ejemplo, la 'e' en inglés o 
la 'a' y 'e' en español). El descifrado de estos mensajes no solo reveló 
la conspiración de Babington, sino que condujo directamente a la ejecución 
de María, demostrando que la seguridad por oscuridad no es suficiente.

A lo largo del siglo XX, con la mecanización (máquina Enigma) y la 
digitalización, la criptografía transitó de la lingüística a las 
matemáticas, culminando en los algoritmos modernos que no dependen del 
secreto del método, sino de la complejidad computacional de las claves.

\subsection{Metas básicas de la criptografía}
Cualquier sistema criptográfico robusto busca satisfacer, total o 
parcialmente, los siguientes objetivos de seguridad de la información 
\cite{stinson2005cryptography}:

\begin{itemize}
    \item \textbf{Confidencialidad:} Garantizar que la información sea 
    accesible únicamente por las partes autorizadas. Es el concepto  
    de «secreto».
    \item \textbf{Integridad:} Asegurar que los datos no han sido alterados, 
    manipulados o corrompidos durante su transmisión o almacenamiento por 
    una entidad no autorizada.
    \item \textbf{Autenticación:} Proveer mecanismos para verificar la 
    identidad de las partes involucradas en la comunicación (confirmar que 
    «Alice» es realmente quien dice ser).
    \item \textbf{No repudio:} Evitar que una entidad niegue la autoría de 
    un mensaje o acción que realizó previamente. Esto es fundamental en 
    contextos legales y financieros.
\end{itemize}

\subsection{Escenario básico}
Para formalizar el estudio de los criptosistemas, se utiliza un modelo 
estándar de comunicación. Supongamos dos partes, tradicionalmente llamadas 
\textbf{Alice} y \textbf{Bob}, que desean comunicarse a través de un canal 
inseguro donde un adversario, \textbf{Eve} (de \textit{eavesdropper}), 
puede interceptar los mensajes.

Un esquema de cifrado se define como una tupla de algoritmos 
probabilísticos $(\mathcal{G}, \mathcal{E}, \mathcal{D})$ sobre un espacio 
de mensajes $\mathcal{M}$, un espacio de claves $\mathcal{K}$ y un espacio 
de textos cifrados $\mathcal{C}$:

\begin{enumerate}
    \item Alice posee un \textit{mensaje} o \textit{texto plano} $M 
    \in \mathcal{M}$.
    \item Utiliza una función de encriptación $e$ y una clave $k$ para 
    generar el \textit{texto cifrado}:
    \begin{equation}
        C = e_k(M)
    \end{equation}
    \item Alice envía $C$ a través del canal inseguro.
    \item Bob recibe $C$ y utiliza una función de desencriptación $d$ 
    con la clave correspondiente para recuperar el mensaje original:
    \begin{equation}
        M = d_k(C) = d_k(e_k(M))
    \end{equation}
\end{enumerate}

La condición fundamental es que la recuperación de $M$ debe ser 
computacionalmente inviable para Eve sin el conocimiento de la clave $k$.



\subsection{Criptografía simétrica y asimétrica}
Dependiendo de la naturaleza de las claves utilizadas, los criptosistemas 
se dividen en dos grandes familias:

\begin{itemize}
    \item \textbf{Criptografía simétrica (Clave Privada):} Alice y Bob 
    comparten una única clave secreta $k$ previamente acordada. Esta misma 
    clave se usa tanto para encriptar como para desencriptar 
    ($k_{enc} = k_{dec}$).
    \begin{itemize}
        \item \textit{Ventaja:} Es computacionalmente muy eficiente y rápida 
        (ej. AES).
        \item \textit{Desventaja:} Sufre del problema de distribución de 
        claves. ¿Cómo comparten Alice y Bob la clave secreta inicialmente 
        sin que Eve la intercepte?
    \end{itemize}
    
    \item \textbf{Criptografía Asimétrica (Clave Pública):} Cada usuario 
    genera un par de claves: una \textbf{clave pública} ($pk$) que puede 
    ser distribuida abiertamente, y una \textbf{clave privada} ($sk$) que 
    debe mantenerse en secreto.
    \begin{itemize}
        \item Si Alice quiere enviar un mensaje a Bob, usa la clave pública 
        de Bob ($pk_{Bob}$) para encriptar.
        \item Solo Bob puede desencriptar el mensaje usando su clave privada 
        ($sk_{Bob}$).
        \item \textit{Ventaja:} Resuelve el problema de distribución de claves 
        y permite firmas digitales.
        \item \textit{Desventaja:} Es órdenes de magnitud más lenta que la 
        simétrica y requiere claves más largas.
    \end{itemize}
\end{itemize}

El sistema RSA, que se abordará más adelante, es el exponente más famoso de 
la criptografía asimétrica.

\section{Fundamentos matemáticos}

La seguridad de los algoritmos de clave pública como RSA no se basa en la 
complejidad estructural del algoritmo, sino en la dificultad de resolver 
ciertos problemas matemáticos en teoría de números, específicamente aquellos 
relacionados con la factorización de enteros grandes. Para ello, es 
necesario establecer las bases de la aritmética modular.

\subsection{Aritmética modular}
Formalmente, la aritmética modular se fundamenta en las propiedades de 
divisibilidad de los números enteros. Su base rigurosa reside en el 
\textbf{Algoritmo de la División}, el cual establece que dados dos enteros 
$a$ y $N$ (con $N > 0$), existen dos enteros únicos $q$ (cociente) y 
$r$ (residuo) tales que:

\begin{equation}
    a = qN + r, \quad \text{donde } 0 \le r < N
\end{equation}

En este contexto, definimos el operador módulo, denotado como $a \bmod N$, 
como el único entero $r$ que satisface la ecuación anterior.

Más allá del operador, en criptografía es crucial la relación de 
\textbf{congruencia}. Decimos que dos enteros $a$ y $b$ son congruentes 
módulo $N$, denotado como:

\begin{equation}
    a \equiv b \pmod N
\end{equation}

si y solo si la diferencia $(a - b)$ es un múltiplo entero de $N$. Es 
decir, $N \mid (a-b)$.

Esta relación de congruencia es una \textit{relación de equivalencia}, 
lo que permite particionar el conjunto de los números enteros $\mathbb{Z}$ 
en $N$ clases de equivalencia disjuntas. El conjunto de estas clases forma 
el anillo de enteros módulo $N$, denotado como $\mathbb{Z}_N$.

\subsection{El grupo multiplicativo $\mathbb{Z}_N^*$}
 
Necesitamos un subconjunto de números que tengan propiedades especiales 
respecto a la multiplicación, específicamente, que posean un 
\textit{inverso multiplicativo}.

Definimos el grupo multiplicativo $\mathbb{Z}_N^*$ como el conjunto de 
enteros positivos menores que $N$ que son coprimos (relativamente primos) 
con $N$:

\begin{equation}
    \mathbb{Z}_N^* = \{ b \in \{1, \dots, N-1\} \mid \gcd(b, N) = 1 \}
\end{equation}

Este conjunto forma un \textbf{grupo abeliano} bajo la operación de 
multiplicación módulo $N$. Las propiedades que lo hacen vital para RSA son:

\begin{enumerate}
    \item \textbf{Cerradura:} Si $a, b \in \mathbb{Z}_N^*$, entonces 
    $(a \cdot b) \bmod N \in \mathbb{Z}_N^*$.
    \item \textbf{Existencia de Inverso:} Para todo $a \in \mathbb{Z}_N^*$, 
    existe un único elemento $a^{-1} \in \mathbb{Z}_N^*$ tal que:
    \begin{equation}
        a \cdot a^{-1} \equiv 1 \pmod N
    \end{equation}
\end{enumerate}

La existencia de este inverso $a^{-1}$ es lo que permite el proceso de 
desencriptación. Si $N$ es un número primo $p$, entonces todos los 
números desde $1$ hasta $p-1$ son coprimos con $p$, por lo tanto 
$|\mathbb{Z}_p^*| = p-1$. Sin embargo, si $N$ es compuesto (como en RSA), 
el tamaño del grupo está dado por la función $\phi(N)$ de Euler, concepto 
que será central para la generación de llaves.

\subsection{Algoritmo de Euclides}
El Algoritmo de Euclides es un método eficiente y antiguo para calcular el 
Máximo Común Divisor (MCD) de dos números enteros, denotado como $\gcd(a, b)$. 
Su importancia en criptografía es doble: permite verificar si dos números son 
coprimos (requisito para generar claves RSA) y es el paso final para extraer los 
factores en el algoritmo de Shor.

El algoritmo se basa en el principio de que el MCD de dos números no cambia si al 
mayor se le resta el menor. De manera recursiva:
\begin{equation}
    \gcd(a, b) = \gcd(b, a \bmod b)
\end{equation}
El proceso se repite hasta que el residuo es 0. El último residuo no nulo es el MCD.

\textbf{Ejemplo:} Calcular $\gcd(48, 18)$.
\begin{align*}
    48 &= 2 \times 18 + 12 & (\gcd(48, 18) \to \gcd(18, 12)) \\
    18 &= 1 \times 12 + 6  & (\gcd(18, 12) \to \gcd(12, 6)) \\
    12 &= 2 \times 6 + 0   & (\text{Residuo 0, fin.})
\end{align*}
Por lo tanto, $\gcd(48, 18) = 6$.

\section{Sistema RSA}

El sistema criptográfico RSA, introducido en 1977 por Ron Rivest, Adi 
Shamir y Leonard Adleman \cite{rivest1978method}, marcó un antes y un 
después en la historia de la seguridad informática al ser la primera 
implementación práctica de un sistema de \textbf{criptografía asimétrica}.

Su importancia radica en que permite la comunicación segura a través de 
canales inseguros sin necesidad de que las partes hayan intercambiado 
previamente una clave secreta. La seguridad del protocolo no se basa en 
la oscuridad del algoritmo, sino en una \textbf{función trampa de un solo 
sentido} (\textit{one-way trapdoor function}): una operación matemática 
que es computacionalmente fácil de realizar en una dirección 
(multiplicación modular), pero extremadamente difícil de revertir 
(factorización de enteros grandes) sin una información especial llamada 
«trampa», es decir la clave privada.

\subsection{Idea general del sistema RSA}
El funcionamiento de RSA se puede describir mediante un protocolo de tres 
etapas: generación de llaves, encriptación y desencriptación. Supongamos 
un escenario donde Bob desea enviar un mensaje confidencial a Alice.

\subsubsection{Generación de la llave}
Para poder recibir mensajes, Alice debe construir su «candado» y su «llave» 
como represnetación se usan números. El procedimiento formal es el siguiente:

\begin{enumerate}
    \item \textbf{Selección de primos:} Se eligen dos números primos grandes 
    y distintos, $p$ y $q$. La seguridad del sistema depende críticamente 
    del tamaño de estos números (actualmente se recomiendan primos de 1024 
    bits o más).
    
    \item \textbf{Cálculo del módulo:} Se calcula el módulo público $N$:
    \begin{equation}
        N = p \times q
    \end{equation}
    Este valor $N$ define el tamaño del conjunto $\mathbb{Z}_N$ donde 
    ocurrirán las operaciones.
    
    \item \textbf{Cálculo del totiente:} Se calcula la función $\phi(N)$ de 
    Euler. Dado que $p$ y $q$ son primos, el número de coprimos menores que 
    $N$ es simplemente:
    \begin{equation}
        \phi(N) = (p-1)(q-1)
    \end{equation}
    
    \item \textbf{Exponente público ($e$):} Se elige un entero $e$ tal 
    que $1 < e < \phi(N)$ y que sea coprimo con $\phi(N)$, es decir:
    \begin{equation}
        \gcd(e, \phi(N)) = 1
    \end{equation}
    Habitualmente se elige $e=65537$ por eficiencia en la encriptación.
    
    \item \textbf{Exponente privado ($d$):} Se calcula el inverso 
    multiplicativo modular de $e$ módulo $\phi(N)$. Esto equivale a 
    resolver la congruencia lineal:
    \begin{equation}
        d \cdot e \equiv 1 \pmod{\phi(N)}
    \end{equation}
    Este valor $d$ se obtiene eficientemente utilizando el 
    \textit{Algoritmo de Euclides Extendido}.
\end{enumerate}

\subsubsection{Llave pública y privada}
El resultado del proceso anterior son dos pares de números:
\begin{itemize}
    \item \textbf{Llave Pública ($pk$):} El par $(N, e)$. Esta llave se 
    distribuye abiertamente y permite a cualquiera encriptar mensajes 
    dirigidos a Alice.
    \item \textbf{Llave Privada ($sk$):} El par $(N, d)$. Esta información 
    debe permanecer estrictamente secreta y permite a Alice desencriptar 
    los mensajes.
\end{itemize}

Es fundamental notar que los primos originales $p$ y $q$, así como el 
valor $\phi(N)$, deben ser destruidos o almacenados con la misma seguridad 
que $d$, ya que su conocimiento permitiría reconstruir la clave privada 
trivialmente.

\subsubsection{Encriptación}
Para enviar un mensaje a Alice, Bob primero debe convertir su mensaje de 
texto plano en una representación numérica $M$, tal que $0 \le M < N$. 
Utilizando la llave pública de Alice $(N, e)$, Bob calcula el texto 
cifrado $C$ mediante la operación de exponenciación modular:

\begin{equation}
    C \equiv M^e \pmod N
\end{equation}

Donde:
\begin{itemize}
    \item $M$: Mensaje original (texto plano).
    \item $e$: Exponente público.
    \item $N$: Módulo público.
    \item $C$: Mensaje encriptado (texto cifrado).
\end{itemize}

\subsubsection{Desencriptación}
Al recibir el texto cifrado $C$, Alice utiliza su exponente privado 
$d$ para recuperar el mensaje original $M$:

\begin{equation}
    M \equiv C^d \pmod N
\end{equation}

Esta operación es factible solo para quien conozca $d$. Un atacante que 
intercepta $C$ y conoce $(N, e)$ se enfrenta al problema del logaritmo 
discreto o a la factorización de $N$ para encontrar $d$.

\subsubsection{Prueba de corrección}
Para confiar en este sistema, debemos demostrar matemáticamente que la 
operación de desencriptación efectivamente recupera el mensaje original, 
es decir, que $(M^e)^d \equiv M \pmod N$.

Partimos de la ecuación de desencriptación:
\begin{equation}
    C^d \equiv (M^e)^d \equiv M^{ed} \pmod N
\end{equation}

Recordemos que $d$ fue elegido tal que $e \cdot d \equiv 1 \pmod{\phi(N)}$. 
Por definición de congruencia, esto implica que existe un entero $k$ tal 
que:
\begin{equation}
    ed = 1 + k\phi(N)
\end{equation}

Sustituyendo esto en la expresión del mensaje:
\begin{equation}
    M^{ed} = M^{1 + k\phi(N)} = M^1 \cdot (M^{\phi(N)})^k
\end{equation}

Aquí usamos el \textbf{Teorema de Euler}, que establece que para todo 
entero $a$ coprimo con $n$, se cumple que $a^{\phi(n)} \equiv 1 \pmod n$. 
Asumiendo $\gcd(M, N) = 1$ (lo cual es extremadamente probable para $N$ 
grandes):
\begin{equation}
    M \cdot (M^{\phi(N)})^k \equiv M \cdot (1)^k \equiv M \pmod N
\end{equation}

Esto demuestra que el proceso es reversible únicamente si se posee la 
inversa modular correcta.

\subsection{RSA y el problema de factorización}
La seguridad del algoritmo RSA no es absoluta, sino computacional. Se basa 
en la hipótesis de que existe una asimetría fundamental en la complejidad 
de ciertas operaciones aritméticas.

\subsubsection{¿Por qué es RSA seguro?}
La seguridad radica en la relación entre la clave pública y la privada. 
Para obtener la clave privada $d$ a partir de la pública $(N, e)$, un 
atacante necesita resolver:
\begin{equation}
    d \equiv e^{-1} \pmod{\phi(N)}
\end{equation}
Para realizar este cálculo, es indispensable conocer el valor de 
$\phi(N) = (p-1)(q-1)$. Dado que $N$ es público, obtener $\phi(N)$ es 
equivalente a encontrar los factores primos $p$ y $q$ de $N$.

Por lo tanto, romper RSA es equivalente a resolver el problema de la 
factorización de enteros: dado un número compuesto $N$, encontrar sus 
factores primos. Si $N$ es lo suficientemente grande, este problema se 
considera intratable para computadoras clásicas.

\subsubsection{Ejemplo con pequeños números}
Para ilustrar el funcionamiento aritmético, consideremos un ejemplo con 
números manejables manualmente, siguiendo los pasos del algoritmo:

\begin{enumerate}
    \item \textbf{Generación:}
    \begin{itemize}
        \item Elegimos $p = 61$ y $q = 53$.
        \item Calculamos el módulo $N = 61 \times 53 = 3233$.
        \item Calculamos el totiente $\phi(N) = (61-1)(53-1) = 3120$.
        \item Elegimos un exponente público $e = 17$ (verificamos que 
        $\gcd(17, 3120)=1$).
        \item Calculamos el exponente privado $d$ tal que $17d \equiv 1 
        \pmod{3120}$. Usando el algoritmo de Euclides, obtenemos $d = 2753$.
    \end{itemize}
    \item \textbf{Operación:}
    \begin{itemize}
        \item Supongamos un mensaje $M = 65$.
        \item \textbf{Encriptación:} $C = 65^{17} \bmod 3233 = 2790$.
        \item \textbf{Desencriptación:} $M' = 2790^{2753} \bmod 3233 = 65$.
    \end{itemize}
\end{enumerate}
El mensaje recuperado coincide con el original, validando el proceso.

\subsection{Complejidad clásica}
La confianza en RSA se basa en que no existe ningún algoritmo clásico 
eficiente que pueda factorizar $N$ en tiempo polinómico. El mejor algoritmo 
clásico conocido hasta la fecha es la \textbf{Criba General del Cuerpo de 
Números} (General Number Field Sieve, GNFS).

La complejidad temporal de este algoritmo para factorizar un entero $N$ 
es sub-exponencial:
\begin{equation}
    O\left(\exp\left(\left(\frac{64}{9}\right)^{1/3} (\ln N)^{1/3} (\ln 
    \ln N)^{2/3}\right)\right)
\end{equation}

Esta complejidad implica que el tiempo requerido crece drásticamente con 
el tamaño de la clave. Se estima que factorizar una clave RSA de 2048 bits 
con la tecnología clásica actual tomaría miles de millones de años, 
haciendo que el sistema sea seguro en la práctica, previo a la existencia 
de las computadoras cuánticas.

\subsection{Conceptos críticos de seguridad: Hash y Entropía}

Para que la implementación de RSA sea segura y eficiente en el mundo real, 
no basta con la aritmética modular. Se requiere garantizar la integridad de 
los mensajes y la imprevisibilidad de las claves. Para ello, se aplican 
dos conceptos fundamentales: las funciones de resumen (Hash) y la entropía 
de la información.

\subsubsection{Funciones Hash criptográficas}
Una función hash $H(\cdot)$ es un algoritmo matemático que transforma una 
entrada de datos de tamaño arbitrario (como un archivo, un correo 
electrónico o una imagen) en una cadena de bits de tamaño fijo, conocida 
como \textit{digest} o «huella digital».

Formalmente, $H: \{0,1\}^* \to \{0,1\}^n$. Para ser útil en criptografía, 
una función hash debe satisfacer propiedades estrictas \cite{katz2020introduction}:

\begin{enumerate}
    \item \textbf{Determinismo:} Para una misma entrada $x$, siempre debe 
    producirse la misma salida $H(x)$.
    \item \textbf{Resistencia a la preimagen (Unidireccionalidad):} Dado un 
    hash $y$, debe ser computacionalmente inviable encontrar un mensaje $x$ 
    tal que $H(x) = y$.
    \item \textbf{Efecto avalancha:} Un cambio minúsculo en la entrada 
    (por ejemplo, cambiar un solo bit) debe producir un cambio drástico e 
    impredecible en la salida.
    \item \textbf{Resistencia a colisiones:} Debe ser computacionalmente 
    inviable encontrar dos entradas diferentes $x_1 \neq x_2$ tales que 
    $H(x_1) = H(x_2)$.
\end{enumerate}



\textbf{Relación con RSA:}
El algoritmo RSA es computacionalmente costoso, especialmente para mensajes 
largos. En la práctica, para realizar una \textbf{Firma Digital}, Alice no 
encripta todo el documento con su clave privada. En su lugar:
\begin{enumerate}
    \item Alice calcula el hash del documento: $h = H(M)$.
    \item Alice encripta solo el hash (que es pequeño y fijo, ej. 256 bits) 
    con su clave privada: $\text{Firma} = h^d \bmod N$.
    \item Esto garantiza eficiencia y que el documento no ha sido alterado 
    (integridad), ya que si alguien modifica $M$, el hash cambiará y la 
    firma no coincidirá.
\end{enumerate}

\subsubsection{Entropía e Incertidumbre}
La seguridad de RSA depende de que $p$ y $q$ sean secretos. Si un atacante 
puede predecir o adivinar los números primos que elegimos, el sistema 
colapsa. Aquí entra el concepto de \textbf{Entropía}.

Definida por Claude Shannon en 1948 \cite{shannon1948}, la entropía es una 
medida de la incertidumbre asociada a una variable aleatoria. 
Matemáticamente, la entropía $H(X)$ de una variable discreta $X$ con 
posibles resultados $x_1, \dots, x_n$ y probabilidades $P(x_i)$ es:

\begin{equation}
    H(X) = - \sum_{i} P(x_i) \log_2 P(x_i)
\end{equation}

En el contexto de la generación de claves:
\begin{itemize}
    \item \textbf{Baja Entropía:} Si utilizamos un generador de números 
    pseudoaleatorios malo (por ejemplo, basado en la hora del reloj del 
    sistema), el espacio de búsqueda se reduce drásticamente. Un atacante 
    no necesita factorizar $N$; solo necesita probar las semillas probables 
    del generador.
    \item \textbf{Alta Entropía:} Para generar claves RSA seguras, se 
    requiere una fuente de \textbf{aleatoriedad criptográficamente segura}, 
    que maximice la entropía. Esto asegura que la probabilidad 
    de elegir cualquier par específico de primos sea uniformemente baja, 
    haciendo inviable la adivinanza.
\end{itemize}



Sin una fuente de alta entropía, las matemáticas de RSA son sólidas, pero 
la implementación es vulnerable.

\subsection{Aplicaciones}
Debido a sus propiedades de seguridad y autenticación, RSA se ha convertido 
en el estándar para la seguridad digital:

\begin{itemize}
    \item \textbf{Navegación Web Segura (HTTPS):} RSA se utiliza en el 
    protocolo TLS/SSL para intercambiar de forma segura las claves 
    simétricas que cifrarán la sesión de navegación.
    \item \textbf{Firmas Digitales:} RSA permite la autenticación y el no 
    repudio. Alice puede firmar un documento encriptando un hash del 
    mismo con su clave \textbf{privada}. Cualquier persona puede verificar 
    la firma desencriptando con la clave \textbf{pública} de Alice. Si el 
    hash coincide, se garantiza matemáticamente que solo Alice pudo haberlo 
    firmado y que el documento no fue alterado.
    \item \textbf{Acceso Remoto (SSH):} Utilizado para autenticar servidores 
    y usuarios en administración de sistemas.
\end{itemize}

\section{Implementación práctica del protocolo RSA}

Con el objetivo de validar los conceptos teóricos del sistema RSA y comprender qué 
es exactamente lo que el algoritmo de Shor debe «romper», se desarrolló un script 
en Python que simula todo el ciclo de vida del criptosistema: desde la generación 
de llaves hasta la transmisión de mensajes cifrados.

El código completo y la documentación técnica se encuentran disponibles en el 
repositorio del proyecto: \url{https://github.com/Jorge-1501/Practicas}.

\subsection{Metodología de la implementación}

A diferencia de las implementaciones industriales que operan sobre enormes bloques 
de bits, este simulador pedagógico opera bajo un esquema cifrado carácter 
a carácter utilizando la codificación UTF-8. Esto permite visualizar cómo las 
propiedades aritméticas actúan sobre cada unidad de información individualmente.

La arquitectura del script se divide en tres módulos funcionales:

\begin{enumerate}
    \item \textbf{Tratamiento de datos:} Conversión de cadenas de texto a listas de 
    enteros y viceversa.
    \item \textbf{Núcleo matemático:} Funciones para pruebas de primalidad, 
    cálculo del totiente, inverso modular y generación de llaves $(e, d)$.
    \item \textbf{Interfaz de cifrado:} El bucle principal que ejecuta las 
    transformaciones $C = M^e \pmod N$ y $M = C^d \pmod N$.
\end{enumerate}

\subsection{Conversión y Espacio de Mensajes}
El primer desafío es transformar el lenguaje natural. 
Utilizando la codificación estándar UTF-8, cada carácter se convierte en un entero,
o una secuencia de enteros.

\begin{lstlisting}[language=Python, caption=Conversión de texto a valores numéricos]
def string_to_numbers(text):
    """Convierte una cadena en una lista de valores enteros UTF-8."""
    return list(text.encode("utf-8"))

def numbers_to_string(numbers):
    """Reconstruye la cadena original desde la lista de enteros."""
    return bytes(numbers).decode("utf-8")
\end{lstlisting}

Esta función es importante porque define nuestro \textbf{espacio de mensaje} $M$. 
En codificación ASCII/UTF-8 estándar, los caracteres imprimibles comunes tienen 
valores entre 32 y 126 (por ejemplo, 'A' = 65, 'a' = 97). Esto impone una 
restricción física sobre el tamaño mínimo de nuestro módulo $N$, como se discutirá 
en las limitaciones.

\subsection{Generación de llaves y Cifrado}
El núcleo del algoritmo implementa la selección aleatoria de exponentes. Para 
garantizar la seguridad del sistema, se busca un exponente público $e$ que sea 
coprimo con $\phi(N)$ y un exponente privado $d$ que sea su inverso modular.

La implementación de las funciones de encriptación y desencriptación utiliza la 
función nativa \texttt{pow(base, exp, mod)} de Python, la cual implementa la 
\textit{exponenciación modular rápida}, esencial para manejar la complejidad 
computacional.

\begin{lstlisting}[language=Python, caption=Funciones de cifrado y descifrado]
def encrypt(text, e, N):
    """Aplica la transformación C = M^e mod N a cada carácter."""
    ascii_values = string_to_numbers(text)
    encrypted_values = [pow(char, e, N) for char in ascii_values]
    return encrypted_values

def decrypt(encrypted_values, d, N):
    """Recupera el mensaje aplicando M = C^d mod N."""
    decrypted_values = [pow(char, d, N) for char in encrypted_values]
    return numbers_to_string(decrypted_values)
\end{lstlisting}

\subsection{Resultados y Análisis de Limitaciones}
Durante las pruebas experimentales del script, se observó un comportamiento crítico 
relacionado con la magnitud de los números primos seleccionados.

El sistema funciona correctamente para pares de primos como $p=13, q=17$, donde 
$N=221$, o superiores. Sin embargo, para primos pequeños, menores a 11, 
el proceso de desencriptación falla, recuperando caracteres erróneos o lanzando 
errores de decodificación.

\subsubsection{La restricción del tamaño del módulo ($M < N$)}
Este error no es un fallo del código, sino consecuencia de una propiedad de la 
aritmética modular. Para que la información se conserve intacta en el anillo 
$\mathbb{Z}_N$, el valor numérico del mensaje $M$ debe ser estrictamente menor 
que el módulo $N$.

\begin{equation}
    M < N
\end{equation}

Si tenemos un mensaje $M$ (por ejemplo, la letra 'z' con valor ASCII 122) y 
elegimos primos pequeños como $p=5$ y $q=11$, nuestro módulo es $N=55$. Al 
intentar encriptar:
\begin{equation}
    C \equiv 122^e \pmod{55}
\end{equation}
La operación módulo reduce el valor 122 a un número menor que 55 
\textit{antes} de que siquiera empiece la encriptación real 
(pérdida de información por desbordamiento del módulo). Matemáticamente, 
$122 \equiv 12 \pmod{55}$. Al desencriptar, recuperaremos 12, no 122.

Por lo tanto, para cifrar texto ASCII estándar, valores hasta 127, el producto de 
los primos debe cumplir $N > 127$.
\begin{itemize}
    \item Con $p=5, q=11 \rightarrow N=55$ (Falla con letras).
    \item Con $p=11, q=13 \rightarrow N=143$ (Funciona para ASCII estándar).
    \item Con $p=13, q=17 \rightarrow N=221$ (Funciona para ASCII extendido).
\end{itemize}

Esta validación experimental confirma que la seguridad y funcionalidad de RSA 
dependen intrínsecamente de la magnitud de los números primos utilizados, 
justificando la necesidad de primos bastante grandes (1024+ bits) en aplicaciones 
reales, no solo por seguridad ante la factorización, sino para poder encapsular 
mensajes significativos.


\section{Herramientas algorítmicas cuánticas}

Para comprender la mecánica interna del algoritmo de Shor, es necesario dominar dos 
subrutinas fundamentales que explotan el paralelismo cuántico: la Transformada 
Cuántica de Fourier y la Estimación de Fase.

\subsection{Transformada Cuántica de Fourier (QFT)}
La QFT es el análogo cuántico de la Transformada Discreta de Fourier (DFT) clásica. 
Mientras que la DFT procesa un vector de amplitudes complejas, la QFT opera 
directamente sobre el vector de estado cuántico, realizando un cambio de base que 
revela la estructura periódica oculta en los datos.

\subsubsection{Definición Matemática}
Consideremos un estado cuántico en un espacio de Hilbert de dimensión $N = 2^n$. La 
QFT se define como un operador unitario que transforma un estado de la base 
computacional $|j\rangle$ en una superposición uniforme de todos los estados de la 
base, modulados por fases complejas \cite{nielsen_chuang}:

\begin{equation}
    \text{QFT} : |j\rangle \mapsto \frac{1}{\sqrt{N}} \sum_{k=0}^{N-1} 
    e^{2\pi i\, \frac{jk}{N}} |k\rangle
    \label{eq:QFT_def}
\end{equation}

\subsubsection{Interpretación Física}
Desde una perspectiva física, la QFT transforma la información de la base 
computacional (posición) a la base de Fourier (momento o frecuencia).
\begin{itemize}
    \item \textbf{Interferencia:} Los factores de fase $e^{2\pi i\, \frac{jk}{N}}$ 
    son cruciales. Al sumar múltiples estados transformados, estos factores provocan 
    interferencia constructiva en las frecuencias que corresponden a la periodicidad 
    del sistema y destructiva en las demás.
    \item \textbf{Unitaridad:} Como toda compuerta cuántica válida, la 
    transformación es reversible ($\text{QFT}^\dagger \text{QFT} = I$).
\end{itemize}

En la fig \ref{fig:Code_QFT} se presenta la implementación de la QFT en código 
utilizando la biblioteca Qiskit en Python, donde se observa la estructura 
recursiva de compuertas Hadamard y rotaciones controladas de fase.

\begin{figure}[h!]
    \centering
    \includegraphics[width=0.7\linewidth]{imagenes/QFT.png}
    \caption{Implementación en Python de la Transformada Cuántica de Fourier.}
    \label{fig:Code_QFT}
\end{figure}

\subsection{Estimación de Fase Cuántica (QPE)}
La Estimación de Fase es quizás la subrutina muy importante en la computación 
cuántica algorítmica, generalizada por Kitaev \cite{kitaev1995quantum}. 
Muchos problemas algebraicos complejos pueden reducirse a encontrar el 
eigenvalor de un operador unitario.

\subsubsection{Definición del Problema}
Dado un operador unitario $U$ y uno de sus eigenestados $|u\rangle$ tal que:
\begin{equation}
    U|u\rangle = e^{2\pi i\phi} |u\rangle
\end{equation}
El objetivo es estimar el valor de la fase $\phi \in [0, 1)$.

\subsubsection{Algoritmo y Circuito}
El procedimiento utiliza dos registros cuánticos: un registro de control de $t$ 
qubits (inicializados en $|0\rangle$) para almacenar la aproximación de la fase, 
y un registro objetivo que almacena el eigenestado $|u\rangle$.

\textbf{Procedimiento:}
\begin{enumerate}
    \item \textbf{Superposición:} Se aplican compuertas Hadamard al registro de 
    control para crear una superposición equiprobable:
    \begin{equation}
        |\psi_1\rangle = \frac{1}{\sqrt{2^t}} \sum_{j=0}^{2^t -1} |j\rangle \otimes 
        |u\rangle
    \end{equation}
    
    \item \textbf{Aplicación del Oráculo (Phase Kickback):} Se aplican operaciones 
    $U$ controladas. El término $U^j$ actuando sobre $|u\rangle$ inyecta la fase 
    $e^{2\pi i j \phi}$ en el registro de control (retroceso de fase):
    \begin{equation}
        |\psi_2\rangle = \frac{1}{\sqrt{2^t}} \sum_{j=0}^{2^t -1} e^{2\pi i j \phi} 
        |j\rangle \otimes |u\rangle
    \end{equation}
    
    \item \textbf{QFT Inversa:} El estado resultante en el primer registro es 
    exactamente la QFT del estado que codifica la fase $\phi$. Por tanto, 
    aplicamos la $\text{QFT}^{-1}$ para recuperar el valor binario de la fase:
    \begin{equation}
        |\psi_3\rangle = |\tilde{\phi}\rangle \otimes |u\rangle
    \end{equation}
    
    \item \textbf{Medición:} Al medir el primer registro, obtenemos una 
    aproximación de $\phi$ con una precisión de $t$ bits.
\end{enumerate}

\begin{figure}[h!]
    \centering
    \begin{quantikz}[column sep=0.5cm]
      \lstick{\( \ket{0}^{\otimes t} \)} & \gate{H^{\otimes t}} & \ctrl{1} & 
      \gate{\text{QFT}^{-1}} & \meter{} \\
      \lstick{\( \ket{u} \)}             & \qw                  & \gate{U^{2^j}} & 
      \qw                    & \qw
    \end{quantikz}
    \caption{Esquema del circuito para la Estimación de Fase Cuántica.}
    \label{fig:circuit_QPE}
\end{figure}

\begin{figure}[h!]
    \centering
    \includegraphics[width=0.7\linewidth]{imagenes/Estimador_phase.png}
    \caption{Ejemplo de código para la rutina de estimación de fase.}
    \label{fig:Code_PE}
\end{figure}

\section{Algoritmo de Shor}

El algoritmo de Shor es, sin duda, el algoritmo cuántico más célebre y con mayor 
impacto potencial en la ciberseguridad global. Mientras que los algoritmos clásicos 
de factorización (como el GNFS) operan en tiempo sub-exponencial, Shor demostró 
que una computadora cuántica puede realizar esta tarea en tiempo polinómico 
$O((\log N)^3)$.

\subsection{Definición y motivaciones}
El problema central es factorizar un número entero compuesto $N = p \times q$.
\begin{itemize}
    \item \textbf{Problema:} Dado un entero compuesto $N$, encontrar sus factores 
    primos no triviales.
    \item \textbf{Motivación:} La seguridad del sistema RSA descansa enteramente 
    en la intratabilidad computacional de este problema para computadoras clásicas. 
    Si la factorización se vuelve eficiente, la infraestructura de clave pública 
    actual colapsa.
    \item \textbf{Idea Clave:} El algoritmo no ataca la factorización directamente 
    mediante división. En su lugar, utiliza la computación cuántica para resolver 
    un problema relacionado: la \textbf{búsqueda de periodo} (Order Finding).
\end{itemize}

\subsection{Reducción del problema de factorización a la búsqueda de periodo}
La conexión entre la teoría de números y la física cuántica se establece mediante 
el siguiente teorema, que permite convertir el problema de factorización en un 
problema de periodicidad.

\begin{teorema}[Factorización vía Periodicidad]
Sea $N$ un número compuesto y sea $a$ un entero tal que $\gcd(a, N) = 1$. Definimos 
la función modular:
\begin{equation}
    f(x) = a^x \pmod N
\end{equation}
Si podemos encontrar el periodo $r$ de esta función (el menor entero tal que $a^r 
\equiv 1 \pmod N$), y si este periodo $r$ cumple con dos condiciones:
\begin{enumerate}
    \item $r$ es un número par.
    \item $a^{r/2} \not\equiv -1 \pmod N$.
\end{enumerate}
Entonces, los factores primos de $N$ pueden calcularse eficientemente mediante el 
Máximo Común Divisor:
\begin{equation}
    p = \gcd(a^{r/2} - 1, N) \quad \text{y} \quad q = \gcd(a^{r/2} + 1, N)
\end{equation}
\end{teorema}

La computación clásica es ineficiente encontrando $r$ porque requiere evaluar la 
función exponencialmente muchas veces. La computación cuántica, mediante la 
Estimación de Fase (QPE), puede encontrar $r$ evaluando la función en superposición 
una sola vez.

\subsection{Algoritmo de Shor: Procedimiento}
El procedimiento general del algoritmo es híbrido, alternando pasos clásicos y 
cuánticos:

\begin{enumerate}
    \item \textbf{(Clásico) Selección:} Elegir un número aleatorio $a < N$.
    \item \textbf{(Clásico) Verificación:} Calcular $\gcd(a, N)$. Si es mayor que 1, 
    hemos encontrado un factor por suerte. Si es 1, continuamos.
    \item \textbf{(Cuántica) Búsqueda de Periodo:} Utilizar el circuito de 
    estimación de fase sobre la función $f(x) = a^x \pmod N$ para encontrar el 
    periodo $r$.
    \item \textbf{(Clásico) Extracción de Factores:} Si $r$ es par y cumple las 
    condiciones del teorema, calcular $\gcd(a^{r/2} \pm 1, N)$. Si no, repetir con 
    un nuevo $a$.
\end{enumerate}

\subsubsection{Construcción explícita del Oráculo para N=15}

Uno de los mayores desafíos en la implementación física de algoritmos cuánticos es 
la construcción de funciones aritméticas (como $f(x) = a^x \pmod N$) en compuertas 
lógicas reversibles.

Para el caso general, se requiere construir sumadores y multiplicadores modulares 
cuánticos, lo cual demanda miles de compuertas. Sin embargo, para casos pequeños 
como $N=15$, es posible aplicar una técnica conocida como \textbf{«Compilación de 
Shor»}, utilizada en la histórica primera demostración experimental del algoritmo 
por Vandersypen et al. en IBM \cite{vandersypen2001experimental}.

\textbf{Análisis de la Permutación:}
El operador $U_a$ actúa sobre el estado $|y\rangle$ transformándolo en $|ay \pmod 
N\rangle$. Analicemos el caso específico de nuestro experimento con $a=7$ y $N=15$. 
Dado que $\gcd(7, 15)=1$, esta operación es una permutación biyectiva de los 
elementos de $\mathbb{Z}_{15}$.

Si observamos la acción de multiplicar por 7 repetidamente sobre los estados base:
\begin{equation}
    1 \xrightarrow{\times 7} 7 \xrightarrow{\times 7} 49 \equiv 4 \xrightarrow{
        \times 7} 28 \equiv 13 \xrightarrow{\times 7} 1
\end{equation}

Esto revela que la operación $U_{7}$ realiza la permutación cíclica $(1 \to 7 
\to 4 \to 13 \to 1)$. En representación binaria de 4 qubits:
\begin{align*}
    |0001\rangle &\to |0111\rangle \\
    |0111\rangle &\to |0100\rangle \\
    |0100\rangle &\to |1101\rangle \\
    |1101\rangle &\to |0001\rangle
\end{align*}

\textbf{Implementación optimizada:}
En lugar de implementar un circuito aritmético completo que calcule la 
multiplicación, podemos construir un circuito lógico simplificado que simplemente 
realice estos intercambios de bits específicos.

Basándonos en la simplificación propuesta por Vandersypen, esta permutación puede 
descomponerse en una serie mínima de compuertas SWAP. El código implementado en 
la función \texttt{c\_amod15} de nuestro script realiza la operación mediante 
tres transposiciones:
\begin{enumerate}
    \item $SWAP(0, 1)$
    \item $SWAP(1, 2)$
    \item $SWAP(2, 3)$
\end{enumerate}

Esta secuencia tiene el efecto neto de rotar y permutar los bits exactamente 
como lo requiere la multiplicación modular por 7, reduciendo la profundidad 
del circuito a $O(1)$.

\textbf{Justificación y Limitaciones:}
Esta técnica es válida experimentalmente pedagógica para demostrar la fase 
de interferencia de Shor sin requerir recursos de hardware inmensos. Sin 
embargo, es una optimización no escalable: para factorizar $N=323$, no podemos 
pre-calcular la tabla de permutaciones (pues implicaría conocer el periodo a 
priori), obligándonos a usar el circuito aritmético general, lo cual, como 
se vio en los resultados, disparó la complejidad computacional de la simulación.

\subsubsection{Circuito cuántico y Post-procesamiento}
El núcleo del algoritmo es el circuito de búsqueda de orden que utiliza dos 
registros: uno de conteo (inicializado en superposición) y otro objetivo.

\begin{center}
    \begin{quantikz}[column sep=0.8cm]
      \lstick{\(\ket{0}^{\otimes n}\)} & \gate{H^{\otimes n}} & \ctrl{1} & 
      \gate{\text{QFT}^{-1}} & \meter{} \\
      \lstick{\(\ket{1}\)}             & \qw                  & \gate{U_a} & \qw                    & \qw
    \end{quantikz}
\end{center}

Donde $U_a$ implementa la exponenciación modular controlada $|x\rangle 
|y\rangle \to |x\rangle |a^x y \pmod N\rangle$.

La salida del circuito es una aproximación de la fase $\phi \approx s/r$. 
Para recuperar el periodo exacto $r$ a partir de esta medición ruidosa, se 
utiliza el algoritmo clásico de \textbf{Fracciones Continuas}, el cual encuentra 
la fracción $s/r$ que mejor aproxima el valor medido con un denominador menor que 
$N$.

\subsubsection{El método de Fracciones Continuas}
La medición del circuito cuántico nos entrega un valor entero $y$. Al dividir 
este valor por el tamaño del espacio de Hilbert $2^t$, obtenemos una aproximación 
racional de la fase:
\begin{equation}
    \frac{y}{2^t} \approx \frac{s}{r}
\end{equation}
Donde $s$ es un entero desconocido y $r$ es el periodo buscado. El problema es 
que $\frac{y}{2^t}$ casi nunca es exactamente igual a $\frac{s}{r}$ debido a la 
imprecisión binaria.

Para recuperar los valores exactos de $s$ y $r$ a partir de esta aproximación 
ruidosa, utilizamos la expansión en \textbf{Fracciones Continuas}. Cualquier 
número real $x$ puede representarse como:
\begin{equation}
    x = a_0 + \frac{1}{a_1 + \frac{1}{a_2 + \frac{1}{\dots}}} = [a_0; a_1, a_2, 
    \dots]
\end{equation}

Si truncamos esta expansión en diferentes pasos, obtenemos una secuencia de 
fracciones $\frac{p_n}{q_n}$ llamadas \textbf{convergentes}. Un teorema 
fundamental de la teoría de números garantiza que si nuestra aproximación cumple:
\begin{equation}
    \left| \frac{y}{2^t} - \frac{s}{r} \right| < \frac{1}{2r^2}
\end{equation}
Entonces, la fracción simplificada $\frac{s}{r}$ aparecerá necesariamente como 
uno de los convergentes de la expansión en fracciones continuas de $\frac{y}{2^t}$.

\textbf{Procedimiento en el algoritmo:}
\begin{enumerate}
    \item Se calcula la expansión en fracciones continuas del valor medido $\frac{y}{2^t}$.
    \item Se calculan los convergentes (fracciones simplificadas) de dicha expansión.
    \item El denominador de cada convergente es un \textbf{candidato a periodo} $r'$.
    \item Se verifica clásicamente si $a^{r'} \equiv 1 \pmod N$. El primer denominador 
    que cumpla esta condición es el periodo verdadero $r$.
\end{enumerate}

\textbf{Ejemplo de aplicación:}
Supongamos que al medir el registro de lectura para el caso $N=15$ y $a=7$, 
obtenemos un valor binario que, normalizado, corresponde a la fase decimal 
$\phi \approx 0.75$. Queremos recuperar la fracción original $s/r$.

Aplicamos la expansión en fracciones continuas a $0.75$:
\begin{enumerate}
    \item Parte entera de $0.75$ es $0$. Resto: $0.75$. Invertimos: $1/0.75 = 1.333\dots$
    \item Parte entera de $1.333\dots$ es $1$. Resto: $0.333\dots$. Invertimos: $1/0.333\dots = 3$.
    \item Parte entera de $3$ es $3$. Resto $0$. Fin.
\end{enumerate}

La expansión es:
\begin{equation}
    0.75 = 0 + \frac{1}{1 + \frac{1}{3}} = [0; 1, 3]
\end{equation}

Calculamos los \textbf{convergentes} (candidatos) truncando la expansión:
\begin{itemize}
    \item \textbf{Primer convergente} $[0; 1]$:
    \begin{equation}
        0 + \frac{1}{1} = \frac{1}{1} \implies \text{Candidato } r=1.
    \end{equation}
    Verificamos: $7^1 \equiv 7 \not\equiv 1 \pmod{15}$. (Incorrecto).
    
    \item \textbf{Segundo convergente} $[0; 1, 3]$:
    \begin{equation}
        0 + \frac{1}{1 + \frac{1}{3}} = \frac{1}{\frac{4}{3}} = \frac{3}{4} 
        \implies \text{Candidato } r=4.
    \end{equation}
    Verificamos: $7^4 = 2401 \equiv 1 \pmod{15}$. \textbf{(Correcto)}.
\end{itemize}

De esta forma, las fracciones continuas nos revelan que el denominador (periodo) 
oculto detrás de la fase $0.75$ es $r=4$.

\section{Simulación y resultados experimentales}

Para validar la teoría, se implementó el algoritmo de Shor utilizando el framework 
Qiskit y el simulador \texttt{aer\_simulator}. El código completo se encuentra en 
el archivo \texttt{Shor\_completo.py} del repositorio.

\subsection{Implementación del Oráculo para N=15}
El mayor desafío técnico en la implementación de Shor es la construcción del 
oráculo $U_a$. Para el caso pedagógico de $N=15$, es posible optimizar el circuito 
manualmente utilizando compuertas SWAP, lo que reduce drásticamente la profundidad 
del circuito.

\begin{lstlisting}[language=Python, caption=Implementación optimizada del oráculo 
    para N=15]
def c_amod15(a, potencia):
    '''Compuerta controlada para a^x mod 15'''
    if a not in [2,4,7,8,11,13]:
        raise ValueError("'a' must be 2, 4, 7, 8, 11 o 13")
    U = QuantumCircuit(4)
    for _iteration in range(potencia):
        if a in [2, 13]:
            U.swap(2,3)
            U.swap(1,2)
            U.swap(0,1)
        if a in [7, 8]:
            U.swap(0,1)          
            U.swap(1,2)
            U.swap(2,3)
        # ... (otros casos omitidos por brevedad)
    U = U.to_gate()
    c_U = U.control()
    return c_U
\end{lstlisting}

\subsection{Resultados: Factorización de 15}
Se realizó una simulación completa factorizando $N=15$ con $a=7$.

\begin{figure}[H]
    \centering
    \begin{minipage}{0.48\textwidth}
        \centering
        \includegraphics[width=\linewidth]{imagenes/Shor.png} 
        \caption{Fragmento del código de ejecución.}
        \label{fig:Shor_code}
    \end{minipage}\hfill
    \begin{minipage}{0.48\textwidth}
        \centering
        \includegraphics[width=\linewidth]{imagenes/Shor_circuit.png} 
        \caption{Circuito generado por Qiskit.}
        \label{fig:Shor_circuit}
    \end{minipage}
\end{figure}

El simulador produjo un histograma de mediciones donde los picos corresponden 
a las fases que interfieren constructivamente.

\begin{figure}[H]
    \centering
    \includegraphics[width=0.6\linewidth]{imagenes/output.png} 
    \caption{Histograma de resultados. Los picos indican las fases correctas.}
    \label{fig:Shor_output}
\end{figure}

Tras aplicar fracciones continuas a la salida medida, el sistema sugirió una 
fase correspondiente a $s/r$ con denominador $r=4$.

\textbf{Verificación y Factores:}
\begin{equation}
    7^4 = 2401 \equiv 1 \pmod{15} \implies r=4 \text{ es correcto.}
\end{equation}
Dado que $r=4$ es par, calculamos:
\begin{align}
    \gcd(7^{4/2} - 1, 15) &= \gcd(48, 15) = 3 \\
    \gcd(7^{4/2} + 1, 15) &= \gcd(50, 15) = 5
\end{align}
El algoritmo factorizó exitosamente $15 = 3 \times 5$.

\subsection{Análisis de Escalabilidad y el Caso N=323}
Como parte del proyecto RSA, se intentó utilizar este mismo script para factorizar 
$N=323$ ($13 \times 17$). Sin embargo, el experimento reveló una limitación crítica 
de la simulación clásica:

\begin{enumerate}
    \item \textbf{Complejidad del Oráculo:} A diferencia de $N=15$, no existen 
    simplificaciones triviales de SWAP para $N=323$. Se requiere un circuito 
    aritmético modular generalizado.
    \item \textbf{Explosión de Recursos:} Para $N=323$, se requieren 27 qubits 
    ($n=9$ estado, $2n=18$ conteo). La simulación de un vector de estado de 
    $2^{27}$ amplitudes, sumado a la profundidad del circuito aritmético, excedió 
    las capacidades de procesamiento convencionales.
\end{enumerate}

Este resultado valida la tesis central de la computación cuántica: simular sistemas 
cuánticos grandes en ordenadores clásicos es intratable; se requiere hardware 
cuántico real para escalar.

\section{Conclusiones}

El desarrollo de este proyecto ha permitido transitar desde los fundamentos 
matemáticos de la criptografía clásica hasta la frontera de la computación 
cuántica.

\begin{itemize}
    \item \textbf{Vulnerabilidad de RSA:} Se ha demostrado teóricamente y mediante 
    simulación (para $N=15$) que la seguridad de RSA es vulnerable ante el 
    algoritmo de Shor.
    \item \textbf{Supremacía Algorítmica:} La reducción de la complejidad de 
    sub-exponencial a polinómica representa una ventaja cualitativa que ninguna 
    mejora en el hardware clásico podrá igualar.
    \item \textbf{Desafíos de Implementación:} La dificultad encontrada al escalar 
    la simulación a $N=323$ subraya los desafíos técnicos: la construcción de 
    oráculos eficientes y la necesidad de coherencia en un gran número de qubits 
    son las barreras principales para la adopción práctica.
    \item \textbf{Necesidad de Criptografía Post-Cuántica:} Dado que la física 
    permite romper RSA, es imperativo para la seguridad global la transición 
    hacia algoritmos resistentes a ataques cuánticos.
\end{itemize}


\part{Aplicaciones en física de partículas}
%--- MAYO-JUNIO ---
\chapter{Aplicaciones de técnicas avanzadas de Machine Learning y Computación Cuántica en Física de Partículas}

Este capítulo documenta la exploración técnica realizada en el contexto de las 
pruebas de calificación para el programa \textit{Google Summer of Code} (GSoC) 2025. 
El objetivo de estas tareas fue demostrar la competencia en el manejo de circuitos 
cuánticos, el procesamiento de datos físicos mediante grafos y la implementación 
de arquitecturas neuronales de vanguardia.

\section{Introducción a QML-HEP}

La organización \textit{Machine Learning for Science} (ML4SCI) es una iniciativa 
de código abierto que reúne a investigadores de universidades y laboratorios como 
el CERN, con el objetivo de aplicar técnicas modernas de inteligencia artificial a 
problemas complejos en ciencia fundamental \cite{ml4sci}.

Dentro de esta organización, el grupo de trabajo \textit{Quantum Machine Learning 
for High Energy Physics} (QML-HEP) se enfoca específicamente en investigar si la 
computación cuántica puede ofrecer ventajas computacionales —ya sea en velocidad 
o en expresividad— para el análisis de datos provenientes del Gran Colisionador 
de Hadrones (LHC). Las tareas presentadas a continuación fueron diseñadas para 
evaluar la viabilidad técnica de integrar algoritmos cuánticos en flujos de trabajo 
de física de altas energías.

\section{Cumplimiento de la Tarea III: Open Task}

Como parte de los requisitos obligatorios de la evaluación (Task III: Open Task), 
se solicita una discusión original sobre un algoritmo o paradigma de computación 
cuántica que demuestre el entendimiento personal del candidato.

En el contexto de este trabajo, dicho requisito se satisface mediante el análisis 
exhaustivo del \textbf{Algoritmo de Shor} y su implementación práctica, el cual ha 
sido desarrollado en profundidad en el \textbf{Capítulo \ref{cap:shor}} 
(ver página \pageref{cap:shor}). En dicha sección, se discuten no solo los 
fundamentos matemáticos de la transformada cuántica de Fourier y la búsqueda de 
periodo, sino también las implicaciones criptográficas y las limitaciones actuales 
del hardware para su ejecución, cumpliendo así con el objetivo de demostrar 
familiaridad y crítica técnica sobre algoritmos cuánticos fundamentales.

\section{Herramientas de simulación y circuitos cuánticos}

El primer desafío consistió en la implementación de operaciones cuánticas 
fundamentales utilizando dos frameworks prominentes en la industria: 
\textit{Cirq} (desarrollado por Google) y \textit{PennyLane} (desarrollado por 
Xanadu). El dominio de estas herramientas es prerrequisito para diseñar los 
circuitos variacionales que se discutirán más adelante.

\subsection{Fundamentos: Compuertas Hadamard y Controladas}
Los circuitos implementados se basan en un conjunto universal de compuertas que 
manipulan el estado de los qubits en la esfera de Bloch.

\textbf{Compuerta Hadamard ($H$):}
Esta operación es esencial para crear superposición cuántica. Transforma los 
estados de la base computacional $\{|0\rangle, |1\rangle\}$ en estados de 
superposición equiprobable $|+\rangle$ y $|-\rangle$. Matricialmente se define 
como:
\begin{equation}
    H = \frac{1}{\sqrt{2}} \begin{pmatrix} 1 & 1 \\ 1 & -1 \end{pmatrix}
\end{equation}

\textbf{Compuerta CNOT ($CX$):}
La Compuerta Controlada «No» opera sobre dos qubits: un control y un objetivo. 
Invierte el estado del qubit objetivo si y solo si el qubit de control se 
encuentra en el estado $|1\rangle$. Es la operación fundamental para generar 
\textit{entrelazamiento} (entanglement).
\begin{equation}
    CNOT = 
    \begin{pmatrix} 
        1 & 0 & 0 & 0 \\ 
        0 & 1 & 0 & 0 \\ 
        0 & 0 & 0 & 1 \\ 
        0 & 0 & 1 & 0 
    \end{pmatrix}
\end{equation}

\subsection{La operación SWAP y el Test de Intercambio}
La compuerta SWAP intercambia la información cuántica entre dos qubits: 
$|a\rangle|b\rangle \mapsto |b\rangle|a\rangle$. Su representación matricial es:
\begin{equation}
    SWAP = 
    \begin{pmatrix} 
        1 & 0 & 0 & 0 \\ 
        0 & 0 & 1 & 0 \\ 
        0 & 1 & 0 & 0 \\ 
        0 & 0 & 0 & 1 
    \end{pmatrix}
\end{equation}

Más allá de mover información, esta compuerta es la base del \textbf{Swap Test} 
\cite{buhrman2001quantum}, un algoritmo utilizado para medir la fidelidad o 
similitud entre dos estados cuánticos $|\psi\rangle$ y $|\phi\rangle$.

El circuito del Swap Test utiliza un qubit auxiliar (ancilla). Si medimos el 
qubit auxiliar, la probabilidad de encontrarlo en el estado $|0\rangle$ está 
dada por:
\begin{equation}
    P(0) = \frac{1}{2} + \frac{1}{2}|\langle \psi | \phi \rangle|^2
\end{equation}
Donde $|\langle \psi | \phi \rangle|^2$ representa el solapamiento entre los estados. 
Si $P(0)=1$, los estados son idénticos; si $P(0)=0.5$, los estados son ortogonales.

\subsection{Matrices y compuerta de rotación}
Para la codificación de datos clásicos en estados cuánticos y para la construcción 
de circuitos variacionales parametrizados (PQC), se utilizan rotaciones alrededor 
de los ejes de la esfera de Bloch. En la Tarea I, se utilizó específicamente la 
rotación en el eje X, definida por la exponenciación de la matriz de Pauli 
$\sigma_x$:

\begin{equation}
    R_x(\theta) = e^{-i\theta \sigma_x / 2} = 
    \begin{pmatrix} \cos\frac{\theta}{2} & -i\sin\frac{\theta}{2} \\ -i\sin\frac{\theta}{2} & \cos\frac{\theta}{2} \end{pmatrix}
\end{equation}

\subsection{Implementación de circuitos (Tarea 1)}
Para completar esta tarea se desarrollaron dos circuitos distintos 
siguiendo las especificaciones de GSoC.

\subsubsection{Circuito 1: Manipulación de 5 qubits con Cirq}
El primer ejercicio requirió crear un circuito de 5 qubits aplicando superposición 
(Hadamard), entrelazamiento en cadena (CNOTs), intercambio de estados (SWAP) y una 
rotación parametrizada.

\begin{lstlisting}[language=Python, caption=Implementación del primer circuito utilizando Google Cirq.]
import cirq
import numpy as np

# Definicion de 5 qubits
qubits = [cirq.LineQubit(i) for i in range(5)]
circuit = cirq.Circuit()

# 1. Hadamard en todos los qubits
circuit.append(cirq.H(q) for q in qubits)

# 2. CNOT en cadena: (0,1), (1,2), (2,3), (3,4)
circuit.append(cirq.CNOT(qubits[i], qubits[i+1]) for i in range(4))

# 3. SWAP entre el primero (0) y el ultimo (4)
circuit.append(cirq.SWAP(qubits[0], qubits[4]))

# 4. Rotacion Rx(pi/2) en el qubit central (indice 2)
circuit.append(cirq.rx(np.pi / 2)(qubits[2]))

print(circuit)
\end{lstlisting}

La ejecución de este código genera un diagrama del circuito donde se verifica la 
estructura lineal de las compuertas CNOT y la rotación final. podemos visualizar el
circuito resultante en la Figura \ref{fig:cirq_circuit_1}.

\begin{figure}[H]
        \centering
        \includegraphics[width=0.75\linewidth]{imagenes/Cirq_1_circuit.png}
        \caption{circuito resultante del código con cirq para aplicar el primer 
        conjunto de compuertas.}
        \label{fig:cirq_circuit_1}
\end{figure}

\subsubsection{Circuito 2: Swap Test con PennyLane}
El segundo ejercicio consistió en implementar un \textit{Swap Test} entre dos 
pares de qubits. Se prepararon los estados aplicando Hadamards y una rotación 
$R_x(\pi/3)$ para diferenciar los subsistemas, y luego se midió su similitud.

\begin{lstlisting}[language=Python, caption=Implementación del Swap Test 
    utilizando PennyLane]
import pennylane as qml
import numpy as np

# 5 qubits: 1 ancilla + 4 para el sistema
dev = qml.device("default.qubit", wires=5)

@qml.qnode(dev, interface="autograd")
def swap_test_circuit():
    ancilla = 0
    q1, q2 = 1, 2
    q3, q4 = 3, 4

    # --- Preparacion de Estados ---
    qml.Hadamard(wires=q1)       # Estado A parte 1
    qml.RX(np.pi/3, wires=q2)    # Estado A parte 2
    qml.Hadamard(wires=q3)       # Estado B parte 1
    qml.Hadamard(wires=q4)       # Estado B parte 2

    # --- Rutina de Swap Test ---
    qml.Hadamard(wires=ancilla)
    
    # SWAP Controlado entre los pares (q1,q3) y (q2,q4)
    qml.CSWAP(wires=[ancilla, q1, q3])
    qml.CSWAP(wires=[ancilla, q2, q4])

    qml.Hadamard(wires=ancilla)

    return qml.probs(wires=ancilla)

probs = swap_test_circuit()
print(f"Probabilidad P(0): {probs[0]:.4f}")
\end{lstlisting}

\subsection{Comparación de paradigmas: PennyLane vs Cirq}
Tras la implementación de ambos ejercicios, se identificaron diferencias clave 
en la filosofía de diseño de ambos frameworks:

\begin{itemize}
    \item \textbf{Cirq (Enfoque de Hardware):} Cirq opera a un nivel de abstracción 
    más bajo\cite{cirq}. Su estructura permite definir qubits específicos (como GridQubit 
    para chips Sycamore) y momentos temporales precisos. Es ideal para algoritmos 
    canónicos como Shor (visto en capítulos anteriores) donde la secuencia exacta 
    de compuertas es crítica.
    
    \item \textbf{PennyLane (Enfoque de Software/ML):} Como se observa en el 
    decorador @qml.qnode(..., interface='autograd'), PennyLane está diseñado 
    para la \textbf{Programación Diferenciable} \cite{bergholm2018pennylane}. 
    Trata a los circuitos cuánticos 
    como nodos en un grafo computacional que pueden ser derivados automáticamente. 
    Esto lo hace superior para tareas de Aprendizaje Automático Cuántico (QML), 
    donde es necesario calcular gradientes de los parámetros de rotación (como el 
    ángulo $\pi/3$ en el ejemplo) para optimizar funciones de costo mediante 
    descenso de gradiente.
\end{itemize}

\section{Aprendizaje profundo geométrico: GNN en física}

La física de altas energías, y otras ramas como la química cuántica, comparten un 
desafío fundamental en el análisis de datos: la naturaleza no estructurada y 
relacional de sus sistemas.
A diferencia de las imágenes, que poseen una estructura de cuadrícula euclidiana 
fija, o del texto, que es secuencial, los sistemas físicos —como moléculas o 
chorros de partículas (jets)— se representan mejor como entidades discretas 
interactuantes distribuidas en un espacio métrico. El Aprendizaje Profundo 
Geométrico (\textit{Geometric Deep Learning}) proporciona el marco teórico 
para abordar estos dominios irregulares, siendo las Redes Neuronales de Grafos 
(GNN) su exponente más exitoso.

\subsection{Fundamentos de grafos}

Un grafo es una estructura matemática que modela un conjunto de objetos y las 
relaciones entre ellos. Formalmente, definimos un grafo $G$ como una tupla 
$G = (V, E)$, donde $V$ es el conjunto de vértices o nodos y $E$ representa el 
conjunto de aristas que conectan pares de nodos.

En el contexto del aprendizaje automático aplicado a la física, extendemos esta 
definición para incluir atributos, \textit{features}, asociados a cada componente
del grafo:

\begin{itemize}
    \item \textbf{Atributos de Nodo ($\mathbf{h}_v$):} Cada nodo $v \in V$ tiene 
    asociado un vector de características $\mathbf{h}_v \in \mathbb{R}^{d_v}$, 
    que describe el estado local de la entidad, por ejemplo el tipo de átomo, 
    momento de una partícula.
    \item \textbf{Atributos de Arista ($\mathbf{e}_{vw}$):} Cada arista $(v, w) 
    \in E$ puede tener un vector de características $\mathbf{e}_{vw} \in 
    \mathbb{R}^{d_e}$, que codifica la relación entre los nodos, por ejemplo 
    la distancia espacial o el tipo de enlace químico.
    \item \textbf{Atributos Globales ($\mathbf{u}$):} Opcionalmente, el grafo 
    puede tener un vector global $\mathbf{u}$ que representa propiedades del 
    sistema completo.
\end{itemize}

La estructura conectiva del grafo se puede representar mediante la matriz de 
adyacencia $\mathbf{A} \in \{0, 1\}^{|V| \times |V|}$, donde $a_{ij} = 1$ si 
existe una arista entre el nodo $i$ y el nodo $j$, y $0$ en caso contrario.

Las GNNs operan sobre esta estructura aprovechando una propiedad clave para la 
física: la \textbf{invariancia a las permutaciones}. Dado que no existe un orden 
canónico para enumerar las partículas en una colisión o los átomos en una 
molécula, cualquier función de aprendizaje $f(G)$ debe cumplir que 
$f(\mathbf{P}\mathbf{A}\mathbf{P}^T, \mathbf{P}\mathbf{H}) = f(\mathbf{A}, 
\mathbf{H})$ para cualquier matriz de permutación $\mathbf{P}$. Esto asegura 
que las predicciones físicas no dependan del orden arbitrario de entrada de 
los datos.

\subsection{Representación de jets como grafos}

En física de altas energía, particularmente cuando se trabaja con colisionadores, 
un \textit{jet} es un chorro colimado de hadrones 
producido por la hadronización de un quark o gluón de alta energía. 
Tradicionalmente, los jets se han analizado proyectando su energía en una 
cuadrícula (imágenes calorimétricas) o tratándolos como secuencias. Sin embargo, 
estas representaciones imponen una estructura artificial o pierden información 
sobre la dispersión natural de las partículas.

Una representación más natural a la física del proceso es considerar el jet como una 
«nube de puntos» o un grafo. Definimos formalmente un jet como un grafo 
dirigido $G_{jet} = (V, E)$ de la siguiente manera:

\begin{itemize}
    \item \textbf{Nodos ($V$):} Cada nodo representa una partícula constituyente 
    del jet. El vector de características inicial del nodo $i$, denotado 
    como $\mathbf{h}_i^{(0)}$, incluye variables cinemáticas fundamentales:
    \begin{equation}
        \mathbf{h}_i^{(0)} = (\Delta\eta_i, \Delta\phi_i, \log p_{T,i}, 
        \dots)
    \end{equation}
    Donde $\Delta\eta_i$ y $\Delta\phi_i$ son las coordenadas de 
    pseudorapidez y ángulo azimutal relativas al eje del jet, y $p_{T,i}$ 
    es el momento transversal.

    \item \textbf{Aristas ($E$):} La topología del grafo se construye 
    dinámicamente para capturar correlaciones locales. Comúnmente se 
    utiliza el algoritmo de $k$-vecinos más cercanos ($k$-NN). Para cada 
    partícula $i$, se establecen aristas dirigidas $e_{ij} \in E$ hacia sus $k$ 
    vecinos más cercanos en el espacio $(\Delta\eta, \Delta\phi)$. El peso o 
    característica de la arista $\mathbf{e}_{ij}$ se define mediante la métrica 
    euclidiana en este espacio:
    \begin{equation}
        \mathbf{e}_{ij} = \sqrt{(\Delta\eta_i - \Delta\eta_j)^2 + 
        (\Delta\phi_i - \Delta\phi_j)^2}
    \end{equation}
\end{itemize}

Esta representación permite que la red neuronal aprenda directamente de la 
estructura de la lluvia de partículas (\textit{parton shower}), donde las 
partículas cercanas en el espacio angular están probablemente relacionadas 
por el mismo proceso de desintegración.

\begin{figure}[H]
    \centering
    \includegraphics[width=0.8\linewidth]{imagenes/jet_graph_representation.png}
    \caption{Ilustración de la transformación de un evento de colisión a una 
    representación de grafo. Las partículas (nodos) se conectan con sus vecinos 
    espaciales más cercanos (aristas), formando una estructura irregular que 
    preserva la geometría del evento.}
    \label{fig:jet_graph}
\end{figure}

\subsection{Formalismo del Paso de Mensajes (Message Passing)}

El marco teórico que unifica la mayoría de las arquitecturas de GNN modernas es 
el de las \textit{Message Passing Neural Networks} (MPNN), formalizado por 
Gilmer et al. (2017) en el contexto de la química cuántica. Este formalismo 
describe el proceso de aprendizaje como un intercambio iterativo de información 
entre nodos a través de las aristas.

El algoritmo de paso de mensajes se ejecuta durante $T$ pasos temporales. 
En cada paso $t$, el estado oculto de cada nodo $\mathbf{h}_v^t$ se actualiza 
basándose en su estado anterior y en la agregación de los mensajes recibidos 
de su vecindario $\mathcal{N}(v)$. El proceso se define mediante dos funciones 
diferenciables principales: la función de mensaje $M_t$ y la función de 
actualización $U_t$.

\textbf{1. Fase de Mensaje y Agregación:}
Para cada nodo $v$, se calcula un mensaje agregado $\mathbf{m}_v^{t+1}$ sumando 
la información proveniente de sus vecinos:
\begin{equation}
    \mathbf{m}_v^{t+1} = \sum_{w \in \mathcal{N}(v)} M_t(\mathbf{h}_v^t, 
    \mathbf{h}_w^t, \mathbf{e}_{vw})
    \label{eq:message_passing}
\end{equation}
Donde:
\begin{itemize}
    \item $M_t$ es una red neuronal (típicamente un perceptrón multicapa o MLP) 
    que computa la interacción entre el nodo $v$, su vecino $w$ y la arista que 
    los une.
    \item La sumatoria $\sum$ actúa como un operador de agregación invariante a 
    permutaciones, asegurando que el resultado no dependa del orden en que se 
    procesan los vecinos.
\end{itemize}

\textbf{2. Fase de Actualización:}
El estado del nodo se actualiza combinando su estado actual con el mensaje agregado:
\begin{equation}
    \mathbf{h}_v^{t+1} = U_t(\mathbf{h}_v^t, \mathbf{m}_v^{t+1})
    \label{eq:update_function}
\end{equation}
Donde $U_t$ suele ser una red neuronal recurrente (como una GRU o LSTM) o 
simplemente otra capa densa.

\textbf{3. Fase de Lectura (Readout):}
Después de $K$ iteraciones de paso de mensajes, la información local se ha 
propagado por el grafo. Para tareas de clasificación a nivel de grafo (como 
distinguir si un jet es quark o gluón), se utiliza una función de lectura $R$ 
para generar una predicción global $\hat{y}$:
\begin{equation}
    \hat{y} = R(\{\mathbf{h}_v^K | v \in G\})
\end{equation}

La función $R$ debe ser también invariante al orden de los nodos, por ejemplo la
suma global o promedio global. Este formalismo permite modelar interacciones 
complejas de muchos cuerpos mediante la composición 
de interacciones locales simples.

\begin{figure}[H]
    \centering
    \includegraphics[width=0.9\linewidth]{imagenes/mpnn_diagram.png}
    \caption{Diagrama esquemático de una capa de Message Passing. (A) Los nodos 
    vecinos envían mensajes a través de las aristas. (B) Los mensajes se agregan 
    (suma). (C) El estado del nodo central se actualiza con la información agregada.}
    \label{fig:mpnn_scheme}
\end{figure}

\subsection{Mecanismo de atención (GAT)}

Una limitación de las MPNN básicas es que el operador de agregación (Ec. 
\ref{eq:message_passing}) suele tratar a todos los vecinos por igual o basarse 
únicamente en características estáticas de la arista. En física de partículas, 
sin embargo, no todas las interacciones son igual de relevantes; una partícula 
con alto momento transversal ($p_T$) cercana al eje del jet puede contener mucha 
más información sobre el proceso original que una partícula de baja energía en la 
periferia.

Para incorporar esta intuición física, empleamos las Redes de Atención de Grafos 
(Graph Attention Networks, GAT). Las GAT introducen un mecanismo de atención que 
permite a la red aprender \textit{qué} vecinos son más importantes para la 
actualización de un nodo específico.

El núcleo de la GAT es el cálculo de los coeficientes de atención $\alpha_{ij}$. 
Primero, se calcula un puntaje de importancia no normalizado $e_{ij}$ entre el 
nodo $i$ y su vecino $j$:
\begin{equation}
    e_{ij} = \text{LeakyReLU}\left( \mathbf{a}^T [\mathbf{W}\mathbf{h}_i \| 
    \mathbf{W}\mathbf{h}_j] \right)
\end{equation}
Donde:
\begin{itemize}
    \item $\mathbf{W}$ es una matriz de transformación lineal aprendible 
    aplicada a cada nodo.
    \item $\mathbf{a}$ es el vector de parámetros del mecanismo de atención.
    \item $\|$ denota la concatenación de vectores.
    \item LeakyReLU es la función de activación que permite gradientes para 
    valores negativos.
\end{itemize}

Para hacer los coeficientes comparables entre diferentes nodos, se normalizan 
utilizando la función softmax sobre todos los vecinos $k \in \mathcal{N}(i)$:
\begin{equation}
    \alpha_{ij} = \frac{\exp(e_{ij})}{\sum_{k \in \mathcal{N}(i)} \exp(e_{ik})}
    \label{eq:attention_coeff}
\end{equation}

Finalmente, la agregación de mensajes se convierte en una suma ponderada por 
estos coeficientes:
\begin{equation}
    \mathbf{h}_i' = \sigma \left( \sum_{j \in \mathcal{N}(i)} \alpha_{ij} 
    \mathbf{W}\mathbf{h}_j \right)
\end{equation}

En el contexto de la física de jets, el coeficiente $\alpha_{ij}$ puede 
interpretarse como la importancia relativa de la partícula $j$ para determinar 
la naturaleza de la partícula $i$. Esto permite que el modelo filtre el ruido de 
fondo y se enfoque en las partículas energéticas que definen la 
subestructura del jet, mejorando significativamente la capacidad de clasificación 
y proporcionando interpretabilidad al modelo.

\subsection{Implementación y resultados}

Para evaluar la eficacia de las arquitecturas geométricas propuestas, se implementó 
un flujo de trabajo completo en \textit{PyTorch Geometric}, abarcando desde la 
ingestión de datos crudos hasta la validación de los modelos MPNN y GAT.

\subsubsection{Base de datos usada: ParticleNet}

El estudio utilizó el conjunto de datos estándar de \textbf{ParticleNet} para la 
clasificación de jets de quarks y gluones. Este dataset consta de eventos simulados 
con el generador Pythia, proporcionando una representación de «nube de puntos» 
ideal para el aprendizaje basado en grafos.

\textbf{Características del Dataset:}
\begin{itemize}
    \item \textbf{Volumen:} El conjunto total consta de 100,000 eventos, 
    balanceados equitativamente entre jets de quarks (señal) y jets de gluones 
    (fondo).
    \item \textbf{Atributos de Entrada:} Cada jet está compuesto por una lista de 
    partículas, donde cada partícula posee cuatro características fundamentales:
    \begin{itemize}
        \item Momento transversal ($p_T$).
        \item Rapidez ($y$).
        \item Ángulo azimutal ($\phi$).
        \item Identificador de partícula (pdgid).
    \end{itemize}
    \item \textbf{Preprocesamiento y Construcción del Grafo:}
    Dado que los jets tienen una multiplicidad variable, los datos originales 
    incluyen padding (relleno con ceros). Para la construcción del grafo, se 
    implementó un filtro, threshold=1e-6, para descartar estas partículas 
    ficticias.
    
    A diferencia de los enfoques estáticos, se optó por una topología de 
    \textbf{Grafo Totalmente Conectado} para cada jet. Esto significa que cada 
    partícula (nodo) tiene una arista dirigida hacia todas las demás partículas 
    del jet. Si bien esto incrementa el costo computacional respecto a un enfoque 
    de $k$-vecinos más cercanos ($k$-NN), permite que el mecanismo de paso de 
    mensajes capture interacciones globales de largo alcance dentro del cono del 
    jet, delegando a la red neuronal la tarea de aprender qué conexiones son 
    relevantes.
\end{itemize}

\textbf{División de Datos:}
Se utilizaron dos archivos \textit{.npz} para el entrenamiento y validación 
(\textit{QG\_jets.npz}, \textit{QG\_jets\_11.npz}), 
reservando un tercer archivo independiente (\textit{QG\_jets\_12.npz}) 
exclusivamente para la evaluación final (Test Set).

\subsubsection{Resultados}

El entrenamiento se realizó utilizando el optimizador \textbf{Adam} con una 
tasa de aprendizaje inicial de $10^{-3}$ y un decaimiento de pesos 
(\textit{weight decay}) de $5 \times 10^{-4}$ para regularización L2. 
Se implementó una estrategia de \textit{Early Stopping} monitoreando la 
pérdida de validación, con una paciencia de 4 épocas para prevenir el 
sobreajuste y optimizar el uso de recursos.

Ambos modelos lograron converger, aunque exhibieron dinámicas de aprendizaje 
marcadamente distintas:
\begin{itemize}
    \item \textbf{Eficiencia Computacional:} La arquitectura MPNN fue 
    notablemente más rápida en completar su entrenamiento. Debido a su 
    simplicidad operativa, agregación por suma; cada época requirió menos 
    tiempo de cómputo en comparación con la GAT, cuyo mecanismo de atención 
    tiene una complejidad cuadrática $O(N^2)$ respecto al número de nodos en 
    el grafo totalmente conectado.
    \item \textbf{Dinámica de Aprendizaje:} La MPNN mostró una convergencia 
    rápida y estable, activando el criterio de parada temprana en la época 7, 
    tras dejar de mejorar en la época 3. Por el contrario, la GAT continuó 
    ajustando sus pesos más lentamente, alcanzando su mejor rendimiento hacia 
    la época 6, lo que evidencia una superficie de optimización más compleja 
    debido a los coeficientes de atención.
\end{itemize}

Para visualizar el desempeño detallado de cada clasificador, presentamos 
las matrices de confusión normalizadas calculadas sobre el conjunto de prueba.

\begin{figure}[H]
    \centering
    \begin{minipage}{0.48\textwidth}
        \centering
        \includegraphics[width=0.9\linewidth]{imagenes/matriz_confusion_MPNN.png}
        \caption{Matriz de confusión MPNN.}
        \label{fig:mpnn_matriz_confusion}
    \end{minipage}\hfill
    \begin{minipage}{0.48\textwidth}
        \centering
        \includegraphics[width=0.9\linewidth]{imagenes/matriz_confusion_GAT.png}
        \caption{Matriz de confusión GAT.}
        \label{fig:conf_matrix_gat}
    \end{minipage}
\end{figure}

\begin{figure}[H]
    \centering
    \begin{minipage}{0.48\textwidth}
        \centering
        % Representación tabular de la matriz de confusión para MPNN
        % Datos crudos: TP_g=37776, FP_q=12224, FN_q=10717, TP_q=39283
        \begin{tabular}{|c|c|c|}
            \hline
            \textbf{MPNN} & Pred: Gluón & Pred: Quark \\
            \hline
            Real: Gluón & \textbf{TP$_{g}$: 75.6\%} & FP$_{q}$: 24.4\% \\
            \hline
            Real: Quark & FN$_{q}$: 21.4\% & \textbf{TP$_{q}$: 78.6\%} \\
            \hline
        \end{tabular}
        \caption{Matriz de confusión MPNN normalizada.}
        \label{fig:conf_matrix_mpnn_normalizada}
    \end{minipage}\hfill
    \begin{minipage}{0.48\textwidth}
        \centering
        % Representación tabular de la matriz de confusión para GAT
        % Datos crudos: TP_g=43707, FP_q=6293, FN_q=17056, TP_q=32944
        \begin{tabular}{|c|c|c|}
            \hline
            \textbf{GAT} & Pred: Gluón & Pred: Quark \\
            \hline
            Real: Gluón & \textbf{TP$_{g}$: 87.4\%} & FP$_{q}$: 12.6\% \\
            \hline
            Real: Quark & FN$_{q}$: 34.1\% & \textbf{TP$_{q}$: 65.9\%} \\
            \hline
        \end{tabular}
        \caption{Matriz de confusión GAT normalizada.}
        \label{fig:conf_matrix_gat_normalizada}
    \end{minipage}
\end{figure}

\textbf{Análisis de Métricas:}
La Tabla \ref{tab:gnn_comparison} resume las métricas finales en el conjunto de 
prueba. Si bien la exactitud global (\textit{Accuracy}) es comparable entre ambos 
modelos ($\approx 77\%$), la distribución de los errores revela especializaciones 
opuestas:

\begin{itemize}
    \item La \textbf{GAT} prioriza la \textbf{Precisión} (83.96\%), minimizando 
    drásticamente los Falsos Positivos de quarks, solo 12.6\% de gluones mal clasificados. 
    Esto indica que el mecanismo de atención aprende a filtrar el ruido de fondo 
    eficazmente.
    \item La \textbf{MPNN} prioriza el \textbf{Recall} (78.57\%), maximizando la 
    detección de quarks reales, aunque a costa de una menor pureza, mayor 
    contaminación por gluones.
\end{itemize}

\begin{table}[H]
    \centering
    \begin{tabular}{|l| c c c c|}
        \toprule
        \textbf{Modelo} & \textbf{Accuracy} & \textbf{Precision} & \textbf{Recall} & \textbf{Training Time} \\
        \midrule
        MPNN (Message Passing) & 0.7706 & 0.7627 & \textbf{0.7857} & $\approx$ 2h 11m \\
        GAT (Graph Attention) & 0.7665 & \textbf{0.8396} & 0.6589 & $>$ 4h \\
        \bottomrule
    \end{tabular}
    \caption{Comparación de rendimiento entre arquitecturas GNN en el dataset 
    de prueba ParticleNet.}
    \label{tab:gnn_comparison}
\end{table}

Estos resultados confirman que la topología del grafo contiene bien la información 
física: la GAT explota las correlaciones fuertes para asegurar predicciones 
de alta confianza, mientras que la MPNN utiliza la información global agregada 
para no perder señales potenciales.

\subsubsection{Discusión}

El análisis comparativo permite establecer criterios claros para la selección del 
modelo según el objetivo físico del análisis:

\textbf{1. Compromiso Precisión-Recall (Pureza vs. Eficiencia)}
\begin{itemize}
    \item \textbf{GAT para Alta Pureza:} Con una precisión cercana al 84\%, la 
    GAT es la arquitectura ideal para mediciones de precisión donde es crítico 
    que la muestra de quarks esté libre de contaminación por gluones, aun si 
    eso implica descartar algunos eventos válidos.
    \item \textbf{MPNN para Máxima Eficiencia:} Con un recall del 78.6\%, la 
    MPNN es superior para búsquedas de señales exóticas o raras, búsqueda de 
    nueva física; donde el costo de perder un evento de señal 
    es inaceptable y se tolera un mayor ruido de fondo.
\end{itemize}

\textbf{2. Costo computacional y escalabilidad}
\begin{itemize}
    \item \textbf{Escalabilidad de GAT:} El costo cuadrático de la atención limita 
    su uso en grafos muy densos o en entornos con recursos limitados.
    \item \textbf{Viabilidad de MPNN:} Su convergencia rápida y menor huella 
    computacional la hacen candidata para despliegues en tiempo real, como en los 
    sistemas de selección de eventos del LHC, donde la latencia 
    es un factor crítico.
\end{itemize}

\subsubsection{Conclusiones y trabajo futuro}

El estudio realizado demuestra que las arquitecturas basadas en grafos, tanto 
MPNN como GAT, logran explotar exitosamente la estructura geométrica intrínseca 
de los datos de jets en \textit{ParticleNet}, superando las limitaciones de los 
enfoques tradicionales que dependen de variables de alto nivel pre-calculadas.

La comparación directa revela una dicotomía clara en la aplicación de estos modelos:
\begin{itemize}
    \item La \textbf{GAT (Graph Attention Network)} se establece como la 
    herramienta superior para tareas de \textbf{alta pureza}. Su capacidad para 
    asignar pesos de importancia a las partículas más relevantes le permite filtrar 
    el ruido de fondo con gran eficacia, logrando una precisión del 83.96\% en la 
    identificación de quarks.
    \item La \textbf{MPNN (Message Passing Neural Network)} destaca por su 
    \textbf{eficiencia de señal} y velocidad. Al capturar una mayor cantidad de 
    quarks reales (Recall del 78.57\%) con un costo computacional significativamente 
    menor, se presenta como la opción viable para análisis donde la estadística 
    es prioritaria sobre la pureza.
\end{itemize}

La elección entre una y otra no es absoluta, sino que debe guiarse por los 
requisitos específicos del análisis físico: minimizar falsos positivos (GAT) o 
maximizar la eficiencia de detección (MPNN).

A partir de los hallazgos de esta investigación, se identifican tres líneas 
principales para extender y mejorar el uso de GNNs en física de altas energías:

\begin{enumerate}
    \item \textbf{Exploración de representaciones de grafo alternativas:}
    Actualmente, la topología totalmente conectada asegura que no se pierda 
    información, pero a un alto costo. Futuros trabajos podrían investigar 
    métodos de construcción de grafos más sofisticados, como el uso de 
    umbrales de distancia dinámica ($\Delta R < R_{cut}$) o métricas de similitud 
    inspiradas en la física, por ejemplo la invariancia de Lorentz; para definir 
    las aristas, reduciendo la complejidad computacional sin sacrificar la física.

    \item \textbf{Mejoras en la arquitectura del modelo:}
    Se propone investigar arquitecturas con mayor poder expresivo teórico, como 
    las \textit{Graph Isomorphism Networks} (GIN), o desarrollar modelos híbridos 
    que combinen la eficiencia del paso de mensajes en las primeras capas 
    (para agregación local) con la selectividad de la atención en las capas 
    profundas (para razonamiento global).

    \item \textbf{Optimización de hiperparámetros:}
    Debido a las restricciones de tiempo del ejercicio, la exploración del 
    espacio de hiperparámetros fue limitada. Un estudio de ablación exhaustivo y
    una sintonización fina (\textit{fine-tuning}) de parámetros como la dimensión 
    del espacio latente, el número de cabezas de atención y las tasas de 
    \textit{dropout} podrían desbloquear un rendimiento superior en la clasificación.
\end{enumerate}

\section{Redes neuronales Kolmogorov-Arnold (KAN)}

Durante la última década, el Perceptrón Multicapa (MLP) ha permanecido como la 
unidad fundamental del aprendizaje profundo. Basado en el \textit{Teorema de Aproximación 
Universal}, los MLPs han demostrado ser increíblemente buenos en aproximar 
funciones no lineales en espacios de varias dimensiones. Sin embargo, recientemente 
se ha propuesto un cambio de paradigma inspirado en un resultado fundamental de 
la teoría de funciones de los años 50: las Redes de Kolmogorov-Arnold, KAN.

A diferencia de las redes tradicionales que fijan las funciones de activación en 
los nodos (neuronas) y aprenden pesos lineales en las aristas, las KANs invierten 
esta lógica, colocando funciones de activación aprendibles en las aristas y 
utilizando nodos como sumatorios simples. Este enfoque, propuesto formalmente 
para el aprendizaje automático por Liu et al. (2025) \cite{liu2025kan}, 
promete ventajas significativas en términos de interpretabilidad y eficiencia 
de parámetros para tareas científicas.

\subsection{El teorema de Kolmogorov-Arnold}

\subsubsection{Teorema y explicación}
El fundamento matemático de esta arquitectura reside en el \textbf{Teorema de 
Representación de Kolmogorov-Arnold}. Originalmente formulado por Andrey 
Kolmogorov (1957) \cite{kolmogorov1957representation} y refinado por Vladimir 
Arnold, este teorema resolvió el decimotercer problema de Hilbert, demostrando 
que cualquier función continua multivariada puede reducirse a una composición 
finita de funciones continuas de una sola variable y la operación de suma.

\textbf{Teorema} 

Para cualquier función continua multivariada $f: [0,1]^n \to \mathbb{R}$, 
existen funciones continuas univariadas $\Phi_q: \mathbb{R} \to \mathbb{R}$ y 
$\phi_{q,p}: [0,1] \to \mathbb{R}$ tales que:

\begin{equation}
    f(x_1, \dots, x_n) = \sum_{q=0}^{2n} \Phi_q \left( \sum_{p=1}^n 
    \phi_{q,p}(x_p) \right)
    \label{eq:kolmogorov_theorem}
\end{equation}

Donde:
\begin{itemize}
    \item $n$ es la dimensión de la entrada.
    \item $x_p$ es la $p$-ésima componente del vector de entrada $\mathbf{x}$.
    \item $\phi_{q,p}$ son las funciones internas que 
    transforman cada coordenada individualmente.
    \item $\Phi_q$ son las funciones externas que 
    procesan la suma de las transformaciones internas.
    \item La sumatoria externa se realiza sobre $2n+1$ términos.
\end{itemize}

Históricamente, este teorema fue considerado una curiosidad teórica en el 
contexto de la aproximación numérica \cite{schmidt2021kolmogorov}. 
La razón es que las funciones internas $\phi_{q,p}$ garantizadas por el teorema 
suelen ser no suaves, como los fractales, y altamente dependientes de la función 
objetivo $f$ específica, lo que las hacía inútiles para el aprendizaje constructivo. 
Sin embargo, la innovación de las KANs modernas radica en relajar la restricción 
de usar exactamente $2n+1$ términos y, en su lugar, parametrizar estas funciones 
univariadas como curvas suaves, \textit{splines}, que pueden aprenderse mediante 
descenso de gradiente.

\begin{figure}[H]
    \centering
    \includegraphics[width=0.9\linewidth]{imagenes/Flujo_KAN.png}
    \caption{Representación esquemática del Teorema de Kolmogorov-Arnold. 
    Cada entrada $x_p$ pasa por funciones univariadas no lineales $\phi$ antes 
    de ser sumada.
    Izquierda: Notaciones de las activaciones que fluyen a través de la red. 
    Derecha: Una función de activación está parametrizada como B-spline, lo que 
    permite alternar entre cuadrículas gruesas y finas. Imagen 
    extraída de liu et al. (2025) \cite{liu2025kan}.}
    \label{fig:ka_theorem}
\end{figure}

\subsubsection{Comparación con MLP y el teorema de aproximación universal}

La diferencia estructural entre un perceptrón multicapa, MLP, y una KAN es 
profunda y radica en dónde reside la no-linealidad aprendible.

\textbf{1. El enfoque MLP (activación en nodos):}

Un MLP se basa en el Teorema de Aproximación Universal de Cybenko 
\cite{cybenko1989approximation}, el cual establece que una red con suficiente 
profundidad puede aproximar cualquier función. Estructuralmente, el procesamiento 
en un MLP ocurre en dos etapas: primero se realiza una \textbf{combinación lineal} 
de las entradas (mediante una matriz de pesos $W$) y posteriormente se aplica una 
función de activación no lineal fija $\sigma$ (como ReLU o Sigmoide):

\begin{equation}
    f_{MLP}(\mathbf{x}) = \sigma \left( \sum_{j=1}^{d_{in}} w_{ij} x_j + b_i \right)
\end{equation}

Aquí, los parámetros aprendibles son los pesos lineales $w_{ij}$ y los sesgos $b_i$. 
La función $\sigma$ es estática y no evoluciona durante el entrenamiento. 
Esto limita la expresividad de cada neurona individual, obligando a la red a 
incrementar su número de neuronas para modelar comportamientos complejos.

\textbf{2. El enfoque KAN (activación en aristas):}

Una KAN generaliza la Ec. \ref{eq:kolmogorov_theorem} a capas arbitrarias de 
profundidad. En una capa KAN, la transformación entre la capa $l$ y $l+1$ se 
define como:
\begin{equation}
    x_i^{(l+1)} = \sum_{j=1}^{n_l} \phi_{l,j,i} (x_j^{(l)})
\end{equation}
Aquí, $\phi_{l,j,i}$ no es un peso escalar, sino una \textbf{función univariada 
aprendible} que conecta el nodo $j$ de la capa anterior con el nodo $i$ de la 
capa actual.

Esta distinción tiene consecuencias críticas:
\begin{itemize}
    \item \textbf{Eficiencia de Parámetros:} Mientras que un MLP necesita apilar 
    muchas capas y neuronas para aproximar una función compleja, como una 
    multiplicación o una función oscilatoria; una KAN puede aprender la función 
    exacta en las aristas. Por ejemplo, una KAN puede aprender $f(x,y) = xy$ 
    usando la identidad logarítmica $\log(xy) = \log x + \log y$, reduciendo la 
    red a sumas y funciones $\phi(x)=\log(x)$ y $\Phi(x)=\exp(x)$, algo 
    extremadamente difícil para un MLP estándar \cite{liu2025kan}.
    \item \textbf{Interpretabilidad:} En un MLP, los pesos $W$ son matrices 
    densas difíciles de interpretar. En una KAN, cada conexión es una función 
    $y = \phi(x)$ que puede visualizarse gráficamente. Si una conexión aprende 
    una función cercana a cero, puede podarse; si aprende una curva cuadrática, 
    puede sustituirse simbólicamente por $x^2$.
\end{itemize}

\begin{figure}[H]
    \centering
    \begin{tabular}{|c|c|c|}
            \hline
            modelo 
            & \textbf{Perceptrón multicapa (MLP)} 
            & \textbf{Red Kolmogorov-Arnold (KAN)} \\
            \hline
            Teorema & Teorema de aproximación universal & Teorema de representación Kolmogorov Arnold \\
            \hline
            Fórmula 
            & $f(\mathbf{x}) \approx \sum^{N(\epsilon)}_{i=1} a_i \sigma(\mathbf{w}_i \cdot \mathbf{x} + b_i)$ 
            & $f(x) = \sum^{2n+1}_{q=1} \Phi_q \left( \sum^{n}_{p=1} \phi_{q,p}(x_p) \right)$ \\
            \hline
            Modelo 
            & \includegraphics[width=0.35\linewidth]{imagenes/MLP_esquema.png} 
            & \includegraphics[width=0.35\linewidth]{imagenes/KAN_esquema.png} \\
            \hline
        \end{tabular}
    \caption{Comparación arquitectónica. (Izquierda) MLP: Las aristas son pesos 
    lineales $w$ y los nodos aplican la no-linealidad $\sigma$. (Derecha) KAN: 
    Las aristas contienen funciones aprendibles $\phi$ y los nodos simplemente 
    suman las señales entrantes. Imagen extraída de liu et al. (2025) 
    \cite{liu2025kan}.}
    \label{fig:mlp_vs_kan}
\end{figure}


\subsection{B-splines y capas densas}

Para implementar computacionalmente las funciones $\phi(x)$ descritas en el 
teorema, no podemos optimizar en el espacio de todas las funciones continuas. 
Debemos restringirnos a un espacio paramétrico flexible. Liu et al. proponen el 
uso de \textbf{B-splines} (Basis Splines).

Una función de activación en una KAN se parametriza como la suma de una función 
base, para regularización y estabilidad, y una combinación lineal de funciones 
B-spline:

\begin{equation}
    \phi(x) = w_b b(x) + w_s \text{spline}(x)
\end{equation}

Donde:
\begin{itemize}
    \item $b(x)$ es una función de activación base, típicamente la función 
    SiLU ($x / (1 + e^{-x})$).
    \item $w_b, w_s$ son factores de escala aprendibles.
    \item $\text{spline}(x)$ se expresa como una combinación lineal de $k$ 
    funciones base B-spline $B_i(x)$ definidas en una retícula (\textit{grid}) de 
    puntos de control $c_i$:
    \begin{equation}
        \text{spline}(x) = \sum_{i} c_i B_i(x)
    \end{equation}
\end{itemize}

\textbf{Optimización y extensión de la cuadrícula (Grid Extension)}

Los coeficientes $c_i$ son los parámetros que se actualizan mediante descenso de 
gradiente. Una ventaja clave de los B-splines es su localidad: modificar un 
coeficiente $c_i$ solo afecta la forma de la función en un intervalo local, 
evitando el «olvido» global que sufren los MLPs.

Además, esta parametrización permite una propiedad única llamada \textit{Grid 
Extension}. Se puede comenzar entrenando una KAN con una retícula gruesa, 
pocos puntos de control, pocos parámetros; para aprender la estructura global 
de la función, y luego refinar la retícula añadiendo más puntos de control, 
inicializando sus valores para preservar la forma de la función aprendida. 
Esto permite escalar la precisión del modelo sin reentrenar desde cero, algo 
imposible en arquitecturas densas tradicionales.

\textbf{Capas KAN (KANLinear):}
En la implementación práctica (usando PyTorch), una Capa KAN sustituye a la capa 
lineal convencional (\textit{nn.Linear}). Dado un batch de entrada $\mathbf{X}$ de 
dimensión $(Batch, N_{in})$, la capa calcula la salida $(Batch, N_{out})$ 
evaluando todos los B-splines correspondientes a las $N_{in} \times N_{out}$ 
aristas y sumando los resultados. Aunque computacionalmente más costoso que una 
simple multiplicación matricial debido a la evaluación de los polinomios spline, 
la reducción dramática en el número de parámetros necesarios para alcanzar una 
precisión dada suele compensar el costo.

\subsection{Base de datos: MNIST}

Para evaluar la capacidad de las KANs en tareas de visión por computadora y 
clasificación, se utilizó el conjunto de datos \textbf{MNIST} 
\cite{lecun1998mnist}. Aunque las KANs fueron diseñadas primariamente para 
regresión y descubrimiento de leyes físicas, probarlas en 
clasificación de imágenes permite verificar su versatilidad.

\textbf{Descripción del Dataset:}
MNIST consiste en una colección de imágenes en escala de grises de dígitos 
manuscritos del 0 al 9.
\begin{itemize}
    \item \textbf{Dimensiones:} Cada imagen tiene un tamaño de $28 \times 28$ píxeles.
    \item \textbf{Conjuntos:} 60,000 imágenes de entrenamiento y 10,000 de prueba.
    \item \textbf{Clases:} 10 clases balanceadas (dígitos 0-9).
\end{itemize}

\textbf{Preprocesamiento para KAN:}
Dado que la arquitectura KAN actual opera sobre vectores de características,
similar a un MLP y a diferencia de una CNN que opera sobre tensores espaciales; 
las imágenes de entrada se someten a un aplanamiento (\textit{flattening}):
\begin{equation}
    \mathbb{R}^{28 \times 28} \xrightarrow{\text{Flatten}} \mathbb{R}^{784}
\end{equation}
El vector resultante de 784 dimensiones se normaliza al rango $[0, 1]$ o $[-1, 1]$ 
para asegurar que los valores de entrada caigan dentro del dominio de soporte de 
los B-splines definidos.

Este experimento busca responder una pregunta crucial: ¿Pueden las funciones de 
activación aprendibles en las aristas capturar las correlaciones espaciales 
complejas de una imagen con la misma eficacia que las combinaciones lineales 
profundas de un MLP? Los resultados de esta implementación se discuten en la 
siguiente sección.


\subsection{Arquitecturas usadas}

Para evaluar empíricamente el potencial de las KANs, se implementaron dos 
arquitecturas distintas en el problema de clasificación MNIST. Ambas comparten 
el principio fundamental de descomposición de Kolmogorov-Arnold, pero difieren 
en cómo integran o sustituyen los componentes lineales tradicionales.

\subsubsection{Arquitectura 1: KAN Pura basada en B-Splines}
Esta arquitectura busca ser fiel a la formulación matemática original 
(Ec. \ref{eq:kolmogorov_theorem}), eliminando por completo las multiplicaciones 
de matrices de pesos convencionales.

\begin{itemize}
    \item \textbf{Capa de Transformación ($\psi$):} La entrada de 784 dimensiones 
    pasa por una capa \texttt{BSplineLayer} que aplica una función univariada 
    aprendible a cada característica individualmente. Cada función se parametriza 
    mediante 50 puntos de control ($k=50$) distribuidos en el intervalo $[0, 1]$.
    \item \textbf{Capa de Suma Agrupada:} Dado que una KAN completa tendría 
    $O(N^2)$ conexiones, implementamos una versión dispersa agrupando las 
    características en 28 grupos. La capa \texttt{GroupSumLayer} realiza la 
    sumatoria de las señales transformadas dentro de cada grupo.
    \item \textbf{Capa de Salida ($\Phi$):} Una segunda capa de B-Splines 
    procesa las sumas agrupadas antes de la clasificación final (Softmax).
\end{itemize}
Esta red no contiene ninguna capa \texttt{Dense} (matriz $W$ completa); todo el 
aprendizaje ocurre modificando la forma de las curvas de activación en las aristas.

\subsubsection{Arquitectura 2: KAN Híbrida}
Reconociendo que las KANs puras pueden carecer de capacidad para capturar 
interacciones globales complejas rápidamente, se diseñó un modelo híbrido.
\begin{itemize}
    \item Mantiene las capas iniciales de B-Splines y Suma Agrupada de la 
    arquitectura pura para la extracción de características no lineales.
    \item \textbf{Componente MLP:} La salida de la KAN se aplana y se concatenan 
    a un Perceptrón Multicapa convencional con dos capas densas (128 y 64 neuronas) 
    y activaciones ReLU.
\end{itemize}

El objetivo es combinar la interpretabilidad y eficiencia de parámetros de la KAN 
en la etapa de características con la robustez de clasificación del MLP.

\subsection{Implementación y resultados}

El entrenamiento se realizó utilizando \textit{TensorFlow y Keras} con el 
optimizador Adam ($\eta=5 \times 10^{-4}$) y una función de pérdida de entropía 
cruzada categórica. Se empleó \textit{Early Stopping} con una paciencia de 5 
épocas para evitar el sobreajuste.

\subsubsection{Análisis Comparativo}
La Tabla \ref{tab:kan_results} muestra el rendimiento de ambas arquitecturas en 
el conjunto de prueba de MNIST.

\begin{table}[H]
    \centering
    \begin{tabular}{|l| c c c c|}
        \toprule
        \textbf{Arquitectura} & \textbf{Accuracy} & \textbf{Precision} 
        & \textbf{Recall} & \textbf{F1-Score} \\
        \midrule
        KAN Pura (Splines) & 0.7207 & 0.7109 & 0.7130 & 0.7080 \\
        \textbf{KAN Híbrida (Splines + MLP)} & \textbf{0.8803} & \textbf{0.8770} 
        & \textbf{0.8775} & \textbf{0.8770} \\
        \bottomrule
    \end{tabular}
    \caption{Resultados de clasificación en MNIST, en el set de evaluación, 
    para arquitecturas KAN.}
    \label{tab:kan_results}
\end{table}

\textbf{Interpretación:}
\begin{itemize}
    \item La \textit{KAN Pura} alcanzó una exactitud del 72\%, lo cual es bueno 
    considerando que carece de conexiones densas que mezclen globalmente la 
    información. Sin embargo, su incapacidad para modelar interacciones cruzadas 
    complejas entre píxeles distantes (algo que una matriz de pesos hace 
    naturalmente) limita su rendimiento en imágenes.
    \item La \textit{KAN Híbrida} logró un salto significativo al 88\% de exactitud. 
    Esto valida la hipótesis de que las transformaciones B-Spline actúan como 
    extractores de características no lineales efectivos, que luego pueden ser 
    clasificados eficientemente por capas densas.
\end{itemize}

\textbf{Conclusiones preliminares:}
Las KANs clásicas ofrecen una nueva perspectiva sobre el aprendizaje de funciones, 
sustituyendo parámetros estáticos por funciones dinámicas. Si bien una KAN pura 
puede no ser el estado del arte para visión por computadora, donde las CNNs dominan; 
su estructura modular y matemática la hace una candidata ideal para ser «cuantizada» 
y adaptada a la computación cuántica, como se propone a continuación.

\subsection{Propuesta de Quantum KAN (Q-KAN)}

La transición hacia una \textbf{Quantum Kolmogorov-Arnold Network} (Q-KAN) no es 
solo una extensión académica, sino una solución potencial a los cuellos de botella 
de la versión clásica.

\subsubsection{Limitaciones de las KANs Clásicas}
\begin{enumerate}
    \item \textbf{Costo de evaluación de Splines:} Evaluar un B-spline requiere 
    operaciones polinómicas por tramos que son computacionalmente más costosas 
    que la simple multiplicación matricial de un MLP.
    \item \textbf{Problemas con la dimensionalidad:} En la práctica, encontrar 
    las funciones univariadas óptimas para datos de muy alta dimensión 
    (como imágenes de colisionadores) es un problema de optimización no convexo.
\end{enumerate}

\subsubsection{Propuesta de Arquitectura Q-KAN}
Proponemos sustituir las funciones de activación univariadas basadas en B-Splines 
por \textbf{Circuitos Cuánticos Variacionales (PQC)}.

\begin{figure}[H]
    \centering
    % 1. Definimos una caja para guardar el circuito cuántico
    % Esto evita errores de "nested tikzpicture"
    \newsavebox{\circuitbox}
    \sbox{\circuitbox}{%
        \begin{quantikz}[row sep=0.3cm, column sep=0.3cm, transparent]
            \lstick{$\ket{0}$} & \gate{R_x(x)} & \gate{R_y(\theta)} & \meter{\langle Z \rangle}
        \end{quantikz}
    }

    \begin{tikzpicture}
        % --- Nodos de la red neuronal (Izquierda y Derecha) ---
        \node[draw, circle, minimum size=0.8cm, thick] (Input) at (0,0) {$x_j^{(l)}$};
        \node[draw, circle, minimum size=0.8cm, thick] (Output) at (9,0) {$x_i^{(l+1)}$};
        
        % --- El circuito cuántico en el medio (Usando la caja guardada) ---
        % Nombramos a este nodo "Caja" para referenciarlo después
        \node[draw, dashed, thick, inner sep=8pt, rounded corners=3pt, 
              label={[yshift=0.2cm]above:\textbf{Quantum Edge $\phi(x)$}}] 
              (Caja) at (4.5,0) {\usebox{\circuitbox}};
        % --- Flechas de conexión ---
        % Conectamos dinámicamente desde el Input al borde Oeste de la Caja
        \draw[->, thick, >=stealth] (Input) -- (Caja.west);
        % Conectamos dinámicamente desde el borde Este de la Caja al Output
        \draw[->, thick, >=stealth] (Caja.east) -- (Output);
        % --- Anotación inferior ---
        \node[text=gray, font=\small] at (4.5, -1.5) {Sustituye al B-Spline clásico};
        
    \end{tikzpicture}
    
    \caption{Arquitectura de una arista en la Q-KAN. La función de activación 
    univariada $\phi(x)$ se implementa mediante un circuito cuántico variacional 
    (PQC) que codifica la entrada en una rotación $R_x$ y aprende los parámetros 
    en $R_y$.}
    \label{fig:quantum_kan_sketch}
\end{figure}

En esta arquitectura (ver Fig. \ref{fig:quantum_kan_sketch}), la transformación 
$\phi(x)$ en cada arista de la red se realiza mediante un pequeño circuito cuántico:

\begin{equation}
    \phi_Q(x; \boldsymbol{\theta}) = \langle 0 | U^\dagger(x, \boldsymbol{\theta}) 
    M U(x, \boldsymbol{\theta}) | 0 \rangle
\end{equation}

Donde:
\begin{itemize}
    \item $U(x, \boldsymbol{\theta})$ es un circuito que codifica el dato de 
    entrada $x$ (mediante rotaciones $R_x(x)$) y aplica capas variacionales 
    parametrizadas por $\boldsymbol{\theta}$.
    \item $M$ es un observable, por ejemplo $\sigma_z$.
    \item La salida del circuito es el valor escalar transformado.
\end{itemize}

\textbf{Implementación Esquemática en PennyLane:}
\begin{lstlisting}[language=Python, caption=Pseudocódigo para una capa Q-KAN usando PQC.]
@qml.qnode(dev, interface='tf')
def quantum_activation(inputs, weights):
    # Codificación de datos
    for i in range(n_qubits):
        qml.RX(np.pi * inputs[i], wires=i)
    
    # Capa Variacional (sustituye al Spline)
    for i in range(n_qubits):
        qml.RY(weights[i], wires=i)
    
    # Entrelazamiento (Captura correlaciones)
    for i in range(n_qubits-1):
        qml.CNOT(wires=[i, i+1])
        
    return [qml.expval(qml.PauliZ(i)) for i in range(n_qubits)]
\end{lstlisting}

\subsubsection{Ventajas Esperadas}
\begin{itemize}
    \item \textbf{Expresividad en el espacio de Hilbert:} 
    Un PQC puede generar funciones de activación con series de Fourier mucho más 
    ricas que un simple spline polinómico, capturando frecuencias más altas de 
    los datos con menos parámetros.
    \item \textbf{Paralelismo Cuántico:} La evaluación de las funciones $\phi$ en 
    superposición podría ofrecer ventajas de velocidad en hardware cuántico real.
\end{itemize}

\section{Conclusiones finales}

El desarrollo de las tareas de evaluación para GSoC 2025 ha permitido una 
exploración profunda en la intersección de la inteligencia artificial y la 
computación cuántica.
\begin{enumerate}
    \item Se demostró la competencia técnica en la manipulación de estados 
    cuánticos mediante \textbf{Cirq y PennyLane}, herramientas esenciales para 
    el diseño de algoritmos.
    \item La implementación de \textbf{Graph Neural Networks, GNNs,} para la 
    clasificación de jets confirmó que respetar la geometría intrínseca de los 
    datos físicos mejora significativamente la precisión frente a métodos 
    agnósticos a la estructura.
    \item Finalmente, la exploración de las \textbf{Redes Kolmogorov-Arnold (KANs)} 
    reveló un camino prometedor hacia modelos más interpretables. La propuesta 
    original de esta tesis, la \textbf{Q-KAN}, busca unificar estos mundos: 
    utilizar la estructura topológica de las KANs para mitigar la complejidad de 
    los circuitos cuánticos profundos, ofreciendo una arquitectura híbrida robusta 
    para la física de altas energías en la era NISQ.
\end{enumerate}

%--- CONCLUSIONES ---
\chapter{Conclusiones generales y perspectivas}

El periodo de prácticas profesionales realizado en el Centro Interdisciplinario de 
Investigación y Enseñanza de la Ciencia (CIIEC) de la BUAP ha representado una 
etapa fundamental en la transición de la formación académica teórica hacia la 
investigación aplicada de frontera. A través del desarrollo de las tareas 
técnicas para la organización \textit{Machine Learning for Science} (ML4SCI) y la 
exploración de algoritmos cuánticos, se ha logrado cumplir satisfactoriamente 
con los objetivos planteados al inicio del programa.

\section{Cumplimiento de Objetivos}
La integración de herramientas de inteligencia artificial y computación cuántica 
permitió abordar problemas complejos de la física de altas energías desde una 
perspectiva moderna, logrando los siguientes hitos:

\begin{itemize}
    \item \textbf{Dominio de herramientas cuánticas:} Se logró no solo la simulación 
    de circuitos básicos, sino la comprensión profunda y replicación del Algoritmo 
    de Shor. Esto valida la competencia técnica para manipular los marcos de 
    trabajo líderes en la industria, como Qiskit, Cirq, PennyLane; y entender sus 
    diferencias operativas en la era NISQ.
    
    \item \textbf{Modernización del análisis de datos en física:} La implementación 
    de Redes Neuronales de Grafos para la clasificación de jets demostró que 
    el respeto a la estructura geométrica de los datos físicos resulta en modelos 
    más precisos y robustos que los enfoques tradicionales. Los resultados obtenidos 
    con la arquitectura \textit{Graph Attention Network} (GAT) confirmaron la 
    importancia de los mecanismos de atención para filtrar el ruido de fondo en 
    colisionadores.
    
    \item \textbf{Innovación con arquitecturas híbridas:} La exploración de las 
    Redes Kolmogorov-Arnold y la propuesta teórica de su versión cuántica, Q-KAN, 
    sitúan a este trabajo en la vanguardia del estado del arte, proponiendo 
    soluciones novedosas a los problemas de interpretabilidad y expresividad que 
    enfrentan los modelos actuales.
\end{itemize}

\section{Reflexión Profesional}
Más allá de los logros técnicos, esta estancia permitió desarrollar competencias 
transversales críticas: la capacidad de autoaprendizaje ágil ante tecnologías 
emergentes, la gestión de proyectos de código abierto bajo estándares internacionales 
y la habilidad para comunicar conceptos físicos complejos a través de implementaciones 
de software.

La colaboración dentro del grupo de computación cuántica del CIIEC evidenció que la 
física moderna es intrínsecamente interdisciplinaria. La frontera entre el físico 
teórico y el científico de datos es delgada, exigiendo un perfil híbrido capaz de 
formular las preguntas correctas desde la teoría y responderlas mediante algoritmos 
avanzados.

\section{Perspectivas Futuras}
Este reporte no constituye un punto final, sino la base metodológica para el 
trabajo de tesis de licenciatura en curso. Las líneas de investigación abiertas, 
particularmente la implementación física de la \textbf{Quantum KAN} y la 
optimización de las \textbf{GNNs} para el Gran Colisionador de Hadrones, definen 
la ruta de trabajo para los próximos meses.

En conclusión, las prácticas profesionales han cumplido su propósito: consolidar 
las herramientas técnicas y la madurez científica necesarias para contribuir 
activamente a la investigación en la intersección de la física de partículas y la 
inteligencia artificial cuántica.




%--- REFERENCIAS ---
\bibliographystyle{plain}
\bibliography{bibliografia}

\end{document}
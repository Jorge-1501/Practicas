\chapter{Introducción}

El presente documento detalla las actividades de investigación y desarrollo 
tecnológico realizadas durante el periodo de prácticas profesionales en el 
\textbf{Centro Interdisciplinario de Investigación y Enseñanza de la Ciencia, CIIEC,}
de la Benemérita Universidad Autónoma de Puebla (BUAP). Este periodo comprendió del 
6 de enero al 6 de julio de 2025.

Si bien el programa se tituló oficialmente «Análisis de Datos enfocado a física de 
partículas», la dinámica de trabajo y los intereses del grupo de investigación 
permitieron una evolución natural hacia áreas de vanguardia. Las actividades se 
centraron en el grupo de computación cuántica, explorando la intersección crítica 
entre el aprendizaje automático, \textit{Machine Learning}, la criptografía 
moderna y el diseño de algoritmos cuánticos para la física de altas energías (HEP).

El repositorio donde se encuentran los resultados de los proyectos realizados durante
las prácticas profesionales está disponible en:
\begin{center}
    \url{https://github.com/Jorge-1501/Practicas}
\end{center}

\section{Contexto: La convergencia entre IA y Física}
La física contemporánea enfrenta un desafío por la cantidad de datos generados. 
Los colisionadores, como el LHC, generan petabytes de información que los 
métodos tradicionales de análisis tardan años en procesar. En este contexto, la 
Inteligencia Artificial, IA, se ha convertido en un pilar fundamental para el 
descubrimiento científico.

Paralelamente, la Computación Cuántica ha entrado en la era NISQ, \textit{Noisy 
Intermediate-Scale Quantum}. La promesa de esta tecnología no solo radica en la 
aceleración de cálculos, como se verá con el algoritmo de Shor; sino en su 
capacidad para modelar sistemas naturales de manera innata, como la mecánica cuántica. 
La fusión de estos campos, conocida como \textit{Quantum Machine Learning}, QML, 
busca aprovechar el espacio de Hilbert para procesar información, que para las 
redes neuronales clásicas sería inalcanzable. Este reporte explora precisamente 
esa frontera: cómo las herramientas de IA geométrica y cuántica pueden aplicarse a 
problemas fundamentales como la clasificación de jets y la optimización de funciones.

\section{Descripción del CIIEC y el Programa}
El Centro Interdisciplinario de Investigación y Enseñanza de la Ciencia, CIIEC, es 
una unidad académica de la BUAP dedicada a la investigación de frontera y la formación 
de recursos humanos especializados. El ambiente multidisciplinario del centro propició 
un espacio ideal para integrar conceptos de física teórica con ciencias de la computación.

Durante el programa, se trabajó bajo un esquema de mentoría semanal, donde el enfoque, 
más allá de la aplicación de herramientas existentes, fue la comprensión 
profunda de los fundamentos matemáticos detrás de las arquitecturas de redes 
neuronales y los circuitos cuánticos.

\section{Objetivos Generales}
El propósito principal de la estancia fue desarrollar competencias avanzadas en 
computación científica. Los objetivos específicos se desglosaron de la siguiente 
manera:

\begin{enumerate}
    \item \textbf{Investigación del estado del arte:} Realizar una revisión 
    bibliográfica exhaustiva sobre técnicas del machine Learning así como los modelos
    más usados al momento.
    \item \textbf{Implementación de modelos de IA:} Construir desde cero modelos de 
    aprendizaje profundo, incluyendo MLPs, CNNs y etc.
    \item \textbf{Implementación de Algoritmos Cuánticos:} Estudiar y replicar 
    algoritmos fundamentales, comprendiendo sus implicaciones y su implementación en 
    simuladores reales.
    \item \textbf{Desarrollo de pruebas técnicas para ML4SCI:} Resolver los desafíos 
    de código propuestos por la organización \textit{Machine Learning for Science} 
    (ML4SCI) para el programa Google Summer of Code (GSoC), abarcando desde circuitos 
    básicos hasta nuevas arquitecturas híbridas.
\end{enumerate}

%--- CONCLUSIONES ---
\chapter{Conclusiones generales y perspectivas}

El periodo de prácticas profesionales realizado en el Centro Interdisciplinario de 
Investigación y Enseñanza de la Ciencia (CIIEC) de la BUAP ha representado una 
etapa fundamental en la transición de la formación académica teórica hacia la 
investigación aplicada de frontera. A través del desarrollo de las tareas 
técnicas para la organización \textit{Machine Learning for Science} (ML4SCI) y la 
exploración de algoritmos cuánticos, se ha logrado cumplir satisfactoriamente 
con los objetivos planteados al inicio del programa.

\section{Cumplimiento de Objetivos}
La integración de herramientas de inteligencia artificial y computación cuántica 
permitió abordar problemas complejos de la física de altas energías desde una 
perspectiva moderna, logrando los siguientes hitos:

\begin{itemize}
    \item \textbf{Dominio de herramientas cuánticas:} Se logró no solo la simulación 
    de circuitos básicos, sino la comprensión profunda y replicación del Algoritmo 
    de Shor. Esto valida la competencia técnica para manipular los marcos de 
    trabajo líderes en la industria, como Qiskit, Cirq, PennyLane; y entender sus 
    diferencias operativas en la era NISQ.
    
    \item \textbf{Modernización del análisis de datos en física:} La implementación 
    de Redes Neuronales de Grafos para la clasificación de jets demostró que 
    el respeto a la estructura geométrica de los datos físicos resulta en modelos 
    más precisos y robustos que los enfoques tradicionales. Los resultados obtenidos 
    con la arquitectura \textit{Graph Attention Network} (GAT) confirmaron la 
    importancia de los mecanismos de atención para filtrar el ruido de fondo en 
    colisionadores.
    
    \item \textbf{Innovación con arquitecturas híbridas:} La exploración de las 
    Redes Kolmogorov-Arnold y la propuesta teórica de su versión cuántica, Q-KAN, 
    sitúan a este trabajo en la vanguardia del estado del arte, proponiendo 
    soluciones novedosas a los problemas de interpretabilidad y expresividad que 
    enfrentan los modelos actuales.
\end{itemize}

\section{Reflexión Profesional}
Más allá de los logros técnicos, esta estancia permitió desarrollar competencias 
transversales críticas: la capacidad de autoaprendizaje ágil ante tecnologías 
emergentes, la gestión de proyectos de código abierto bajo estándares internacionales 
y la habilidad para comunicar conceptos físicos complejos a través de implementaciones 
de software.

La colaboración dentro del grupo de computación cuántica del CIIEC evidenció que la 
física moderna es intrínsecamente interdisciplinaria. La frontera entre el físico 
teórico y el científico de datos es delgada, exigiendo un perfil híbrido capaz de 
formular las preguntas correctas desde la teoría y responderlas mediante algoritmos 
avanzados.

\section{Perspectivas Futuras}
Este reporte no constituye un punto final, sino la base metodológica para el 
trabajo de tesis de licenciatura en curso. Las líneas de investigación abiertas, 
particularmente la implementación física de la \textbf{Quantum KAN} y la 
optimización de las \textbf{GNNs} para el Gran Colisionador de Hadrones, definen 
la ruta de trabajo para los próximos meses.

En conclusión, las prácticas profesionales han cumplido su propósito: consolidar 
las herramientas técnicas y la madurez científica necesarias para contribuir 
activamente a la investigación en la intersección de la física de partículas y la 
inteligencia artificial cuántica.